\documentclass[12pt]{article}
\usepackage[T2A]{fontenc}
\usepackage[utf8]{inputenc}
\usepackage{multirow}
\usepackage{caption}
\usepackage{subcaption}
\usepackage{amsmath}
\usepackage{amssymb}
\usepackage{changepage}
\usepackage{graphicx}
\usepackage{float}
\usepackage[english,russian]{babel}
\usepackage{amsmath, amsfonts, amssymb, amsthm, mathtools}
\usepackage{xcolor}
\usepackage{array}
\usepackage{hyperref}
\usepackage{physics}
\usepackage[top = 1.5cm, left = 1.5 cm, right = 1.5 cm, bottom = 3 cm]{geometry}
\usepackage{import}
\usepackage{xifthen}
\usepackage{pdfpages}
\usepackage{transparent}

\newcommand{\incfig}[1]{
    \import{./figures/}{#1.pdf_tex}
}

\title{Практика 9.}
\author{Шахматов Андрей, Б02-304}
\date{\today}

\begin{document}
\maketitle
\tableofcontents

\section{1.1 Сдано}
Если $x \in A \triangle C$, то он принадлежит либо в $A \setminus C$, либо в $C \setminus A$. Тогда без ограничения общности 
$x \in A$ и $x \not \in C$. Тогда если $x \in B$, то он принадлежит $B \triangle C \implies x \in (A \triangle B) \cup (B \triangle C)$, иначе 
$x \in (A \triangle B)$.    
\section{1.2 Сдано}
Так как мера множества равна нулю, то существует покрытие элементарными $X \in A = \bigcup_{k=1}^{\infty} A_k$ , такое, что: 
\[
    \sum_{k=1}^{\infty} m(A_k) < \varepsilon
\]  
Возьмём множество $A_1$: 
\[
    \mu^{\ast}(X \triangle A_1) \leq \mu(A_1 \triangle A) + \mu(A \triangle X) < 2\varepsilon 
\]
\section{1.3 Сдано}
Если мера Лебега равна $0$, то его внешняя мера Лебега тоже равна $0$, 
тогда по субаддитивности верхней меры лебега: 
\[
    \mu^{\ast}(X_1 \subset X) \leq \mu^{\ast}(X) = 0
\]   
Тогда по предыдущей задаче $X_1$ - измеримо. 
\section{1.4}
\[
    \left\vert \mu(A) - \mu(B) \right\vert \leq \mu (A \triangle B) = 0
\]
Тогда $\mu(A) = \mu(B)$. А дальше как... 
\section{2.1 Сдано}
а) Данное множество соответствует графику функции $y(x) = 1 - x$ на множестве $[0, 1]$. 
Тогда так как такая функция равномерно непрерывна, то для всякого $\varepsilon > 0$ найдётся $\delta > 0$, 
такая, что $\vert f(x_1) - f(x_2) \vert < \varepsilon, \, \vert x_1 - x_2 \vert  < \delta$. Разобъём 
множество $[0, 1]$ на дельта промежутки $[x_k, x_{k+1}], x_{k+1} - x_k < \delta$, 
тогда весь график покрывается $[x_k, x_{k+1}] \times [f(x_k), f(x_{k+1})]$, причём верхняя мера Жордана 
такого покрытия:
\[
    \mu^{\ast} X \leq \sum_{k=1}^{n} [x_k, x_{k+1}] \cdot \varepsilon = \varepsilon  
\]     
Тогда так как верхняя мера сколь угодно мала, то мера множества равна $0$. Значит множество 
измеримо по Жорадну и по Лебегу. 
б) Граница такого множества - весь квадрат, его мера не равна 0, значит множество не измеримо по Жордану. 
Представленное множество можно представить как счётное объединение множеств: 
\[
    X = \bigcup_{q \in \mathbb{Q}} \left\{ y = q - x \mid (x, y) \in [0, 1]^2 \right\}  
\]
Каждое из таких множеств представляет двумерную прямую, то есть имеет Лебегову меру $0$. А значит 
по суббаддитивности верхней меры Лебега множество $X$ имеет меру $0$, а значит множество измеримо по Лебегу 
с мерой $0$.

\section{2.2 Сдано}
Построим такое множество, разделим отрезок на 10 частей и выбросим из него 3 часть, тогда в полученном 
множестве $F_1$ не будет чисел с 4 в первом разряде. Далее из каждой из 9 оставшихся частей проведём аналогичную операцию - 
в полученном множестве $F_2$ не будет чисел с 4 в первом и втором разряде. Тогда множество:
\[
    F = \bigcap_{n=1}^{\infty} F_n
\]  
не будет иметь 4 в своей десятичной записи. Такое множество измеримо по Лебегу, так как является 
пересечением измеримых множеств. При этом мера $\mu F_n = \frac{9}{10} \mu F_{n-1}$, тогда 
$\mu F_n = \left( \frac{9}{10} \right)^n \to 0, n \to \infty$, тогда из непрерывности меры Лебега 
$\mu F = 0$. Так как множество нигде не плотно, то его внутренность пустая, тогда мера Лебега его 
границы равна $0$. Так как такая граница компактна, то она также имеет нулевую меру Жордана. А значит измеримо по 
Жордану. 
\section{2.3}
Докажем по индукции, база индукции очевидна. Для начала покажем, с учётом $\mu(X_k) \leq 1$: 
\[
    \sum_{k=1}^{n} X_k > n - 1 \implies \sum_{k=1}^{n - 1} X_k > n - 1 - \mu(X_n) > (n - 1) - 1
\]
То есть условие сохраняется при шаге индукции. Тогда докажем от противного, пусть $\sum_{k=1}^{n} X_k > n - 1$
и $\mu \left( \bigcap_{k=1}^{n} X_k \right)  = \mu (X \cap X_n) = 0$: 
\[
    \mu(X \cap X_n) + \mu(X \cup X_n) = \mu(X) + \mu(X_n) > n - 1 \implies 1 \geq \mu(X \cup X_n) > n - 1  
\]     
Получили противоречие для $n > 1$. 
\section{2.4}
Представленное множество монжно записать как:
\[
    X = \bigcap_{n=1}^{\infty} \left( \bigcup_{k=n}^{\infty} X_k \right) 
\] 
Тогда помтроим последовательность вложенных множеств: 
\[
    \bigcup_{k=1}^{\infty} X_k \supset \bigcup_{k=2}^{\infty} X_k \supset \dots \supset \bigcup_{k=n}^{\infty} X_k
\]
Тогда по непрерывности меры: 
\[
    \mu(X) = \lim_{n \to \infty} \mu \left( \bigcup_{k=n}^{\infty} X_k \right) \leq \lim_{n \to \infty} \sum_{k=n}^{\infty} \mu(X_k) = 0
\]


\section{2.5}
Из-за $K \subset X \subset U \implies \mu^{\ast} (X \setminus K) = \mu^{\ast} (X \triangle K) < \varepsilon$ 
Разобъём открытое множество на объединение открытых шаров с центрами в рациональных точках: 
\[
    U = \bigcup_{k=1}^{\infty} A_k =  \bigcup_{k=1}^{\infty} (x_k - \frac{1}{n} dist(x_k - \mathbb{R}^n \setminus U), x_k - \frac{1}{n} dist(x_k - \mathbb{R}^n \setminus U))^{\otimes n}
\]
Тогда так как $K \subset U$, то наше разбиение покрывает $K$, тогда выберем счётное подпокрытие по определению 
компактности. Тогда полученное $A = A_1 \cup A_2 \cup \dots \cup A_N$ по монотонности меры: 
\[
    \mu(K) \leq \mu(A) \leq \mu(U)
\]
С учётом $\mu(U \setminus K) < \varepsilon$ получим $\mu(U \setminus A) < \varepsilon$ и $\mu(A \setminus K) < \varepsilon$. 
Каждое из шаров $A_k$ измеримо и потому приближается с точностью $\frac{\varepsilon}{2^k}$ элементарными $P_k$, $P = \bigcup_{k=1}^{N} P_k$: 
\[
    \mu(A \triangle P) \leq \sum_{k=1}^{N} \mu(A_k \triangle P_k) < \varepsilon
\]
Тогда: 
\[
    \mu^{\ast} (X \triangle P) \leq \mu^{\ast} (X \triangle K) + \mu^{\ast} (K \triangle A) + \mu^{\ast} (A \triangle P) < 3 \varepsilon.
\]
\section{3.1}
По регулярности меры найдём открытое $U$, в котором содердится $X$, и выполняется $\mu(U \setminus X) < \varepsilon$. 
Тогда так как $U$ - открытое, то оно представляется как 
\[
    U = \bigsqcup_{k=1}^{\infty} (a_k, b_k)
\]    
Тогда запишем условие задачи для каждого из интервалов:
\[
    \mu(X \cap (a_k, b_k)) \leq \frac{b_k - a_k}{2}
\] 
Тогда счётно просуммировав неравенства получим: 
\[
    \mu(X \cap U) \leq \frac{\mu (U)}{2}
\]
Тогда получаем: 
\[
    \mu(U) = \mu(X \cap U) + \mu(U \setminus X) < \frac{\mu (U)}{2} + \varepsilon \implies 
    \mu(U) < 2\varepsilon \implies \mu(X) < 2\varepsilon.
\]
Так как $\varepsilon$ выбиралось произвольное, то $\mu (X) = 0$.  


\section{3.2}
Рассмотрим произвольное $\alpha$. Построим множество Витали $V$ на отрезке $[0, \alpha]$, его внешняя мера
$\mu^{\ast}(V) = \beta \leq \alpha$. Тогда дополним множество $V$ до $V^{\prime} = V \sqcup (\beta - \alpha - 1, -1)$. 
Так как множество покрытий множества витали непересекается с множеством покрытий $(\beta - \alpha - 1, -1)$, то 
их внешние меры суммируются. Также полученное множество не может быть измеримым. Тогда мы нашли неизмеримое 
$V^{\prime}$ с мерой $\mu^{\ast}(V^{\prime}) = \beta + (\alpha - \beta) = \alpha$. 
P.S Я знаю что дополнение у множеству Витали на отрезке $[0, \alpha]$ имеет внешнюю меру $\alpha$, но я не знаю как 
это доказать.    

\section{3.3}
Докажем, что из того, что множество $X$ измеримо, то $\forall A$ выполняется: 
\[
    \mu^{\ast}(A) = \mu^{\ast}(A \setminus X) + \mu^{\ast} (A \cap X)
\]
По теореме (5.31) найдём измеримое множество $B$, 
содержащее $A$, такое, что $\mu^{\ast}(A) = \mu(B)$: 
\[
    \mu^{\ast}(A) = \mu(B) = \mu(B \cap X) + \mu(B \setminus X) = 
    \mu^{\ast} (B \cap X) + \mu^{\ast} (B \setminus X) \geq 
    \mu^{\ast} (A \cap X) + \mu ^{\ast} (A \setminus X) 
\]
Покажем неравенство в обратную сторону по суббаддитивности внешней меры: 
\[
    \mu^{\ast} (A \cap X) + \mu ^{\ast} (A \setminus X) \geq \mu^{\ast}\left( (A \cap X) \cup (A \setminus X) \right) = 
    \mu^{\ast} (A)
\]
Теперь докажем, что из выполнения равенства для любого $A$ следует измеримость $X$.
В другую сторону не знаю). 



\end{document}