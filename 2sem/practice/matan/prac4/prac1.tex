\documentclass[12pt]{article}
\usepackage[T2A]{fontenc}
\usepackage[utf8]{inputenc}
\usepackage{multirow}
\usepackage{caption}
\usepackage{subcaption}
\usepackage{amsmath}
\usepackage{amssymb}
\usepackage{changepage}
\usepackage{graphicx}
\usepackage{float}
\usepackage[english,russian]{babel}
\usepackage{amsmath, amsfonts, amssymb, amsthm, mathtools}
\usepackage{xcolor}
\usepackage{array}
\usepackage{hyperref}
\usepackage{physics}
\usepackage[top = 1.5cm, left = 1.5 cm, right = 1.5 cm, bottom = 3 cm]{geometry}
\usepackage{import}
\usepackage{xifthen}
\usepackage{pdfpages}
\usepackage{transparent}

\newcommand{\incfig}[1]{
    \import{./figures/}{#1.pdf_tex}
}

\title{Практика 4.}
\author{Шахматов Андрей, Б02-304}
\date{\today}

\begin{document}
\maketitle
\tableofcontents

\section{1.1}
а) 
\[
    \sum_{n=1}^{\infty} \frac{3 + 2 \sin n}{(2 + (-1)^n)\sqrt{n}} > \sum_{n=1}^{\infty} \frac{1}{\sqrt{n}}
\]
Так как ряд $\sum_{n=1}^{\infty} \frac{1}{\sqrt{n} }$ расходится, то и исходный ряд расходится. 
\\ б) 
\[
    \sum_{n=1}^{\infty} \left( 1 - \cos \frac{2\pi}{n} \right) = \sum_{n=1}^{\infty} \left( 1 - 1 + \frac{4\pi^2}{n^2} + O(\frac{1}{n^3}) \right) 
\]
Так как $\sum_{n=1}^{\infty} O(\frac{1}{n^3})$ и $\sum_{n=1}^{\infty} \frac{4\pi^2}{n^2}$ сходятся абсолютно, то и исходый ряд сходится.  
\section{1.2}
а)
\[
    \sum_{n=1}^{\infty} \frac{2^n n!}{n^n}
\]
Воспользуемся критерием Даламбера:
\[
    q = \lim_{n \to \infty}  \frac{2^{n+1}}{2^n} \frac{(n+1)!}{n!} \frac{n^n}{(n+1)^{n+1} } = 2 \lim_{n \to \infty} \frac{1}{\left( 1 + \frac{1}{n} \right)^{n+1} } = \frac{2}{e} < 1
\]
Ряд сходится.
\\
б)
\[
    \sum_{n=1}^{\infty} 3^n \left( \frac{n}{n+1} \right)^{n^2} 
\]
Воспользуемся критерием Коши:
\[
    q = \lim_{n \to \infty} 3 \left( \frac{1}{1 + \frac{1}{n}} \right)^n = \frac{3}{e} > 1 
\]
Ряд расходится. 
\section{1.3}
\[
    \sum_{n=1}^{\infty} \frac{(-1)^n \sqrt{n} }{2n - 1}
\]
Рассмотрим $a_n = \frac{\sqrt{n} }{2n - 1}$:
\[
    a_n = \frac{1}{2\sqrt{n} - \frac{1}{\sqrt{n} }} \to 0, n \to \infty 
\] 
Также проверим монотонность введя функцию $f(x) = \frac{\sqrt{x} }{2x - 1}$:
\[
    f^{\prime}(x) = -\frac{2x + 1}{2\sqrt{x}(2x-1)^2}
\]
Производная при $x > 1$ меньше 0, а значит функция убывает, т.е $a_{n+1} < a_n \forall n > 1$. 
Тогда по теореме Лейбница ряд сходится. 
Проверим ряд на абсолютную сходимость:
\[
    \sum_{n=1}^{\infty} \frac{\sqrt{n} }{2n - 1} > \sum_{n=1}^{\infty} \frac{1}{2n}
\]
А значит ряд расходится, т.е нет абсолютной сходимости.
\section{2.1}
\[
    \sum_{n=1}^{\infty} \frac{1}{a_n},
\]
где $a_n$ - числа Фибоначчи. Воспользуемся признаком Даламбера:
\[
    q = \lim_{n \to \infty} \frac{a_{n+1} }{a_{n}}
\] 
Докажем, что последовательность $\frac{a_n}{a_{n+1} }$ убывает: 
\[
    \frac{a_n^2 }{a_{n+1} a_{n-1}} < 1 
\]
Т.е $a_n^2 < (a_n + a_{n-1} ) a_{n-1} \implies t^2 < t + 1$, t = $\frac{a_n}{a_{n-1} }$, такое неравентсво выполняется при $t < \phi$, т.е 
для любого числа Фибоначчи. Тогда последовательность $\frac{a_n}{a_{n+1} }$ убывает и ограничена 0 снизу, а значит у неё есть предел. Тогда 
пусть $\frac{1}{q} = \lim_{n \to \infty} \frac{a_n}{a_{n+1}}$:
\[
    \frac{a_{n+1} }{a_n} = 1 + \frac{a_n}{a_{n-1} } \implies q = 1 + \frac{1}{q} \implies q = \phi > 1
\]
А значит исходный ряд расходится. 
\section{2.2}
\[
    \sum_{n=1}^{\infty} \left( 1 + \frac{\alpha \ln n}{n} \right)^n = 
    \sum_{n=1}^{\infty} 1 + \alpha \ln n + O(\frac{\ln ^2 n}{n^2})
\]
То есть ряд расходится для любого $\alpha$ 
\section{2.3}
a)
\[
    \sum_{n=1}^{\infty} \frac{\cos n}{n^\alpha }
\]
При $\alpha \leq 0$ ряд расходится так как невыполняется необходимое условие сходимости (Ряд косинусов тоже не сходится). 
При $\alpha > 0$ ряд сходится условно по тригонометрическому признаку.
Проверим ряд на абсолютную сходимость: 
\[
    \sum_{n=1}^{\infty} \frac{\vert \cos n \vert }{n^\alpha} \geq  \sum_{n=1}^{\infty} \frac{\cos^2 n}{n^\alpha} = 
    \sum_{n=1}^{\infty} \left( \frac{1}{2n^\alpha } + \frac{\cos 2n}{2n^\alpha} \right) 
\]
Тогда характер сходимости такого ряда задаётся его первой подсуммой, т.е ряд расходится если расходится сумма 
$\frac{1}{2n^\alpha }$, т.е при $\alpha \leq 1$. При $\alpha > 1$ ряд очевидно сходится абсолютно.
Ответ: при $a \in (0, 1]$ ряд сходится условно, при $a > 1$ ряд сходится абсолютно.  
б) 
\[
    \sum_{n=1}^{\infty} \frac{\sin 2n \cdot \ln ^2 n}{n^\alpha}
\]
Условная сходимость существует, когда $\frac{\ln^2 n}{n^\alpha }$ монотонно стремится к 0, т.е при $\alpha > 0$.
Проверим ряд на абсолютную сходимость: 
\[
    \sum_{n=1}^{\infty} \frac{\ln^2 n}{n^\alpha } \geq \sum_{n=1}^{\infty} \frac{\vert \sin 2n \vert \ln^2 n}{n^\alpha} \geq \sum_{n=1}^{\infty} \left( \frac{\ln^2 n}{2n^\alpha } - \frac{\sin 4n \ln^2 n}{n^\alpha} \right) 
\]
Тогда абсолютная сходимость ряда определяется сходимостью ряда $\sum_{n=1}^{\infty} \frac{\ln^2 n}{n^\alpha }$, который является эталонным рядом и сходится при $\alpha > 1$.  
\section{2.4}
Пусть $R_n = \sum_{k=n}^{\infty} a_k$, тогда $a_n = R_n - R_{n+1}$, тогда:
\[
    \left| \sum_{k=n}^{m} a_k b_k \right| = \left| R_n b_n - R_{m+1} b_m + \sum_{k=n+1}^{m} R_k(b_k - b_{k-1}) \right|
\]
Так как $\sum_{n=1}^{\infty} a_n$ сходится, то $R_n \to 0, n \to 0$. Также в силу абсолютной сходимости $b_n$ выполняется 
$\sum_{k=n+1}^{m} \vert b_k \vert < M$. 
\[
    \left| \sum_{k=n}^{m} a_k b_k \right| \leq \varepsilon \left( \vert b_n \vert + \vert b_m \vert + \sum_{k=n+1}^{m} \vert b_k \vert + \sum_{k=n+1}^{m} \vert b_{k-1} \vert \right) \leq  4 \varepsilon M 
\]

\section{2.5}
\[
    \lim_{n \to \infty} \left( \sum_{k=1}^{n} \frac{1}{k} - \ln n \right) = \lim_{n \to \infty} 1 + \sum_{k=2}^{n} \left( \frac{1}{k} - \ln \frac{k}{k-1} \right) = 
    1 + \sum_{n=2}^{\infty} \frac{1}{n} - \ln \left( \frac{1}{1 + \frac{1}{n}} \right) = 1 + \sum_{n=2}^{\infty} O(\frac{1}{n^2})  
\]
Так как сумма сходится, то исходный предел существует.

\section{3.2}


\section{3.3}
а) Пусть $a_n = \frac{(-1)^n}{\sqrt{n}}$, тогда сумма $\sum_{n=1}^{\infty} a_n$ сходится по признаку Лейбница. В то время как 
$\sum_{n=1}^{\infty} a_n^2 = \sum_{n=1}^{\infty} \frac{1}{n}$ - расходится. 
\\ б) Пусть $a_n = \frac{\cos \frac{2\pi}{3}n}{\sqrt[3]{n} }$. Сумма ряда с членами такой последовательности сходится. Но 
$a_n^3 = \frac{1}{4} \left( \frac{1}{n} + \frac{3\cos \frac{2\pi}{3}n}{n} \right)$, первая подсумма последовательности расходится а вторая сходится, 
значит сумма $\sum_{n=1}^{\infty} a_n^3$ расходится.  
При $a_n \geq 0$ ряды сходятся абсолютно, а значит их произведение само на себя также сходится. 




\end{document}