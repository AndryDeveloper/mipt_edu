\documentclass[12pt]{article}
\usepackage[T2A]{fontenc}
\usepackage[utf8]{inputenc}
\usepackage{multirow}
\usepackage{caption}
\usepackage{subcaption}
\usepackage{amsmath}
\usepackage{amssymb}
\usepackage{changepage}
\usepackage{graphicx}
\usepackage{float}
\usepackage[english,russian]{babel}
\usepackage{amsmath, amsfonts, amssymb, amsthm, mathtools}
\usepackage{xcolor}
\usepackage{array}
\usepackage{hyperref}
\usepackage{physics}
\usepackage[top = 1.5cm, left = 1.5 cm, right = 1.5 cm, bottom = 3 cm]{geometry}
\usepackage{import}
\usepackage{xifthen}
\usepackage{pdfpages}
\usepackage{transparent}

\newcommand{\incfig}[1]{
    \import{./figures/}{#1.pdf_tex}
}

\title{Практика 8.}
\author{Шахматов Андрей, Б02-304}
\date{\today}

\begin{document}
\maketitle
\tableofcontents

\section{1.1}
Для равномерного разбиения сумма Римана имеет вид:
\[
    \int_{1}^{2} x \,\mathrm{d}x = \lim_{n \to \infty} \sum_{k=1}^{n} \left( 1 + \frac{k}{n} \right) \frac{1}{n} = 
    \lim_{n \to \infty} \frac{1}{2n} \left( 3n + 1 \right) = \frac{3}{2} 
\]
Геометрическая прогрессия потом. 
\section{1.2}
Рассмотрим $f(x) = 2D(x) - 1$, где $D(x)$ - функция Дирихле. Такая функция очевидно не интегрируема.  
\\а) Однако $\vert f(x) \vert \equiv 1$ - интегрируема. 
\\б) $\vert f^2(x) \vert \equiv 1$ - интегрируема. 
\\в) $g(x) = 1$ - интегрируема, и $0 \leq D(x) \leq 1$. 
\section{1.3}
Докажем через определение по Коши учитывая ограниченность функции $f$: 
\[
    \left\vert \int_{a}^{x_0} f(t) \,\mathrm{d}t - \int_{a}^{x} f(t) \,\mathrm{d}t \right\vert \leq 
    \left\vert \int_{x_0}^{x} f(t) \,\mathrm{d}t \right\vert \leq \left\vert  w_f([x_0, x])(x - x_0) \right\vert \leq 
    M \vert x - x_0 \vert 
\]    
Получили, что функция Липшицева, а значит непрерывна. 
\section{2.1}
\[
    \lim_{n \to \infty} n \sum_{k=1}^{n} \frac{1}{n^2 + k^2} = 
    \lim_{n \to \infty} \sum_{k=1}^{n} \frac{1}{1 + \frac{k^2}{n^2}} \frac{1}{n} = 
    \int_{0}^{1} \frac{1}{1 + x^2} \,\mathrm{d}x = \arctg 1 = \frac{\pi}{4} 
\]
\section{2.2}
В одну сторону:
\[
    \int_{0}^{\frac{3}{4}} \frac{2^x}{\sqrt{1 + x^2} } \,\mathrm{d}x > \int_{0}^{\frac{3}{4}} \frac{1 + x \ln 2}{\sqrt{1 + x^2}} \,\mathrm{d}x = \frac{5}{4} \ln 2 > \ln 2 
\]
В другую сторону: 
\[
    \int_{0}^{\frac{3}{4}} \frac{2^x}{\sqrt{1 + x^2} } \,\mathrm{d}x < \int_{0}^{\frac{3}{4}} 2^x \,\mathrm{d}x = \frac{1}{\ln 2} \left( \sqrt[4]{8} - 1 \right) < \frac{1}{\ln 2} 
\]

\section{2.3}
Из критерия интегрирования через взвешенные колебания рассмотрим произвольное 
$\varepsilon > 0$, выделим промежуток $\delta = \frac{\varepsilon}{4}$, тогда $w_f([0, \delta)) = 2 \cdot \frac{\varepsilon}{4} = \frac{\varepsilon}{2}$, 
Рассмотрим функцию на отрезке $[\delta, 1]$, на нём функция непрерывна, а значит и интегрируема, тогда 
существует такое разбиение $\tau$, что $\Omega(f_{[\delta, 1]}, \tau) < \frac{\varepsilon}{2}$, тогда полная 
взвешенная сумма колебаний, полченная объединением $\tau^{\prime} = \tau \cup [0, \delta)$: 
\[
    \Omega(f, \tau^{\prime}) < \frac{\varepsilon}{2} + \frac{\varepsilon}{2} = \varepsilon
\]      
Что означает интегрируемость исходной функции функции. 
\section{2.4}
а) Так как функция $g$ - непрерывно дифференцируема, то $g^{\prime}$ ограничена, а значит $\vert g^{\prime} \vert < M$. 
Тогда для любого колебания по теореме Лагранжа: 
\[
    w_{g \circ f}(\Delta) = \vert g^{\prime}(\xi) \vert w_f(\Delta) \leq M w_f(\Delta)
\]    
Из чего следует: 
\[
    \Omega(g \circ f, \tau) \leq M \Omega(f, \tau) \leq M \varepsilon
\]
\\б) 
По теореме 4.123 из "An explanations" если $f$ - интегрируема, то $\vert f \vert $ - тоже интегрируема. 
Остаётся доказать, если $h(x) = \vert f(x) \vert $ - интегрируема и $g(x) = x^p$, то $g \circ h$ - интегрируема. 
Так как $g(x)$ - непрерывно дифференцируема, то согласно пункту а) композиция интегрируема.        
\section{3.1}
Докажем для функции Римана на отрезке $[0, 1]$, а так как функция Римана периодична, то и для произвольного 
отрезка она будет интегрируема. 
Для любого $\varepsilon > 0$ тогда функция Римана принимает значение большее $\frac{\varepsilon}{2}$ конечное число раз, 
покроем все точки $x$ для которых $R(x) > \frac{\varepsilon}{2}$ семейством окрестностей $U_{\frac{\varepsilon}{4}}, U_{\frac{\varepsilon}{8}}, \dots$, 
тогда взвешенная сумма колебаний по таким окрестностям не превосходит: 
\[
    \Omega(f_U, \tau_U) \leq 1\cdot\frac{\varepsilon}{4} + 1\cdot\frac{\varepsilon}{8} + \dots < \frac{\varepsilon}{2}
\]   
В остальных точках значение функции Римана не превосходит $\frac{\varepsilon}{2}$, а значит взвешенная сумма колебаний 
не превосходит $1\cdot\frac{\varepsilon}{2}$, тогда взвешенная сумма колебаний по всему разбиению не превосходит $\varepsilon$.    
\section{3.2}
Мы уже доказывали, что такая функция интегрируема. Тогда остаётся доказать, что произовдная 
первообразной равна 0 в точке 0, т.е: 
\[
    \int_{0}^{x} f(x) \,\mathrm{d}x = o(x)
\]
А что дальше не понятно.
\section{3.3}
Предположим $\int_{a}^{b} f(x) \,\mathrm{d}x = 0$, тогда верхние суммы Дарбу должны стремиться к 0 с уменьшением 
мелкости разбиения. Выберем такое разбиение для которого 
$S(f, \tau_1) < 1$, должен существовать отрезок $I_1$ на котором супермум меньше 1, тогда выполнено, что $f(x \in I_1) < 1$. 
Далее найдём разбиение $\tau_2$  такое, что $S(f, \tau_2) < \frac{\left\vert I_1 \right\vert}{2}$, должен существовать 
отрезок $I_2 \subset I_1$, такой что $f(x \in I_2) < \frac{1}{2}$, иначе бы верхняя сумма Дарбу была бы больше $\frac{\left\vert I_1 \right\vert }{2}$. 
Производя нахождение отрезков так далее получим, что $f(x \in I_n) < \frac{1}{n}$, и $I_n \subset I_{n-1} \subset \dots \subset I_1$. 
Получили последовательность вложенных отрезков, известно, что они имеют ненулевое пересечение, тогда $\forall x \in \bigcap_{n=1}^{\infty} I_n$ $f(x) = 0$ , то есть
функция не строго положительна.          

\section{3.4}
Выберем промежуток $I \in [a, b]$ и добавим к нему точку $t$, разбивающую его на 2 промежутка $I_1$ и $I_2$. 
Тогда рассмотрим одно из слагаемых суммы Дарбу: 
\[
    \vert I \vert \sup_I f(x) = \vert I_1 \vert \sup_I f(x) + \vert I_2 \vert \sup_I f(x) \geq \vert I_1 \vert \sup_{I_1} f(x) + \vert I_1 \vert \sup_{I_2} f(x)
\]  
То есть при разбиении на более мелкие части сумма Дарбу может только уменьшиться.
Дальше не знаю пока.

\end{document}