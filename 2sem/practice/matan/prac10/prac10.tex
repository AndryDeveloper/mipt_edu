\documentclass[12pt]{article}
\usepackage[T2A]{fontenc}
\usepackage[utf8]{inputenc}
\usepackage{multirow}
\usepackage{caption}
\usepackage{subcaption}
\usepackage{amsmath}
\usepackage{amssymb}
\usepackage{changepage}
\usepackage{graphicx}
\usepackage{float}
\usepackage[english,russian]{babel}
\usepackage{amsmath, amsfonts, amssymb, amsthm, mathtools}
\usepackage{xcolor}
\usepackage{array}
\usepackage{hyperref}
\usepackage{physics}
\usepackage[top = 1.5cm, left = 1.5 cm, right = 1.5 cm, bottom = 3 cm]{geometry}
\usepackage{import}
\usepackage{xifthen}
\usepackage{pdfpages}
\usepackage{transparent}

\newcommand{\incfig}[1]{
    \import{./figures/}{#1.pdf_tex}
}

\title{Практика 10.}
\author{Шахматов Андрей, Б02-304}
\date{\today}

\begin{document}
\maketitle
\tableofcontents

\section{1.1}
\[
    f^{-1} (\{1\}) = A, f^{-1}(\{0\}) = \mathbb{R}^n \setminus A, f^{-1}(\{0, 1\}) = \mathbb{R}^n 
\]
Так как $A$ измеримо тогда и только тогда когда $\mathbb{R}^n \setminus A$ измеримо, то 
необходимым и достаточным условием измеримости является измеримость $A$. 
\section{1.2}
Так как функция монотонна, то она имеет счётное число точек разрыва, тогда можно разбить 
её область определения на счётоное объединение непрерывных множеств: 
\[
    \mathbb{R} = \bigsqcup_{k = 1}^{\infty} X_k
\]   
На каждом из $X_k$ функция непрерывна, а значит и измерима: 
\[
    \{x \in \mathbb{R} \mid f(x) < c\} = \bigcup_{k=1}^{\infty} \{{x \in X_k \mid f(x) < c}\}
\] 
Тогда такое множество является счётным объединением измеримых, а значит измеримо.
\section{1.3}
а) 
\[
    f^3 = g \circ f, 
\] 
где $g(y) = y^3$. Так как $y^3$ - непрерывна, то по теореме о композиции непрерывной и измеримой 
композиция измерима. 
\\ б)
аналогично пункту а). 
\\ в) 
\[
    f(x + a) = f \circ g, 
\]
где $g(x) = x + a$, тогда для любого борелевского $X$, $f^{-1}(X)$ - измеримо. Тогда так как 
мера Лебега инварианта относительно переноса, то $f^{-1}(X) - a$ - тоже измеримо.  Тогда композиция тоже 
измерима.     
\section{1.4}
\[
    \{x \in X \mid f(x) \geq c \} = \mathbb{R}^n \setminus \left\{ x \in X \mid f(x) < c \right\}  
\]
\section{2.1}
Так как $g(x)$ - непрерывна, то $g^{-1}(X)$ переводит открытые в открытые. А значит она переводит 
борелевские в борелевские. А так как как измеримое отображение, то оно переведёт 
борелевское в измеримое, тогда $g^{-1} \circ f^{-1}$ переводит борелевские в измеримые, а значит 
$f \circ g$ измеримо по определению. 

\section{2.3}
Так как множество нигде не плотно, то множество его предельных точек совпадает со всем $\mathbb{R}$. 
Тогда для любого $y_0 \in \mathbb{R} \setminus E$ найдётся последовательность $y_k \to y_0$, выберем из 
$y_k$ такую подпоследовательность где $y_{k_j} > y_0$ (если таковой нет, выберем $y_{k_j} < y_0$). 
Тогда
\[
    \left\{ x \in X \mid f(x) < y_0 \right\} = \bigcap_{j = 1}^{\infty} \left\{ x \in X \mid f(x) < y_{k_j} \right\}
\]
\section{2.4}
Построим множество точек, для которых $\left\{ f_k(x) \right\} $ ограничена: 
\[
    \left\{ x \in X \mid \forall k \in \mathbb{N} \, \vert f_k(x) \vert  < M \right\} = 
    \bigcap_{i=1}^{\infty} \left\{ x \in X \mid \vert f_k(x) < M \vert  \right\}  
\] 
- объединение измеримых измеримо.
\section{2.5}
а) 
Чтобы доказать требуемый факт, нужно доказать, что $q(x) = \{x\}$ - борелевская функция. 
Прообраз $q^{-1}((\alpha, \beta) \subset [0, 1)) = \bigcup_{z \in \mathbb{Z}} (\alpha + z, \beta + z)$. 
Так как прообраз интервала является счётным пересечением интервалов, то функция дробной части действительно 
борелевская. Композиция борелевской и измеримой - измерима.
\\ б) 
В данном пункте следует доказать, что $q(x)$ - "Лебег-измерима", то есть прообраз измеримого измерим. 
Аналогично пункту а) 
\[
    q^{-1} (A \in [0, 1)) = \bigcup_{z \in \mathbb{Z}} \left\{ a + z \mid a \in A \right\},   
\]
получили счётное объединение измеримых - измеримо.
\section{3.1}
а) Нужно доказать, что любой прообраз $p(x) = [x]$ является борелевским. 
Пусть $X \subset Z$.
\[
    p^{-1}(\left\{ X \right\}) = \bigcup_{x \in X} [x, x + 1), 
\]
счётное пересечение борелевских. 
б) Аналогичное нужно доказать для функции Римана. Прообраз любой точки из образа функции Римана, то есть 
точки вида $x = \frac{1}{n}, n \in \mathbb{N}$ есть конечное число точек со знаменателем $n$. 
Тогда $\forall X \in R((0, 1))$, причём $X$ - счётно: 
\[
    R^{-1}(X) = \bigcup_{k=1}^{\infty} \bigcup_{i=1}^{n} \left\{ \frac{i}{n} \right\} - 
\]   
Каждую точку монжно представить как счётное пересечение отрезков, в итоге получим: 
\[
    R^{-1}(X) = \bigcup_{k=1}^{\infty} \bigcup_{i=1}^{n} \bigcap_{p=1}^{\infty} \left[ \frac{i}{n}, \frac{i}{n} + \frac{1}{p} \right]   - 
\] 
очевидно борелевское множество.

\section{3.2}
Если $f$ - измерима, то 
\[
    \{x \in X \mid f(x) = c\} = \{x \in X \mid f(x) \leq c\} \cup \left( \mathbb{R}^n \setminus \{x \in X \mid f(x) < c\} \right) 
\]  
Все такие множества измеримы, потому $\{x \in X \mid f(x) = c\}$ измеримо. 
В обратную сторону неверно, например, рассмотрим множество Витали на отрезке $[0, 1]$. Так как 
множество Витали неизмеримо, то его мощность не может быть счётной, а занчит она континум. 
Тогда существует биекция $f:V \to \mathbb{R}$. Тогда для любого $c \in \mathbb{R}$, 
$\left\{ x \in \mathbb{R} \mid f(x) = c\right\} = \left\{ a \in [0, 1] \right\}$ - очевидно измеримо, 
однако полный прообраз очевидно неизмерим. 
\section{3.3}
Рассмотрим функцию $f(x) = \frac{1}{2} (x + c(x))$ на $(0, 1)$, где $c(x)$ - канторова лестница, такая функция 
непрерывна и монотонна, а значит измерима. Обозначим 
множество Кантора как $K$. Тогда $\mu (f(K)) = \frac{1}{2}$, так как $\mu(f((0, 1) \setminus f(K))) = \frac{1}{2}$, ведь 
$\mu(\frac{1}{2}c((0, 1) \setminus f(K))) = 0$, так как является счётным, а $\mu(\frac{1}{2}id((0, 1) \setminus f(K))) = \frac{1}{2}$. 
Тогда можно найти неизмеримое множество $X \subset f(K)$. 
Прообраз $f^{-1}(X) = A$ - измерим, так как является подмножеством множества Кантора с нулевой мерой. 
Тогда взяв в качестве измеримой функции $g = f^{-1}(x)$ и измеримово множества $A$ получим, что 
$g^{-1}(A) = X$ - неизмеримо.


\end{document}