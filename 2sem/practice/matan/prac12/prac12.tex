\documentclass[12pt]{article}
\usepackage[T2A]{fontenc}
\usepackage[utf8]{inputenc}
\usepackage{multirow}
\usepackage{caption}
\usepackage{subcaption}
\usepackage{amsmath}
\usepackage{amssymb}
\usepackage{changepage}
\usepackage{graphicx}
\usepackage{float}
\usepackage[english,russian]{babel}
\usepackage{amsmath, amsfonts, amssymb, amsthm, mathtools}
\usepackage{xcolor}
\usepackage{array}
\usepackage{hyperref}
\usepackage{physics}
\usepackage[top = 1.5cm, left = 1.5 cm, right = 1.5 cm, bottom = 3 cm]{geometry}
\usepackage{import}
\usepackage{xifthen}
\usepackage{pdfpages}
\usepackage{transparent}

\newcommand{\incfig}[1]{
    \import{./figures/}{#1.pdf_tex}
}

\title{Практика 12.}
\author{Шахматов Андрей, Б02-304}
\date{\today}

\begin{document}
\maketitle
\tableofcontents

\section{1.1}
Да так как $D(x)$ - конечно ступенчатая функция на множестве $\mathbb{Q}_{[0,1]}$ с интегралом 
$\int_{[0, 1]} D(x) \,\mathrm{d}x = \mu(\mathbb{Q}_{[0, 1]}) = 0$. 
\section{1.2}
Выбросим все множетсва меры нуль и получим: 
\[
    \int_{[0, 1]} x \,\mathrm{d}x = \frac{1}{2} 
\] 
\section{2.2}
Пусть функция: 
\[
    f(x) = \begin{dcases}
        \frac{1}{x}, x \in (0, 1);\\
        2 - x, x \in [1, 2];\\
        - \frac{1}{x - 3}, x \in (2, 3).
    \end{dcases}
\]
Очевидно, что положительная и отрицательная части интеграла равны $+\infty$ и $-\infty$.  
\section{2.3}
Разобъём область определения на два измеримых множества: $X_1 = \left\{ x \in X \mid f(x) > g(x) \right\}$ и 
$X_2 = \mathbb{R} \setminus X_1$. Тогда на $X_1$: $\max \left\{ f(x), g(x) \right\} = f(x)$, 
из чего следует интегрируемость $\max \left\{ f(x), g(x) \right\}$ на $X_1$. Аналогично для $X_2$. 
Тогда остаётся рассмотреть не произойдёт ли так, что на $X_1$ максимум интегрируется в $+\infty$, 
а на $X_2$ в $-\infty$. Предположим противное, то есть $\int_{X_1} f(x) \,\mathrm{d}x = +\infty$ 
и $\int_{X_2} g(x) \,\mathrm{d}x = -\infty$. Но тогда из монотонности интеграла и определения
множества $X_1$ следует $\int_{X_1} f(x) \,\mathrm{d}x = -\infty$. Но тогда на одном множестве 
функция $f$ интегрируема в $+\infty$, а на другом в $-\infty$, что означает не интегриуремость 
$f$ на всём $\mathbb{R}$.             


\section{2.6}
Будем делать всё по лемме 5.85 
\\а) $\vert f \vert \leq g \implies -g \leq f \leq g$, так как $g$ - интегрируема с конечным интегралом, то 
она может быть оценена снизу ступенчатой с конечным интегралом: 
\[
    -h \leq f \leq h
\], 
что по лемме $5.85$ означает интегрируемость с конечным интегралом.
\\б) Функция может быть оценена константой $\vert f \vert \leq M$, так как $\int_{X} M \,\mathrm{d}x = M \mu X$ - 
конечный, то по пункту а) имеем интегрируемость $f$. 
\\в) Так как функция непрерывна на компакте, то она ограничена, а так как компакт в $\mathbb{R}$, 
то он ограничен и замкнут, а значит измерим с конечной мерой. 
\section{3.1}
Так как $f$ - интегрируема, то она измерима, тогда $f^2$ - измерима. 
Тогда на измеримом множестве $X_1 = \left\{ x \in X \mid \vert f(x) < 1 \vert  \right\}$ 
выполнено $\vert f \vert \leq f^2 \leq f^4$, тогда по признаку $2.6$ имеем, что $f$ интегрируема на 
$X_1$, аналогично со сменой знакак в неравенствах получим интегрируемость на множестве 
$X_2 = \left\{ x \in X \mid \vert f(x) \geq 1 \vert \right\}$. Тогда так как $f^2$ 
интегрируема с конечным интегралом на $X_1$ и $X_2$, то она интегрируема на 
$\mathbb{R} = X_1 \sqcup X_2$. 
\section{3.2}
Стоит отметить, что наличие конечного интеграла у модуля функции равносильно наличию конечного интеграла у 
самой функции. Тогда достаточно доказать для неотрицательной функции. 
Введём ступенчатую функцию: 
\[
    \phi(x) = \sum_{n=1}^{\infty} n \chi_{\left\{ x \in [a, b] \mid n \leq f(x) \leq n + 1 \right\}} (x)
\]
Тогда верно 
\[
    \phi \leq f \leq \phi + 1
\]      
По определению интеграла ступенчатой функции: 
\[
    \int_{[a, b]} \phi(x) \,\mathrm{d}x = \sum_{n=1}^{\infty} n \mu(x \in [a, b] \mid n \leq f(x) \leq n + 1 )
\]
Тогда проинтегрировав неравенство получим: 
\[
    \int_{[a, b]} \phi(x) \,\mathrm{d}x \leq \int_{[a, b]} f(x) \,\mathrm{d}x \leq \int_{[a, b]} \phi(x) \,\mathrm{d}x + \mu([a, b]) 
\]
Если $\mu A < +\infty$, то из неравенств немедленно следует требуемый факт. 



\end{document}