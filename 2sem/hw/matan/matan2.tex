\documentclass[12pt]{article}
\usepackage[T2A]{fontenc}
\usepackage[utf8]{inputenc}
\usepackage{multirow}
\usepackage{caption}
\usepackage{subcaption}
\usepackage{amsmath}
\usepackage{amssymb}
\usepackage{changepage}
\usepackage{graphicx}
\usepackage{float}
\usepackage[english,russian]{babel}
\usepackage{amsmath, amsfonts, amssymb, amsthm, mathtools}
\usepackage{xcolor}
\usepackage{array}
\usepackage{hyperref}
\usepackage{physics}
\usepackage[top = 1.5cm, left = 1.5 cm, right = 1.5 cm, bottom = 3 cm]{geometry}
\usepackage{import}
\usepackage{xifthen}
\usepackage{pdfpages}
\usepackage{transparent}

\newcommand{\incfig}[1]{
    \import{./figures/}{#1.pdf_tex}
}

\title{Матан вторая домашка.}
\author{Шахматов Андрей, Б02-304}
\date{\today}

\begin{document}
\maketitle
\tableofcontents

\section{T1}
б)
\[
    f_n(x) = \frac{x}{n} \ln \frac{x}{n} \to 0, n \to 0
\]
При $x > 1$ выберем последовательность $x_n = 2n$: 
\[
    f_n(x_n) = 2 \ln 2 = \varepsilon
\]
При $0 < x < 1$ исследуем функцию на монотонность: 
\[
    f_n^{\prime}(x) = \frac{1}{n} \left( \ln \frac{x}{n} + 1 \right) 
\] 
Тогда функция $\vert f_n(x) \vert $ возрастает при $x < \frac{n}{e}$, то есть при $n > 3$ функция 
монотонна на $(0, 1)$. Тогда она принимает максимальное значение в точке $x = 1$: 
\[
    \vert f_n(x) \vert \leq \frac{1}{n} \ln \frac{1}{n} \to 0, n \to \infty
\]
\\г)
\[
    f_n(x) = n \arctg \frac{x}{n} \to x
\]
При $x > 1$ выберем $x_n = 2n$: 
\[
    n \arctg 2 \geq \arctg 2 = \varepsilon 
\]  
При $0 < x < 1$: 
\[
    \vert f_n(x) - x \vert = \vert n(\frac{x}{n} + \frac{1}{2(1 + \varepsilon^2)} (\frac{x}{n})^2) - x \vert \leq \frac{1}{2n} \to 0, n \to \infty.   
\]
\\д) 
\[
    f_n = x^n - x^{n+1} = x^n(1 - x) \to 0 
\]
Рассмотрим $f_{n+1}(x) - f_n(x)$: 
\[
    f_{n+1}(x) - f_n(x) = x^{n+1}(1 - x) - x^n(1 - x) = x^n(1 - x)(x - 1) \leq 0  
\] 
То есть $f_n$ - монотонна по $n$, тогда по признаку Дини сходимость равномерная. 
\\е)
\[
    f_n = x^n - x^{2n} = x^n(1 - x^n) \to 0 
\]
Функция достигает максимума в точке $x^n = \frac{1}{2} \implies f_{max} = \frac{1}{4} \implies \sup f_n(x) = \frac{1}{4} \not \to 0$ 
\section{T2}
Потом сделаю... 
\section{T3}
\section{T4}
Докажем по признаку Абеля, для этого нужно доказать, что $b_n = \frac{1}{n^x}$ монотонна и ограничена. 
Ограниченность очевидна $b_n \leq 1$, покажем монотонность: 
\[
    \frac{\frac{1}{(n+1)^x}}{\frac{1}{n^x}} = \frac{1}{\left( 1 + \frac{1}{n} \right)^x } \leq 1 
\] 
последовательность убывает при любом фиксированном $x$. 

\section{T6}
Запишем $w_f(t_n) = \sup \{\vert f(x) - f(x + \delta ) \vert \mid \delta \leq  t_n\} \geq \vert f(x) - f(x - t_n) \vert $. 
Тогда по теореме Кантора функция равномерно-непрерывна, тогда $w_f(t_n) \to 0, t_n \to 0$.   
\section{T7. Признак Дини}
Рассмотри множество $Q_n = \{x \mid \vert f_n(x) - f(x) \vert \leq \varepsilon\}$, 
каждое из таких множеств является открытым, так как $\vert f_n(x) - f(x) \vert$ - непрерывна, и 
множество задаётся строгим неравенством. Так как $f_n \to f$ следует, что $[a, b] \subset \bigcup_{n=1}^{\infty} Q_n$. 
Из того, что функции монотонны по $n$ следует вложеннность $Q_n$ $Q_1 \subset Q_2 \subset \dots \subset Q_n$.
Тогда так как $[a, b]$ - компакт следует, что из $\bigcup_{n=1}^{\infty} Q_n$ можно выбрать конечное подпокрытие 
$Q_k \cup \dots \cup Q_N = Q_N$. Получили, что найдётся $N$, такое что $\forall n > N$ $\forall x \in [a, b]$ $x \in Q_N \subset Q_n$.              


\end{document}