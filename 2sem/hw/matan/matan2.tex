\documentclass[12pt]{article}
\usepackage[T2A]{fontenc}
\usepackage[utf8]{inputenc}
\usepackage{multirow}
\usepackage{caption}
\usepackage{subcaption}
\usepackage{amsmath}
\usepackage{amssymb}
\usepackage{changepage}
\usepackage{graphicx}
\usepackage{float}
\usepackage[english,russian]{babel}
\usepackage{amsmath, amsfonts, amssymb, amsthm, mathtools}
\usepackage{xcolor}
\usepackage{array}
\usepackage{hyperref}
\usepackage{physics}
\usepackage[top = 1.5cm, left = 1.5 cm, right = 1.5 cm, bottom = 3 cm]{geometry}
\usepackage{import}
\usepackage{xifthen}
\usepackage{pdfpages}
\usepackage{transparent}

\DeclareMathOperator*\lowlim{\underline{lim}}
\DeclareMathOperator*\uplim{\overline{lim}}

\newcommand{\incfig}[1]{
    \import{./figures/}{#1.pdf_tex}
}

\title{Матан вторая домашка.}
\author{Шахматов Андрей, Б02-304}
\date{\today}

\begin{document}
\maketitle
\tableofcontents

\section{T1}
б)
\[
    f_n(x) = \frac{x}{n} \ln \frac{x}{n} \to 0, n \to \infty
\]
При $x > 1$ выберем последовательность $x_n = 2n$:
\[
    f_n(x_n) = 2 \ln 2 = \varepsilon
\]
При $0 < x < 1$ исследуем функцию на монотонность:
\[
    f_n^{\prime}(x) = \frac{1}{n} \left( \ln \frac{x}{n} + 1 \right)
\]
Тогда функция $\vert f_n(x) \vert $ возрастает при $x < \frac{n}{e}$, то есть при $n > 3$ функция
монотонна на $(0, 1)$. Тогда она принимает максимальное значение в точке $x = 1$:
\[
    \vert f_n(x) \vert \leq \frac{1}{n} \ln \frac{1}{n} \to 0, n \to \infty
\]
\\г)
\[
    f_n(x) = n \arctg \frac{x}{n} \to x
\]
При $x > 1$ выберем $x_n = 2n$:
\[
    n \arctg 2 \geq \arctg 2 = \varepsilon
\]
При $0 < x < 1$:
\[
    \left\vert f_n(x) - x \right\vert  = \left\vert n \left[ \frac{x}{n} + \frac{1}{2(1 + \varepsilon^2)} \left( \frac{x}{n} \right)^2 \right]   - x \right\vert  \leq \frac{1}{2n} \to 0, n \to \infty.
\]
\\д)
\[
    f_n = x^n - x^{n+1} = x^n(1 - x) \to 0
\]
Рассмотрим $f_{n+1}(x) - f_n(x)$:
\[
    f_{n+1}(x) - f_n(x) = x^{n+1}(1 - x) - x^n(1 - x) = x^n(1 - x)(x - 1) \leq 0
\]
То есть $f_n$ - монотонна по $n$, тогда по признаку Дини сходимость равномерная.
\\е)
\[
    f_n = x^n - x^{2n} = x^n(1 - x^n) \to 0
\]
Функция достигает максимума в точке $x^n = \frac{1}{2} \implies f_{max} = \frac{1}{4} \implies \sup f_n(x) = \frac{1}{4} \not \to 0$
\section{T2}
б)
\[
    \sum_{n=1}^{\infty} \frac{\sqrt{x} }{n}\sin \frac{x}{n}
\]
При $x \in (0, 1)$:
\[
    \sum_{n=1}^{\infty} \left\vert \frac{\sqrt{x} }{n}\sin \frac{x}{n} \right\vert  \leq \sum_{n=1}^{\infty} \frac{x \sqrt{x}}{n^2} \leq \sum_{n=1}^{\infty} \frac{1}{n^2}
\]
Тогда по признаку Вейерштрасса ряд сходится равномерно.
При $x \in (1, +\infty)$ рассмотрим сумму из отрицания критерия Коши при $n(N) = N, p(N) = N, x = 2N$:
\[
    \sum_{k=N}^{2N} \frac{\sqrt{2N}}{k} \sin \frac{2N}{k} \geq N \sqrt{2N} \sin 1 \frac{1}{2N} = \sqrt{2N} \frac{\sin 1}{2} \geq \frac{\sin 1}{\sqrt{2} }
\]
\\ в)
\[
    \sum_{n=1}^{\infty} \frac{nx}{n^2 + x^2} \arctg \frac{x}{n}
\]
При $x \in (0, 1)$:
\[
    \sum_{n=1}^{\infty} \left\vert \frac{nx}{n^2 + x^2} \arctg \frac{x}{n} \right\vert  \leq \sum_{n=1}^{\infty} \frac{x^2}{n^2 + x^2} \leq \sum_{n=1}^{\infty} \frac{1}{2n^2}
\]
Тогда по признаку Вейерштрасса ряд сходится равномерно.
Рассмотрим последовательность $x = n$, тогда с $n > 1$ выполняется:
\[
    u_n(x_n) = \frac{n^2}{n^2 + n^2} \arctg \frac{n}{n} = \frac{1}{2} \arctg 1 = \varepsilon
\]
То есть невыполняется необходимое условие сходимости ряда, а значит ряд не сходится равномерно, при $x \in (1, +\infty)$.
\\ г)
\[
    \sum_{n=1}^{\infty} \frac{n^2 x^2}{n^4 + x^4} \sin \frac{n}{x}
\]
При $x > 1$ рассмотрим последовательность $x_n = n$, тогда:
\[
    u_n(x_n) = \frac{n^3}{2 n^4} \sin 1 = \frac{1}{2} \sin 1 = \varepsilon
\]
Не выполняется необходиомое условие равномерной сходимости.
При $0 < x < 1$:
\[
    \sum_{n=1}^{\infty} \left\vert \frac{n^2 x^2}{n^4 + x^4} \sin \frac{n}{x} \right\vert \leq \sum_{n=1}^{\infty} \frac{n^2 x^2}{n^4 + x^4} \leq \sum_{n=1}^{\infty} \frac{1}{2n^2}
\]
По признаку Вейерштрасса сходится равномерно.
\\ е)
\[
    \sum_{n=1}^{\infty} \frac{x\ln nx}{n^2}
\]
При $x > 1$ выбрем $x_n = 2n^2$:
\[
    u_n(x_n) = 2 \ln 2n^3 = 2\ln 2 + 6 \ln n > 2\ln 2 = \varepsilon
\]
Не выполняется необходимое условие сходимости. Для определения равномерной сходимости исследуем
функцию $u_n(x) = \left\vert \frac{x\ln nx}{n^2} \right\vert $ на максимум на интервале $(0, 1)$:
\[
    u_n^{\prime}(x) = \frac{1}{n^2}(\ln nx + 1)
\]
Тогда в точке $x = \frac{1}{ne}$ находится экстремум, а значит
максимальное значение функции:
\[
    \sup u_n = \max \left\{ u_n(\frac{1}{ne}), u_n(1) \right\} =
    \max \left\{ \frac{1}{n^3 e}, \frac{\ln n}{n^2} \right\}
\]
Так как оба ряда $\sum_{n=1}^{\infty} \frac{1}{n^3 e}$ и $\sum_{n=1}^{\infty} \frac{\ln n}{n^2}$ сходятся, то
исходный ряд сходится по признаку Вейерштрасса.

\section{T3}
Так как функции $u_n$ - монотонны на $[a, b]$, то:
\[
    \left\vert u_n \right\vert \leq \sup \left\vert u_n \right\vert = \max \left\{ \left\vert u_n(a) \right\vert , \left\vert u_n(b) \right\vert  \right\} \leq
    \left\vert u_n(a) \right\vert + \left\vert u_n(b) \right\vert
\]
Но так как ряды $\sum_{n=1}^{\infty} \left\vert u_n(a) \right\vert$ и $\sum_{n=1}^{\infty} \left\vert u_n(b) \right\vert$ сходятся абсолютно, то
и ряд $\sum_{n=1}^{\infty} \left\vert u_n(a) \right\vert + \left\vert u_n(b) \right\vert$ сходится абсолютно, а значит по признаку
Вейерштрасса ряд $\sum_{n=1}^{\infty} u_n$ равномерно сходится на $[a, b]$.

\section{T4}
Докажем по признаку Абеля, для этого нужно доказать, что $b_n = \frac{1}{n^x}$ монотонна и ограничена.
Ограниченность очевидна $b_n \leq 1$, покажем монотонность:
\[
    \frac{\frac{1}{(n+1)^x}}{\frac{1}{n^x}} = \frac{1}{\left( 1 + \frac{1}{n} \right)^x } \leq 1
\]
последовательность убывает при любом фиксированном $x$.

\section{T6}
Запишем $w_f(t_n) = \sup \{\vert f(x) - f(x + \delta ) \vert \mid \delta \leq  t_n\} \geq \vert f(x) - f(x - t_n) \vert $.
Тогда по теореме Кантора функция равномерно-непрерывна, тогда $w_f(t_n) \to 0, t_n \to 0$.
\section{T7. Признак Дини}
Рассмотри множество $Q_n = \{x \mid \vert f_n(x) - f(x) \vert \leq \varepsilon\}$,
каждое из таких множеств является открытым, так как $\vert f_n(x) - f(x) \vert$ - непрерывна, и
множество задаётся строгим неравенством. Так как $f_n \to f$ следует, что $[a, b] \subset \bigcup_{n=1}^{\infty} Q_n$.
Из того, что функции монотонны по $n$ следует вложеннность $Q_n$ $Q_1 \subset Q_2 \subset \dots \subset Q_n$.
Тогда так как $[a, b]$ - компакт следует, что из $\bigcup_{n=1}^{\infty} Q_n$ можно выбрать конечное подпокрытие
$Q_k \cup \dots \cup Q_N = Q_N$. Получили, что найдётся $N$, такое что $\forall n > N$ $\forall x \in [a, b]$ $x \in Q_N \subset Q_n$.
\section{T8}
б)
\[
    \sum_{n=1}^{\infty} \frac{(n!)^2}{(2n)!} z^n
\]
Воспользуемся формулой Даламбера:
\[
    \frac{1}{R} = \lim_{n \to \infty} \frac{\left( (n+1)! \right)^2 }{(n!)^2} \frac{(2n)!}{(2n + 2)!} =
    \lim_{n \to \infty} \frac{(n + 1)^2}{2(2n + 1)(n+1)} = \frac{1}{2} \lim_{n \to \infty} \frac{n+1}{2n+1} = \frac{1}{4}
\]
А значит радиус сходиомсти $R = 4$.
\\ доп)
\[
    \sum_{n=1}^{\infty} \frac{x^{pn}}{(1 - i)^n}
\]
По формуле Коши-Адамара:
\[
    \frac{1}{R} = \uplim_{n \to \infty} \vert c_n \vert^{\frac{1}{n}} = \lim_{k \to \infty} \left\vert \frac{1}{(1 - i)^k} \right\vert^{\frac{1}{pk}} =
    \frac{1}{\sqrt[p]{2}}
\]
\section{T9}
а)
\[
    \sum_{n=1}^{\infty} (\sqrt[n]{a} - 1) x^n
\]
По формуле Даламбера:
\[
    \frac{1}{R} = \lim_{n \to \infty} \frac{a^{\frac{1}{n+1}} - 1}{a^{\frac{1}{n}} - 1} =
    \lim_{n \to \infty} \frac{\frac{1}{n+1} \ln a}{\frac{1}{n} \ln a} + o(1) = 1
\]
Радиус сходимости равен 1.
При $x = 1$:
\[
    \sum_{n=1}^{\infty} \sqrt[n]{a} - 1 = \sum_{n=1}^{\infty} e^{\frac{1}{n} \ln a} - 1 \geq \sum_{n=1}^{\infty} \frac{1}{n} \ln a - \text{расходится}
\]
При $x = -1$:
\[
    \sum_{n=1}^{\infty} (-1)^n \left( \sqrt[n]{a} - 1 \right)
\]
По признаку Лейбница сходится условно.
\section{T10}
а)
\[
    \frac{1}{x^2 - 2x - 3} = \frac{1}{(x+1)(x-3)} = \frac{1}{4} \left( \frac{1}{x - 3} - \frac{1}{x + 1} \right) =
    \frac{1}{4} \cdot \frac{1}{1 + x} - \frac{1}{12} \cdot \frac{1}{1 - \frac{x}{3}}
\]
\[
    \frac{1}{4} \cdot \frac{1}{1 + x} = \sum_{n=0}^{\infty} \frac{1}{4} (-1)^n x^n
\]
\[
    \frac{1}{12} \cdot \frac{1}{1 - \frac{x}{3}} = \sum_{n=1}^{\infty} \frac{x^n}{12 \cdot 3^n}
\]
Тогда:
\[
    \frac{1}{x^2 - 2x - 3} = \sum_{n=1}^{\infty} x^n \left( \frac{(-1)^n}{4} + \frac{1}{12 \cdot 3^n} \right)
\]
Радиус сходимости равен минимуму из радиусов сходимости составных рядов, т.е $R = 1$.
\\ б)
\[
    \frac{1}{(x^2 + 2)^2} = \frac{1}{\sqrt{2}} \cdot \frac{1}{\left( 1 + \left( \frac{x}{\sqrt{2}} \right)^2 \right)^2 } = \sum_{n=0}^{\infty} \frac{(-1)^n}{\sqrt{2}} (n + 1) \frac{x^{2n}}{2^n}
\]
Радиус сходимости равен $R = 1^2 \cdot \sqrt{2} = \sqrt{2}$
\\ в)
\[
    \ln \frac{2 + x^2}{\sqrt{1 - 2x^2}} =
    \ln 2 + \ln \left( 1 + \frac{x^2}{2} \right)  - \frac{1}{2} \ln \left( 1 - 2x^2 \right)  =
    \ln 2 + \sum_{n=1}^{\infty} \frac{x^{2n}}{n} \left( \frac{(-1)^{n+1}}{2^n} - 2^{n-1} \right)
\]
Радиус сходимости $R = \frac{1}{\sqrt{2}}$
\\ г)
\[
    \sin^3 x = 3\sin x - 4\sin 3x = \sum_{n=0}^{\infty} \frac{3 (-1)^n x^{2n + 1}}{(2n + 1)!} - \sum_{n=0}^{\infty} \frac{4 (-1)^n x^{6n + 3}}{(2n + 1)!}
\]
Ну тут дальше можно кусочно задать явную формулу для коэффициентов ряда. Радиус сходимости $R = \infty$.
\\ д)
\[
    \arctg \frac{2 - x}{1 + 2x} = \arctg 2 - \arctg x = \arctg 2 + \sum_{n=1}^{\infty} (-1)^n \frac{x^{2n-1}}{2n-1}
\]
Радиус сходимости $R = 1$.
\section{T11}
б)
\[
    \sum_{n=0}^{\infty} n^2 x^n
\]
Рассмотрим сумму из пункта а:
\[
    \frac{x}{(1 - x)^2} = \sum_{n=1}^{\infty} n x^n
\]
Почленно продифференцировав получим:
\[
    \sum_{n=1}^{\infty} n^2 x^{n - 1} = \frac{1 + x}{(1 - x)^3} \implies \sum_{n=0}^{\infty} n^2 x^n = \frac{x(1 + x)}{(1 - x)^{3} }
\]
\section{T13}
\[
    \int_{0}^{\pi } \sin x \,\mathrm{d}x
\]
Разобъём отрезок равномерно, тогда выбрав представителя в виде $f(k) = \sin \pi \frac{k}{n}$:
\[
    S = \lim_{n \to \infty} \sum_{k=1}^{n} \frac{\pi }{n} \sin \pi \frac{k}{n} = \lim_{n \to \infty} \frac{\pi}{n} \frac{\sin \left( \frac{n+1}{2} \frac{\pi}{n} \right) \sin \frac{\pi}{2}}{\sin \left( \frac{\pi}{2n} \right) } = 2
\]
\section{T14}
Так как $\frac{1}{x}$ непрерывна, то она интегрируема на любом отрезке из области определения.
Найдём интеграл $\int_{1}^{2} \frac{1}{x} \,\mathrm{d}x$, выберем разбиение, где точки
составляют геометрическую прогрессию: $1, q^1, q^2 \dots q^n$, где $q = \sqrt[n]{2}$,
в качестве представителя выберем самые правые точки, т.е $f(t_k) = \frac{1}{q^k}$.
Тогда сумма Римана будет иметь вид:
\[
    S = \frac{1}{q}(q - 1) + \frac{1}{q^2}(q^2 - q) + \dots \frac{1}{q^n}(q^n - q^{n-1}) =
    \frac{n}{q}(q - 1) = \frac{n(\sqrt[n]{2} - 1)}{\sqrt[n]{2}} = \frac{n( 2^{\frac{1}{n}} - 1) }{2^{\frac{1}{n}}} \to \ln 2
\]
Предел является одним из замечательных пределов для логарифма. Далее замечаем, что искомая сумма:
\[
    S^{\prime}  = \sum_{k=1}^{n} \frac{1}{1 + \frac{k}{n}} \frac{1}{n}
\]
тоже является суммой Римана, но только для равномерного разбиения, тогда так как эти суммы сходятся к одному и тому же
интегралу получим, что сумма $S^{\prime} = \ln 2$.
\section{T17}
а)
\[
    1 + x^n \leq e^{-x^n} \implies \int_{0}^{1} 1 + x^n \,\mathrm{d}x  < \int_{0}^{1} e^{-x^n} \,\mathrm{d}x  \implies 1 - \frac{1}{n} < \int_{0}^{1} e^{-x^n} \,\mathrm{d}x
\]
\section{Интегрируемость функции Римана}
Для любого $\varepsilon > 0$ тогда функция Римана принимает значение большее $\frac{\varepsilon}{2}$ конечное число раз,
покроем все точки $x$ для которых $R(x) > \frac{\varepsilon}{2}$ семейством окрестностей $U_{\frac{\varepsilon}{4}}, U_{\frac{\varepsilon}{8}}, \dots$,
тогда взвешенная сумма колебаний по таким окрестностям не превосходит:
\[
    \Omega(f_U, \tau_U) \leq 1\cdot\frac{\varepsilon}{4} + 1\cdot\frac{\varepsilon}{8} + \dots < \frac{\varepsilon}{2}
\]
В остальных точках значение функции Римана не превосходит $\frac{\varepsilon}{2}$, а значит взвешенная сумма колебаний
не превосходит $1\cdot\frac{\varepsilon}{2}$, тогда взвешенная сумма колебаний по всему разбиению не превосходит $\varepsilon$.
\section{20.13}
Для нахождения радиуса сходимости воспользуемся формулой Даламбера:
\[
    \frac{1}{R} = \lim_{n \to \infty} \frac{(\alpha + n)(\beta + n)}{n (\gamma + n)} = 1
\]
Для исследования ряда на границе сходиомсти подробнее изучим коэффициенты ряда $F_n$:
\[
    \begin{split}
        F_n = \prod_{k = 1}^{n} \frac{(\alpha + k)(\beta + k)}{(\gamma + k)(1 + k)} =
        \prod_{k = 1}^{n} \frac{(1 + \frac{\alpha}{k})(1 + \frac{\beta}{k})}{(1 + \frac{\gamma}{k})(1 + \frac{1}{k})} = \\
        \exp \left\{ \sum_{k=1}^{n} \left[ \ln \left( 1 + \frac{n}{k} \right) + \ln \left( 1 + \frac{\beta }{k} \right) - \ln \left( 1 + \frac{\gamma}{k} \right) - \ln \left( 1 + \frac{1}{k} \right) \right]  \right\}  = \\
        \exp \left\{ \sum_{k=1}^{n} \left[ \frac{\alpha + \beta - \gamma - 1}{k} + O\left( \frac{1}{k^2} \right) \right]  \right\} =
        \exp \left\{ (\alpha + \beta - \gamma - 1)\ln k + A \right\} = e^A k^{\alpha + \beta - \gamma - 1},
    \end{split}
\]
в преобразованиях использована ассимтотическая формула разложения гармонического ряда $\sum_{k=1}^{n} \frac{1}{k} = \ln k + C + o(k)$.
Мы получили ассимптотическую формулу, где $A$ - некоторая положительная константа, зависящая от $\alpha, \beta, \gamma$.
Далее нетрудно провести анализ сходимости ряда, при $x = 1$ ряд ведёт себя как эталонный и сходится при $\alpha + \beta - \gamma - 1 < -1 \implies \alpha + \beta < \gamma$.
При $\alpha + \beta \geq \gamma$ ряд расходится. При $x = -1$ ряд сходится абсолютно при $\alpha + \beta < \gamma$ и сходится условно при $\alpha + \beta - \gamma - 1 < 0$ по признаку Даламбера.
При $\alpha + \beta - \gamma - 1 \geq 0$ ряд расходится.
\section{T.21}
Пусть $X$ - множество значений последовательности $a_n = \frac{1}{n}$. Докажем, что его мера Жордана равна $0$.
Для этого нужно доказать, что для $\forall \varepsilon > 0$, можно найти такое покрытие $X$ из
полукольца, что его мера Жордана меньше $\varepsilon$. Найдём такое $N$, что $\forall n > N \, \frac{1}{n} < \frac{\varepsilon}{2}$,
и покроем все такие точки множеством $[0, \frac{\varepsilon}{2}]$, его мера очевидно $\frac{\varepsilon}{2}$.
Так как последовательность имеет предел, то точек за пределами этого множества конечное число, покроем
их множествами длин $\frac{\varepsilon}{4}, \frac{\varepsilon}{8}, \dots, \frac{\varepsilon}{4n}$. Тогда их мера не превышает $\frac{\varepsilon}{2}$.
Тогда мера покрывающег оэлементарного множества меньше $\varepsilon$.
\section{T.22}
Пусть существует счётное множество с мерой Жордана большей нуля. Тогда так как оно измеримо по Жордану, то
мера его границы равна 0, а значит в силу аддитивности меры Жордана имеем, что мера внутренности множества больше 0.
Так как внутренность это открытое множество, то его можно разбить на дизъюнктное объединение интервалов.
Так как мера их объединения больше нуля, в объединении есть непустой интервал. Так как этот интервал
принадлежит исходному множеству и содержит несчётен, то исходное множество тоже несчётно. Получили протеворечие.
\section{5.8}
$A$ - система всех полуинтервалов. Если $X_1 = [\alpha_1, \beta_1) \in A$ и  $X_2 = [\alpha_2 , \beta_2) \in A$,
то $X_1 \cap X_2 = [\max \{\alpha_1, \alpha_2\}, \min \{\beta_1, \beta_2\}]$, в случае если интервалы пересекаются и
пустому интервалу если не пересекаются. Проверим третье условие. $X_1 = [\alpha_1, \beta_1) \subset [\alpha, \beta) \in A$,
тогда $X = [\alpha, \beta) = [\alpha, \alpha_1) \cup X_1 \cup [\beta_1, \beta)$.
\section{5.9}
Аналогично $5.8$. Так как из отрезка можно сделать полуинтервал дизъюнктным объединением $[a, c) = [a, b] \cup (b, c)$.
\section{5.10}
Множества будут пересекаться по координатам декартового произведения и соответственно давать
в каждой из координат промежуток. Докажем, что если $[(\alpha^1, \beta^1)] \subset [(\alpha, \beta)]$, то
такое множество раскладывается в дизъюнктное объединение содержащее $[(\alpha^1, \beta^1)]$.
Сделаем индукцией по количеству измерений, база очевидна из задач $5.8, 5.9$, тогда шаг индукции:
\[
    [(\alpha_1, \beta_1)] \times \dots \times [(\alpha_n, \beta_n)] =
    [(\alpha_1, \alpha^1_1)] \times \dots \times [(\alpha_n, \beta_n)] \cup
    [(\alpha^1_1, \beta^1_1)] \times \dots \times [(\alpha_n, \beta_n)] \cup
    [(\beta^1, \beta)] \times \dots \times [(\alpha_n, \beta_n)]
\]
Так как согласно гипотезе индукции $[(\alpha_1, \alpha^1_1)] \times \dots \times [(\alpha_n, \beta_n)]$ раскладывается
на дизъюнктное объединение содержащее $[(\alpha^1, \beta^1)]$, доказали шаг индукции.
\section{5.17}
Так как $A \triangle B = (A \cup B) \setminus (A \cap B)$, так как $(A \cup B) \in S$ и $(A \cap B) \in S$,
а их разность $(A \cup B) \setminus (A \cap B) = A_1 \cup A_2 \cup \dots \cup A_n \in S$  по третьей аксиоме полукольца и правилу объединения, тогда $A \triangle B \in S$. ЧТД.
\section{5.18}
По индукции по количеству множеств, для одного утверждение следует из 3 аксиомы полукольца. Шаг индукции, имеем:
\[
    A = \bigsqcup_{i=1}^{n-1} A_i \sqcup \bigsqcup_{k=n}^{m} B_k
\]
Тогда рассмотрим множество $D_j = A_n \cap B_k$, для каждого $D_j$ множно найти дополнение:
\[
    A = \bigsqcup_{i=1}^{n-1} A_i \sqcup \bigsqcup_{j=1}^{s} D_j \sqcup \bigsqcup_{j=1}^{s} \bigsqcup_{l=1}^{p} C_{j, l}
\]
Тогда так как $A_n \cap A_j = \emptyset$ и $A_n \cap \bigsqcup_{j=1}^{s} \bigsqcup_{l=1}^{p} C_{j, l} = \emptyset$, то $A_n = \bigsqcup_{j=1}^{s} D_j$.
\section{5.19}
Докажем по индукции. База очевидна. Пусть по предположению индукции верно для набора из $n - 1$ множеств и 
$B_1, \dots , B_q$ - система множеств, которая подходит для данного набора. Обозначим $C_s = B_s \cap A_n$ при 
$s = 1, 2, \dots q.$ По предыдущей задаче верно: 
\[
    A_n = \left( \bigsqcup_{s=1}^q C_s \right) \sqcup \left( \bigsqcup_{p=1}^m D_p \right) 
\]    
По определению полукольца получаем, что 
\[
    B_s = C_s \sqcup \left( \bigsqcup_{r=1}^{r_s} B_{s, r}\right) 
\]
при $s = 1, \dots , q$, где $B_{s, r} \in S$ для всех $r$. Тогда исследуемые множества попарно не пересекаются и образуют систему, порождающую $A_i$.    
\section{5.21}
\section{5.22}
Из определения исследуемого множества $R_1 \subset R$. Пусть $A \in R$, т.е 
существует $A = \bigcup_{i=1}^{n} A_i$, где $A_i \in S$. Согласно задаче 5.19 
существует набор множеств $D_1, D_2, \dots, D_r \in S$, что дл якаждого $i$ выполнено равенство
$A_i = \bigsqcup_{l \in T_i} D_l$, где $T_i \subset \{1, 2, \dots, r\}$. Тогда $A = \bigsqcup_{l=1}^r D_l \in R_1$.         
\section{6.8}
\section{6.9}
Из задачи $5.18$ имеем:
\[
    A = \bigsqcup_{i=1}^{n} A_i \cup \bigsqcup_{j=1}^{m} B_j
\]
Тогда по аддитивности меры имеем:
\[
    m(A) = \sum_{i=1}^{n} m(A_i) - \sum_{j=1}^{m} m(B_j) \geq \sum_{i=1}^{n} m(A_i)
\]
\section{6.10}
Перейдя к пределу по $n$ в задаче $6.9$ получим искомое неравенство.
\section{6.11}
1) Пусть $A = \bigsqcup_{i=1}^{\infty} A_n$, тогда по задаче $6.10$ получаем,
\[
    m(A) \geq \sum_{i=1}^{\infty} m(A_i)
\]
С учётом $m(A) \leq \sum_{i=1}^{\infty} m(A_i)$ получаем:
\[
    m(A) = \sum_{i=1}^{\infty} m(A_i)
\]
2) Пусть для некоторого множества не выполнено неравенство, тогда из $5.19$:
\[
    A = \bigcup_{i=1}^{\infty} (A \cap A_i) = \bigsqcup_{i=1}^{\infty} \bigsqcup_{k=1}^{k_i} (B_{i, k} \cap A)
\]
\[
    m(A) \leq \sum_{i=1}^{\infty} \sum_{k=1}^{k_i} m(B_{i, k}) \leq \sum_{i=1}^{\infty} m(A_i)
\]
\section{6.12}
Ясно, что $m \geq 0$ и также очевидна аддитивность $m$. Далее докажем $\sigma$ аддитивность. 
Неравенство в одну сторнону показано ранне: 
\[
    m([(a, b)]) \geq \sum_{i=1}^{\infty} m([(a_i, b_i)])
\]
Покажем неравенство в обратную сторону. Для произвольного $\varepsilon > 0$ 
выберем такой набор интревалов $[(a_i, b_i)] \subset (\alpha_i, \beta_i)$, что 
$m((\alpha_i, \beta_i)) < m([(a_i, b_i)]) + \frac{\varepsilon}{2^{i+1}}$ и такой отрезок $[\alpha, \beta] \subset [(a, b)]$, что 
$m([\alpha, \beta]) > m([(a_i, b_i)]) - \frac{\varepsilon}{2}$. система интервалов будет покрывать 
отрезок, тогда из компактности отрезка можно выбрать конечное покртыие: 
\[
    m([(a, b)]) < m([\alpha, \beta])  +\frac{\varepsilon}{2} \leq 
    \frac{\varepsilon}{2} + \sum_{i=1}^{\infty} \left( m([(a_i, b_i)]) + \frac{\varepsilon}{2^{i+1}} \right) \leq \varepsilon + 
    \sum_{i=1}^{\infty} m([(a_i, b_i)])
\]     
Так как полученные неравенства выполнены для любого $\varepsilon$, то оно должно быть выполнено безусловно: 
\[
    m([(a, b)]) \leq \sum_{i=1}^{\infty} m([(a_i, b_i)])
\] 
\section{7.19}
$\forall \varepsilon > 0 \, \mu^{\ast} (A \triangle A_{\varepsilon}) < \varepsilon$, тогда так как $X$ - единица кольца, то
$(x \setminus A_{\varepsilon}) \in R(S)$. Тогда по свойствам симметрической разности:
\[
    \mu^{\ast}  ((X \setminus A) \triangle (X \setminus A_{\varepsilon} )) = \mu^{\ast}  (A \triangle A_{\varepsilon} )< \varepsilon.
\]
\section{7.20}
Так как мера $\mu^{\ast} < \mu_{J}^{\ast}$, то если для $\mu^{\ast}_J (A \triangle A_{\varepsilon}) < \varepsilon$, то 
$\mu^{\ast} < \mu^{\ast}_J (A \triangle A_{\varepsilon}) < \varepsilon$, что значит, что множестов измеримо и по Лебегу. То есть 
все измеримые по Жордану измеримы и по Лебегу.
\section{7.21}
Измеримость следует так как множество из кольца можно приблизить им самим же: 
\[
    \mu^{\ast}_J (A \triangle A) = 0 < \varepsilon
\]
Выбрав в определении верхней меры покрытие, состоящее из элемента $A$ получим равенство мер Жордана и Лебега. 
\section{7.22}
Так как $\vert \mu^{\ast}(A) - \mu^{\ast}(B) \vert < \mu^{\ast} (A \triangle B)$. Считая $A$ - измеримым по Жорадну, 
а $B$ - приближающим множеством кольца имеем $\mu^{\ast}_J (A \triangle B) < \varepsilon$. Тогда:
\[
    \mu_J(A) \leq \mu^{\ast}_J(A) \leq \mu^{\ast}_J (A \triangle B) + \mu^{\ast}_J(B) < \mu^{\ast}_J(B) + \varepsilon
\]
Так как множество $B \in R(S)$ имеем $\mu^{\ast}_J(B) = \mu^{\ast}(B)$:
\[
    \mu_J(A) < \mu^{\ast}_J(B) + \varepsilon = \mu^{\ast}(B) + \varepsilon < \mu^{\ast}(A) + \mu^{\ast} (A \triangle B) + \varepsilon < 
    \mu^{\ast}(A) + 2\varepsilon = \mu(A) + 2\varepsilon
\] 
Переходя к пределу имеем $\mu_J(A) \leq \mu(A)$, неравенство в обратную сторону очевидно. 
\section{7.70}
Найдём псоледовательность $(a_n)$, такую, что: 
\[
    \sum_{i=1}^{\infty} a_i = 1 - \alpha
\] 
Разделим отрезок на 3 части, длина одного из которых $a_1$, уберём этот отрезок, далее каждый из 2 полученных отрезков тоже разделим на 
3 части и выкинем части с длиной $\frac{a_2}{2}$. Продолжая операцию дальше получим последовательность множеств м мерами: 
\[
    \mu(A_n) = 1 - \sum_{i=1}^{n} a_i 
\]  
Из непрерывности меры следует, что $\mu (\bigcap_{i=n}^{\infty} A_n) = 1 - ( 1 - \alpha) = \alpha$. Такое множество нигде не плотно, так 
как для любого интервала $(a, b) \subset [0, 1]$ найдётся такое $a_n$, что $a_n < \mu (a, b)$, что означает, что 
есть полость меры $a_n$ не пренадлежащая множеству.      
\section{7.71}
Рассмотрим множество, полученное в задаче $7.70$. Пусть оно измеримо по Жордану, тогда мера совпадает с его мерой Лебега. 
Так как множество нигде не плотно, то его внутренность пуста, тогда мера Лебега (а значит и Жордана) его границы больше $\alpha > 0$, что означает невозможность 
его измерить по Жордану, получили противоречие. 
\section{T.23}
Пример неизмеримого компактного множества - множество из задачи $7.71$. Докажем его компактность, для этого 
нужно доказать его замкнутость. На каждом шаге итерации мы получали замкнутые множества, а пересечение любого количества 
замкнутых является замкнутым.
\section{Эквивалентность сумм Дарбу и ступенчатых функций}
В одну сторону. Найдём две ступенчатые функции $g, h: g \leq f \leq h$, такие, что 
\[
    \int_{a}^{b} (h(x) - g(x)) \,\mathrm{d}x < \varepsilon 
\]  
Рассмотрим взвешенную сумму колебаний(разность сумм Дарбу) на разбиении, на котором обе функции ступенчатые: 
\[
    \Omega(f, \tau) = \sum_{\Delta \in \tau} w(f, \Delta)\vert \Delta \vert \leq 
    \sum_{\Delta \in \tau} (h(\Delta) - g(\Delta))\vert \Delta \vert = \int_{a}^{b} (h(x) - g(x)) \,\mathrm{d}x < \varepsilon 
\]
В другую сторону. $\Omega(f, \tau) < \varepsilon$ для некоторого разбиения, положим 
$g(\Delta) = \inf f(\Delta), \, h(\Delta) = \sup f(\Delta)$, тогда 
$g \leq f \leq h$, причём:
\[
    \int_{a}^{b} (h(x) - g(x)) \,\mathrm{d}x = 
    \sum_{\Delta \in \tau} (h(\Delta) - g(\Delta))\vert \Delta \vert = 
    \sum_{\Delta \in \tau} (\sup f(\Delta) - \inf f(\Delta))\vert \Delta \vert = \Omega(f, \Delta) < \varepsilon
\] 
\section{4. Зорич}
Ну не знаю для любых сумм Римана выполняеются неравенства, а значит после перехода к пределу 
они сохранятся...




\end{document}