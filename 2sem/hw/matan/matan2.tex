\documentclass[12pt]{article}
\usepackage[T2A]{fontenc}
\usepackage[utf8]{inputenc}
\usepackage{multirow}
\usepackage{caption}
\usepackage{subcaption}
\usepackage{amsmath}
\usepackage{amssymb}
\usepackage{changepage}
\usepackage{graphicx}
\usepackage{float}
\usepackage[english,russian]{babel}
\usepackage{amsmath, amsfonts, amssymb, amsthm, mathtools}
\usepackage{xcolor}
\usepackage{array}
\usepackage{hyperref}
\usepackage{physics}
\usepackage[top = 1.5cm, left = 1.5 cm, right = 1.5 cm, bottom = 3 cm]{geometry}
\usepackage{import}
\usepackage{xifthen}
\usepackage{pdfpages}
\usepackage{transparent}

\DeclareMathOperator*\lowlim{\underline{lim}}
\DeclareMathOperator*\uplim{\overline{lim}}

\newcommand{\incfig}[1]{
    \import{./figures/}{#1.pdf_tex}
}

\title{Матан вторая домашка.}
\author{Шахматов Андрей, Б02-304}
\date{\today}

\begin{document}
\maketitle
\tableofcontents

\section{T1}
б)
\[
    f_n(x) = \frac{x}{n} \ln \frac{x}{n} \to 0, n \to 0
\]
При $x > 1$ выберем последовательность $x_n = 2n$: 
\[
    f_n(x_n) = 2 \ln 2 = \varepsilon
\]
При $0 < x < 1$ исследуем функцию на монотонность: 
\[
    f_n^{\prime}(x) = \frac{1}{n} \left( \ln \frac{x}{n} + 1 \right) 
\] 
Тогда функция $\vert f_n(x) \vert $ возрастает при $x < \frac{n}{e}$, то есть при $n > 3$ функция 
монотонна на $(0, 1)$. Тогда она принимает максимальное значение в точке $x = 1$: 
\[
    \vert f_n(x) \vert \leq \frac{1}{n} \ln \frac{1}{n} \to 0, n \to \infty
\]
\\г)
\[
    f_n(x) = n \arctg \frac{x}{n} \to x
\]
При $x > 1$ выберем $x_n = 2n$: 
\[
    n \arctg 2 \geq \arctg 2 = \varepsilon 
\]  
При $0 < x < 1$: 
\[
    \left\vert f_n(x) - x \right\vert  = \left\vert n \left[ \frac{x}{n} + \frac{1}{2(1 + \varepsilon^2)} \left( \frac{x}{n} \right)^2 \right]   - x \right\vert  \leq \frac{1}{2n} \to 0, n \to \infty.   
\]
\\д) 
\[
    f_n = x^n - x^{n+1} = x^n(1 - x) \to 0 
\]
Рассмотрим $f_{n+1}(x) - f_n(x)$: 
\[
    f_{n+1}(x) - f_n(x) = x^{n+1}(1 - x) - x^n(1 - x) = x^n(1 - x)(x - 1) \leq 0  
\] 
То есть $f_n$ - монотонна по $n$, тогда по признаку Дини сходимость равномерная. 
\\е)
\[
    f_n = x^n - x^{2n} = x^n(1 - x^n) \to 0 
\]
Функция достигает максимума в точке $x^n = \frac{1}{2} \implies f_{max} = \frac{1}{4} \implies \sup f_n(x) = \frac{1}{4} \not \to 0$ 
\section{T2}
б) 
\[
    \sum_{n=1}^{\infty} \frac{\sqrt{x} }{n}\sin \frac{x}{n} 
\]
При $x \in (0, 1)$: 
\[
    \sum_{n=1}^{\infty} \left\vert \frac{\sqrt{x} }{n}\sin \frac{x}{n} \right\vert  \leq \sum_{n=1}^{\infty} \frac{x \sqrt{x}}{n^2} \leq \sum_{n=1}^{\infty} \frac{1}{n^2}  
\] 
Тогда по признаку Вейерштрасса ряд сходится равномерно. 
При $x \in (1, +\infty)$ рассмотрим сумму из отрицания критерия Коши при $n(N) = N, p(N) = N, x = 2N$:
\[
    \sum_{k=N}^{2N} \frac{\sqrt{2N}}{k} \sin \frac{2N}{k} \geq N \sqrt{2N} \sin 1 \frac{1}{2N} = \sqrt{2N} \frac{\sin 1}{2} \geq \frac{\sin 1}{\sqrt{2} }    
\]
\\ в)
\[
    \sum_{n=1}^{\infty} \frac{nx}{n^2 + x^2} \arctg \frac{x}{n}
\]
При $x \in (0, 1)$: 
\[
    \sum_{n=1}^{\infty} \left\vert \frac{nx}{n^2 + x^2} \arctg \frac{x}{n} \right\vert  \leq \sum_{n=1}^{\infty} \frac{x^2}{n^2 + x^2} \leq \sum_{n=1}^{\infty} \frac{1}{2n^2} 
\] 
Тогда по признаку Вейерштрасса ряд сходится равномерно. 
Рассмотрим последовательность $x = n$, тогда с $n > 1$ выполняется:
\[
    u_n(x_n) = \frac{n^2}{n^2 + n^2} \arctg \frac{n}{n} = \frac{1}{2} \arctg 1 = \varepsilon
\]  
То есть невыполняется необходимое условие сходимости ряда, а значит ряд не сходится равномерно, при $x \in (1, +\infty)$. 
\\ г) 
\[
    \sum_{n=1}^{\infty} \frac{n^2 x^2}{n^4 + x^4} \sin \frac{n}{x}
\]
При $x > 1$ рассмотрим последовательность $x_n = n$, тогда:
\[
    u_n(x_n) = \frac{n^3}{2 n^4} \sin 1 = \frac{1}{2} \sin 1 = \varepsilon 
\] 
Не выполняется необходиомое условие равномерной сходимости. 
При $0 < x < 1$: 
\[
    \sum_{n=1}^{\infty} \left\vert \frac{n^2 x^2}{n^4 + x^4} \sin \frac{n}{x} \right\vert \leq \sum_{n=1}^{\infty} \frac{n^2 x^2}{n^4 + x^4} \leq \sum_{n=1}^{\infty} \frac{1}{2n^2}
\]
По признаку Вейерштрасса сходится равномерно. 
\\ е) 
\[
    \sum_{n=1}^{\infty} \frac{x\ln nx}{n^2}
\]
При $x > 1$ выбрем $x_n = 2n^2$: 
\[
    u_n(x_n) = 2 \ln 2n^3 = 2\ln 2 + 6 \ln n > 2\ln 2 = \varepsilon 
\]  
Не выполняется необходимое условие сходимости. Для определения равномерной сходимости исследуем 
функцию $u_n(x) = \left\vert \frac{x\ln nx}{n^2} \right\vert $ на максимум на интервале $(0, 1)$: 
\[
    u_n^{\prime}(x) = \frac{1}{n^2}(\ln nx + 1)
\] 
Тогда в точке $x = \frac{1}{ne}$ находится экстремум, а значит 
максимальное значение функции: 
\[
    \sup u_n = \max \left\{ u_n(\frac{1}{ne}), u_n(1) \right\} = 
    \max \left\{ \frac{1}{n^3 e}, \frac{\ln n}{n^2} \right\} 
\]  
Так как оба ряда $\sum_{n=1}^{\infty} \frac{1}{n^3 e}$ и $\sum_{n=1}^{\infty} \frac{\ln n}{n^2}$ сходятся, то 
исходный ряд сходится по признаку Вейерштрасса.

\section{T3}
Так как функции $u_n$ - монотонны на $[a, b]$, то: 
\[
    \left\vert u_n \right\vert \leq \sup \left\vert u_n \right\vert = \max \left\{ \left\vert u_n(a) \right\vert , \left\vert u_n(b) \right\vert  \right\} \leq
    \left\vert u_n(a) \right\vert + \left\vert u_n(b) \right\vert
\]
Но так как ряды $\sum_{n=1}^{\infty} \left\vert u_n(a) \right\vert$ и $\sum_{n=1}^{\infty} \left\vert u_n(b) \right\vert$ сходятся абсолютно, то 
и ряд $\sum_{n=1}^{\infty} \left\vert u_n(a) \right\vert + \left\vert u_n(b) \right\vert$ сходится абсолютно, а значит по признаку 
Вейерштрасса ряд $\sum_{n=1}^{\infty} u_n$ равномерно сходится на $[a, b]$.   

\section{T4}
Докажем по признаку Абеля, для этого нужно доказать, что $b_n = \frac{1}{n^x}$ монотонна и ограничена. 
Ограниченность очевидна $b_n \leq 1$, покажем монотонность: 
\[
    \frac{\frac{1}{(n+1)^x}}{\frac{1}{n^x}} = \frac{1}{\left( 1 + \frac{1}{n} \right)^x } \leq 1 
\] 
последовательность убывает при любом фиксированном $x$. 

\section{T6}
Запишем $w_f(t_n) = \sup \{\vert f(x) - f(x + \delta ) \vert \mid \delta \leq  t_n\} \geq \vert f(x) - f(x - t_n) \vert $. 
Тогда по теореме Кантора функция равномерно-непрерывна, тогда $w_f(t_n) \to 0, t_n \to 0$.   
\section{T7. Признак Дини}
Рассмотри множество $Q_n = \{x \mid \vert f_n(x) - f(x) \vert \leq \varepsilon\}$, 
каждое из таких множеств является открытым, так как $\vert f_n(x) - f(x) \vert$ - непрерывна, и 
множество задаётся строгим неравенством. Так как $f_n \to f$ следует, что $[a, b] \subset \bigcup_{n=1}^{\infty} Q_n$. 
Из того, что функции монотонны по $n$ следует вложеннность $Q_n$ $Q_1 \subset Q_2 \subset \dots \subset Q_n$.
Тогда так как $[a, b]$ - компакт следует, что из $\bigcup_{n=1}^{\infty} Q_n$ можно выбрать конечное подпокрытие 
$Q_k \cup \dots \cup Q_N = Q_N$. Получили, что найдётся $N$, такое что $\forall n > N$ $\forall x \in [a, b]$ $x \in Q_N \subset Q_n$.              
\section{T8}
б)
\[
    \sum_{n=1}^{\infty} \frac{(n!)^2}{(2n)!} z^n 
\]
Воспользуемся формулой Даламбера:
\[
    \frac{1}{R} = \lim_{n \to \infty} \frac{\left( (n+1)! \right)^2 }{(n!)^2} \frac{(2n)!}{(2n + 2)!} = 
    \lim_{n \to \infty} \frac{(n + 1)^2}{2(2n + 1)(n+1)} = \frac{1}{2} \lim_{n \to \infty} \frac{n+1}{2n+1} = \frac{1}{4} 
\]
А значит радиус сходиомсти $R = 4$.
\\ доп)
\[
    \sum_{n=1}^{\infty} \frac{x^{pn}}{(1 - i)^n}
\]
По формуле Коши-Адамара: 
\[
    \frac{1}{R} = \uplim_{n \to \infty} \vert c_n \vert^{\frac{1}{n}} = \lim_{k \to \infty} \left\vert \frac{1}{(1 - i)^k} \right\vert^{\frac{1}{pk}} = 
    \frac{1}{\sqrt[p]{2}}
\]
\section{T9}
а) 
\[
    \sum_{n=1}^{\infty} (\sqrt[n]{a} - 1) x^n
\]
По формуле Даламбера: 
\[
    \frac{1}{R} = \lim_{n \to \infty} \frac{a^{\frac{1}{n+1}} - 1}{a^{\frac{1}{n}} - 1} = 
    \lim_{n \to \infty} \frac{\frac{1}{n+1} \ln a}{\frac{1}{n} \ln a} + o(1) = 1
\]
Радиус сходимости равен 1. 
При $x = 1$:
\[
    \sum_{n=1}^{\infty} \sqrt[n]{a} - 1 = \sum_{n=1}^{\infty} e^{\frac{1}{n} \ln a} - 1 \geq \sum_{n=1}^{\infty} \frac{1}{n} \ln a - \text{расходится}
\] 
При $x = -1$:
\[
    \sum_{n=1}^{\infty} (-1)^n \left( \sqrt[n]{a} - 1 \right) 
\] 
По признаку Лейбница сходится условно.
\section{T10}
а)
\[
    \frac{1}{x^2 - 2x - 3} = \frac{1}{(x+1)(x-3)} = \frac{1}{4} \left( \frac{1}{x - 3} - \frac{1}{x + 1} \right) = 
    \frac{1}{4} \cdot \frac{1}{1 + x} - \frac{1}{12} \cdot \frac{1}{1 - \frac{x}{3}}
\]
\[
    \frac{1}{4} \cdot \frac{1}{1 + x} = \sum_{n=0}^{\infty} \frac{1}{4} (-1)^n x^n
\]
\[
    \frac{1}{12} \cdot \frac{1}{1 - \frac{x}{3}} = \sum_{n=1}^{\infty} \frac{x^n}{12 \cdot 3^n}
\]
Тогда: 
\[
    \frac{1}{x^2 - 2x - 3} = \sum_{n=1}^{\infty} x^n \left( \frac{(-1)^n}{4} + \frac{1}{12 \cdot 3^n} \right) 
\]
Радиус сходимости равен минимуму из радиусов сходимости составных рядов, т.е $R = 1$. 
\\ б) 
\[
    \frac{1}{(x^2 + 2)^2} = \frac{1}{\sqrt{2}} \cdot \frac{1}{\left( 1 + \left( \frac{x}{\sqrt{2}} \right)^2 \right)^2 } = \sum_{n=0}^{\infty} \frac{(-1)^n}{\sqrt{2}} (n + 1) \frac{x^{2n}}{2^n}
\]
Радиус сходимости равен $R = 1^2 \cdot \sqrt{2} = \sqrt{2}$ 
\\ в) 
\[
    \ln \frac{2 + x^2}{\sqrt{1 - 2x^2}} = 
    \ln 2 + \ln \left( 1 + \frac{x^2}{2} \right)  - \frac{1}{2} \ln \left( 1 - 2x^2 \right)  = 
    \ln 2 + \sum_{n=1}^{\infty} \frac{x^{2n}}{n} \left( \frac{(-1)^{n+1}}{2^n} - 2^{n-1} \right) 
\]
Радиус сходимости $R = \frac{1}{\sqrt{2}}$
\\ г)
\[
    \sin^3 x = 3\sin x - 4\sin 3x = \sum_{n=0}^{\infty} \frac{3 (-1)^n x^{2n + 1}}{(2n + 1)!} - \sum_{n=0}^{\infty} \frac{4 (-1)^n x^{6n + 3}}{(2n + 1)!}
\]
Ну тут дальше можно кусочно задать явную формулу для коэффициентов ряда. Радиус сходимости $R = \infty$. 
\\ д) 
\[
    \arctg \frac{2 - x}{1 + 2x} = \arctg 2 - \arctg x = \arctg 2 + \sum_{n=1}^{\infty} (-1)^n \frac{x^{2n-1}}{2n-1} 
\]
Радиус сходимости $R = 1$. 
\section{T11}
б) 
\[
    \sum_{n=0}^{\infty} n^2 x^n 
\]
Рассмотрим сумму из пункта а:
\[
    \frac{x}{(1 - x)^2} = \sum_{n=1}^{\infty} n x^n
\] 
Почленно продифференцировав получим: 
\[
    \sum_{n=1}^{\infty} n^2 x^{n - 1} = \frac{1 + x}{(1 - x)^3} \implies \sum_{n=0}^{\infty} n^2 x^n = \frac{x(1 + x)}{(1 - x)^{3} } 
\]
\section{T13}
\[
    \int_{0}^{\pi } \sin x \,\mathrm{d}x 
\]
Разобъём отрезок равномерно, тогда выбрав представителя в виде $f(k) = \sin \pi \frac{k}{n}$:
\[
    S = \lim_{n \to \infty} \sum_{k=1}^{n} \frac{\pi }{n} \sin \pi \frac{k}{n} = \lim_{n \to \infty} \frac{\pi}{n} \frac{\sin \left( \frac{n+1}{2} \frac{\pi}{n} \right) \sin \frac{\pi}{2}}{\sin \left( \frac{\pi}{2n} \right) } = 2
\] 
\section{T14}
...
\section{T17}
а) 
\[
    1 + x^n \leq e^{-x^n} \implies \int_{0}^{1} 1 + x^n \,\mathrm{d}x  < \int_{0}^{1} e^{-x^n} \,\mathrm{d}x  \implies 1 - \frac{1}{n} < \int_{0}^{1} e^{-x^n} \,\mathrm{d}x
\]
\section{Интегрируемость функции Римана}
Для любого $\varepsilon > 0$ тогда функция Римана принимает значение большее $\frac{\varepsilon}{2}$ конечное число раз, 
покроем все точки $x$ для которых $R(x) > \frac{\varepsilon}{2}$ семейством окрестностей $U_{\frac{\varepsilon}{4}}, U_{\frac{\varepsilon}{8}}, \dots$, 
тогда взвешенная сумма колебаний по таким окрестностям не превосходит: 
\[
    \Omega(f_U, \tau_U) \leq 1\cdot\frac{\varepsilon}{4} + 1\cdot\frac{\varepsilon}{8} + \dots < \frac{\varepsilon}{2}
\]   
В остальных точках значение функции Римана не превосходит $\frac{\varepsilon}{2}$, а значит взвешенная сумма колебаний 
не превосходит $1\cdot\frac{\varepsilon}{2}$, тогда взвешенная сумма колебаний по всему разбиению не превосходит $\varepsilon$. 
\section{20.13}
Для нахождения радиуса сходимости воспользуемся формулой Даламбера: 
\[
    \frac{1}{R} = \lim_{n \to \infty} \frac{(\alpha + n)(\beta + n)}{n (\gamma + n)} = 1
\]
Для исследования ряда на границе сходиомсти подробнее изучим коэффициенты ряда $F_n$:
\[
    \begin{split}
        F_n = \prod_{k = 1}^{n} \frac{(\alpha + k)(\beta + k)}{(\gamma + k)(1 + k)} = 
    \prod_{k = 1}^{n} \frac{(1 + \frac{\alpha}{k})(1 + \frac{\beta}{k})}{(1 + \frac{\gamma}{k})(1 + \frac{1}{k})} = \\
    \exp \left\{ \sum_{k=1}^{n} \left[ \ln \left( 1 + \frac{n}{k} \right) + \ln \left( 1 + \frac{\beta }{k} \right) - \ln \left( 1 + \frac{\gamma}{k} \right) - \ln \left( 1 + \frac{1}{k} \right) \right]  \right\}  = \\
    \exp \left\{ \sum_{k=1}^{n} \left[ \frac{\alpha + \beta - \gamma - 1}{k} + O\left( \frac{1}{k^2} \right) \right]  \right\} = 
    \exp \left\{ (\alpha + \beta - \gamma - 1)\ln k + A \right\} = e^A k^{\alpha + \beta - \gamma - 1},
    \end{split}
\]
в преобразованиях использована ассимтотическая формула разложения гармонического ряда $\sum_{k=1}^{n} \frac{1}{k} = \ln k + C + o(k)$.
Мы получили ассимптотическую формулу, где $A$ - некоторая положительная константа, зависящая от $\alpha, \beta, \gamma$. 
Далее нетрудно провести анализ сходимости ряда, при $x = 1$ ряд ведёт себя как эталонный и сходится при $\alpha + \beta - \gamma - 1 < -1 \implies \alpha + \beta < \gamma$. 
При $\alpha + \beta \geq \gamma$ ряд расходится. При $x = -1$ ряд сходится абсолютно при $\alpha + \beta < \gamma$ и сходится условно при $\alpha + \beta - \gamma - 1 < 0$ по признаку Даламбера. 
При $\alpha + \beta - \gamma - 1 \geq 0$ ряд расходится.  



\end{document}