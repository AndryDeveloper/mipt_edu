\documentclass[12pt]{article}
\usepackage[T2A]{fontenc}
\usepackage[utf8]{inputenc}
\usepackage{multirow}
\usepackage{caption}
\usepackage{subcaption}
\usepackage{amsmath}
\usepackage{amssymb}
\usepackage{changepage}
\usepackage{graphicx}
\usepackage{float}
\usepackage[english,russian]{babel}
\usepackage{amsmath, amsfonts, amssymb, amsthm, mathtools}
\usepackage{xcolor}
\usepackage{array}
\usepackage{hyperref}
\usepackage{physics}
\usepackage[top = 1.5cm, left = 1.5 cm, right = 1.5 cm, bottom = 3 cm]{geometry}
\usepackage{import}
\usepackage{xifthen}
\usepackage{pdfpages}
\usepackage{transparent}

\DeclareMathOperator*\lowlim{\underline{lim}}
\DeclareMathOperator*\uplim{\overline{lim}}

\newcommand{\incfig}[1]{
    \import{./figures/}{#1.pdf_tex}
}

\title{Матан третья домашка.}
\author{Шахматов Андрей, Б02-304}
\date{\today}

\begin{document}
\maketitle
\tableofcontents

\section{T.25}
Такое множество является объединением двух множеств $X = X_1 \cup X_2$ : 
\[
    X_1 = \left\{ (x, y) \mid x \in \mathbb{Q} \right\} 
\]
\[
    X_2 = \left\{ (x, y) \mid x \in \mathbb{Q} \right\}
\]
В свою очередь $X_1$:
\[
    X_1 = \bigcup_{x \in \mathbb{Q}}^{\infty} \left\{ (x, y) \mid y \in \mathbb{R} \right\}   
\]  
Так как $\left\{ (x, y) \mid y \in \mathbb{R} \right\}$ по существу является прямой, то оно измеримо 
с мерой $0$. Тогда $X_1$ в силу счётной аддитивности тоже измеримо с мерой $0$. Аналогично измеримо 
и $X_2$. Тогда $X$ измеримо так как является объединением измеримых.   
\section{T.29}
Я не уверен, но
\[
    \mu(X \setminus (X + t)) = \mu(X) - \mu 
\]

\section{8.3}
\[
    f^{-1}(\left\{ +\infty \right\} ) = \bigcap_{i=1}^{\infty} f^{-1}((i, +\infty] ) 
\] 
\[
    f^{-1}(\left\{ -\infty \right\} ) = A \setminus \bigcup_{i=1}^{\infty} f^{-1}((-i, +\infty ] ) 
\] 
\[
    f^{-1}(\mathbb{R}) = A \setminus \left( f^{-1}(\left\{ -\infty \right\} ) \cup f^{-1}(\left\{ +\infty \right\} ) \right) 
\]
\section{8.4}
\[
    f^{-1}((a, b)) = f^{-1}((a, +\infty]) \setminus \left( \bigcap_{i=1}^{\infty} f^{-1}((b - \frac{1}{i}, +\infty])  \right)   
\]
\section{8.13}




\end{document}