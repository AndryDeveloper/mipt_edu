\documentclass[12pt]{article}
\usepackage[T2A]{fontenc}
\usepackage[utf8]{inputenc}
\usepackage{multirow}
\usepackage{caption}
\usepackage{subcaption}
\usepackage{amsmath}
\usepackage{amssymb}
\usepackage{changepage}
\usepackage{graphicx}
\usepackage{float}
\usepackage[english,russian]{babel}
\usepackage{amsmath, amsfonts, amssymb, amsthm, mathtools}
\usepackage{xcolor}
\usepackage{array}
\usepackage{hyperref}
\usepackage{physics}
\usepackage[top = 1.5cm, left = 1.5 cm, right = 1.5 cm, bottom = 3 cm]{geometry}
\usepackage{import}
\usepackage{xifthen}
\usepackage{pdfpages}
\usepackage{transparent}

\newcommand{\incfig}[1]{
    \import{./figures/}{#1.pdf_tex}
}

\title{Матан первая домашка.}
\author{Шахматов Андрей, Б02-304}
\date{\today}

\begin{document}
\maketitle
\tableofcontents

\section{T1}
$$f(x, y) = \sqrt{xy}$$
Тогда область определения $D_f = \{(x, y) \mid (x, y) \in \mathbb{R}^2 \, (x \leq 0 \, \land y \leq 0) \lor (x \geq 0 \, \land y \geq 0)\}$
\begin{figure}[H]
    \centering
    \def\svgwidth{0.7\columnwidth}
    \incfig{T1_1}
    \caption{Область определения функции}
    \label{fig:T1_1}
\end{figure}
Область опредлеения будет замкнутым множеством, так как совпадает со своим замыканием. 
Из рисунка \ref{fig:T1_1} видно, что множество не является выпуклым, однако является линейно свзяным, ведь любые 2 точки можно соединить кривой проходящей через точку (0, 0).
$$f(x, y) = \frac{1}{x^4 + y^4 - 1}$$
Область определения данной функции $D_f = \{(x, y) \mid (x, y) \in \mathbb{R}^2 \, x^4 + y^4 \ne 1 \}$
\begin{figure}[H]
    \centering
    \def\svgwidth{0.7\columnwidth}
    \incfig{T1_2}
    \caption{Область определения функции}
    \label{fig:T1_2}
\end{figure}
Областью определения является всё пространство кроме кривой $x^4 + y^4 = 1$. Так как такая кривая является замкнутой, то её дополнение открыто,
а значит область определения открыта. Также область определения не является выпуклой, связной или линейно связной, так как разбивается кривой на два открытых непересекающихся
множества.
$$f(x, y) = \ln(1 - 2x - x^2 - y^2)$$
Тогда область определения $D_f = \{(x, y) \mid (x, y) \in \mathbb{R}^2 \, x^2 + y^2 < 1 - 2x\}$.
Решим полученное неравенство $x^2 + 2x + 1 + y^2 < 2$, что эквивалентво $(x + 1)^2 + y^2 < \sqrt{2}^2$, 
что соответствует открытому шару радиуса $\sqrt{2}$ с центром в точке $(-1, 0)$.
\begin{figure}[H]
    \centering
    \def\svgwidth{0.7\columnwidth}
    \incfig{T1_3}
    \caption{Область определения функции}
    \label{fig:T1_3}
\end{figure}

Так как область определения - открытый шар, то она открыта. Также область определения связна, линейно связна и выпукла.
\section{T2}
$$M = \{(e^t \cos{t}, e^t \sin{t}) \mid t \in \mathbb{R}\}$$
Данное множество является образом непрерывной кривой, потому оно линейно свзяно (любые две точки можно соединить данной кривой), потому $M$ - связно.
Данное множество не является открытым, поскольку содержит некоторые свои граничные точки. 
При этом замыкание множества содержит точку $(0, 0)$, однако сама кривая её не содержит, так как $R = \sqrt{x^2 + y^2} = e^t > 0$, что 
означает незамкнутость $M$.
\begin{figure}[H]
    \centering
    \def\svgwidth{0.4\columnwidth}
    \incfig{T2_graph}
    \caption{График кривой}
    \label{fig:T2}
\end{figure}
\section{T3}
$$M = {(x_1, x_2, x_3, x_4) \in \mathbb{R}^4 \mid x_1^2 + x_2^2 + x_3^2 < x_4^2}$$
Рассмотрим функцию $g:\mathbb{R}^4 \to \mathbb{R},\, g(x_1, x_2, x_3, x_4) = x_4^2 - x_1^2 + x_2^2 + x_3^2$, такая функция является непрерывной. Рассмотрим прообраз множества
$g^{-1}(Q), \, Q = (0, +\infty)$, при этом прообраз равен исследуемому множеству $M$. По топологическому
определению непрерывности прообраз открытого $Q$ открыт, а значит $M$ - открыто. Также множество $M$ не является линейно свзяным, так как
все кривые, соединяющие точки с координатами $x_4$ разных знаков должны проходить через точку с координатой $x_4 = 0$, которая не содержится в множестве $M$.
Тогда так как $M$ - открыто и не линейно связно, то оно не связно.
\section{T5}
Нужно доказать, что $\bigcap_{n \in \mathbb{N}}^{\infty}B_0\left(\frac{1}{n}\right) = \{0\}$.
С одной стороны $\forall k \in \mathbb{N} \, 0 \in B_0\left(\frac{1}{k}\right) \Rightarrow \{0\} \subseteq \bigcap_{n \in \mathbb{N}}^{\infty}B_0\left(\frac{1}{n}\right)$.
С другой стороны рассмотрим множество $A = \mathbb{R} \setminus \{0\}$, для любого $a \in A$
существует $k$ такое что $\frac{1}{k} < |a|$, а значит $a \not \in B_0\left(\frac{1}{k}\right) \Rightarrow \bigcap_{n \in \mathbb{N}}^{\infty}B_0\left(\frac{1}{n}\right) \not \in A \Rightarrow B_0\left(\frac{1}{k}\right) \Rightarrow \bigcap_{n \in \mathbb{N}}^{\infty}B_0\left(\frac{1}{n}\right) \in \{0\}$. Тогда через двойное включение получаем требуемый факт.
\section{T6}
Рассмотрим множество значений последовательности Гейне $a_n \rightarrow 0$. Данное множество не будет замкнутым, так как
его замыкае будет содержать $0$, но ни один из членов последовательности не равен $0$. А также каждая точка данного множества является членом последовательности $a_n$ и потому изолирована.
\section{2.39}
Рассмотрим две последовательности Гейне $x_n \to x_0$ и $y_k \to y_0$. Требуется доказать, что при условии
$$\lim_{n \to \infty} f(x_n, y_k) = B(y_k)$$ и $$\lim_{n \to \infty} f(x_n, y_n) = A$$ следует, что $$\lim_{k \to \infty} B(y_k) = A$$.
В силу существования первых двух пределов для достаточно больших $n, k$ выполняется:
$$|f(x_n, y_n) - A| < \epsilon$$
$$|f(x_n, y_k) - B(y_k)| < \epsilon$$
$$|f(x_n, y_k) - f(x_k, y_k)| < \epsilon$$
Последнее неравенство выполняется в силу фундаментальности последовательности $f(x_n, y_k)_n$.
Рассмотрим $|B(y_k) - A| < |B(y_k) - f(x_n, y_k)| + |f(x_n, y_k) - f(x_k, y_k)| + |f(x_k, y_k) - A| < 3\epsilon$. Что означает
$$\lim_{k \to \infty} B(y_k) = A$$
\section{Т10}
Пусть существует $g:\mathbb{R}^2 \to \mathbb{R}$ и $g$ - непрерывна и инъективна. Рассмотрим сужение $g_y: \mathbb{R} \to \mathbb{R}$, $g_y(x) = g(x, y)$.
Тогда $g_y$ также непрерывна, а значит он переводит связные множества в связные, из чего следует $g_y(\mathbb{R}) = Q(y)$, где $Q(y)$ - отрезок, интервал или полуинтервал.
Тогда так как $g$ - инъективна выполняется: $\mathbb{R} = \bigsqcup_{y \in \mathbb{R}} Q(y)$.
Докажем промежуточную лемму: инъёктивная и непрерывная функция монотонна. Предположим противное:
$\exists x_1 < x_2 < x_3 \mid f(x_1) < f(x_2) > f(x_3) \lor f(x_1) > f(x_2) < f(x_3)$, строгие знаки 
получены с учётом инъёктивности. Без ограничения общности рассмотрим первый вариант $f(x_1) < f(x_2) > f(x_3)$.
Пусть $f(x_3) > f(x_1)$, тогда по теореме о промежуточных значениях существует $x \in (x_1, x_2) \mid f(x) = f(x_3)$, 
тогда так как $x < x_3 \Rightarrow x \not = x_3$, но $f(x) = f(x_3)$ - противоречие с инъективностью.
Тогда оказывается, что каждая из $g_y$ - монотонна. Так как функция $g_y$ монотонна то по теореме об обратной
функции обратная $f^{-1}$ - непрерывна, а значит $f$ переводит открытые в открытые. А значит все множества
$Q(y)$ - интервалы. Тогда так как из покрытия открытого множества интервалами можно выбрать счётное подпокрытие $\mathbb{R} = \bigsqcup_{n \in \mathbb{N}} Q(y_n)$.
Но тогда мы получили дизъюнктное разбиение действительной прямой непустыми интервалами - противоречие со связностью.
\section{T11}
\end{document}