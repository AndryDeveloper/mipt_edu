\documentclass[12pt]{article}
\usepackage[T2A]{fontenc}
\usepackage[utf8]{inputenc}
\usepackage{multirow}
\usepackage{caption}
\usepackage{subcaption}
\usepackage{amsmath}
\usepackage{amssymb}
\usepackage{changepage}
\usepackage{graphicx}
\usepackage{float}
\usepackage[english,russian]{babel}
\usepackage{amsmath, amsfonts, amssymb, amsthm, mathtools}
\usepackage{xcolor}
\usepackage{array}
\usepackage{hyperref}
\usepackage{physics}
\usepackage[top = 1.5cm, left = 1.5 cm, right = 1.5 cm, bottom = 3 cm]{geometry}
\usepackage{import}
\usepackage{xifthen}
\usepackage{pdfpages}
\usepackage{transparent}

\newcommand{\incfig}[1]{
    \import{./figures/}{#1.pdf_tex}
}

\title{Теория вероятности 1.}
\author{Шахматов Андрей, Б02-304}
\date{\today}

\begin{document}
\maketitle
\tableofcontents

\section{Введение}
\[
    p(A+B) = p(A) + p(B) - p(AB)
\]
\begin{figure}[H]
    \centering
    \def\svgwidth{0.4\columnwidth}
    \incfig{tp1}
    % \caption{}
    \label{fig:tp1}
\end{figure}
\[ 
    \overline{A + B} = \overline{A}\cdot\overline{B}
\]
\[
    \overline{A \cdot B} = \overline{A} + \overline{B}   
\]
\[
    A(B + C) = A B + B C
\]
\[
    A + BC = (A + B)(A + C)
\]
\[
    A - B \coloneqq A \setminus B
\]
\section{Независимость}
\fbox{События независимы если \(p(AB) = p(A)p(B)\) }
\begin{table}[H]
    \centering
    \begin{tabular}{c|c|c}
             & 0 & 1  \\
             \hline
             0 & \(\frac{1}{4}\)  & \(\frac{1}{4}\)   \\
             \hline
             1 & \(\frac{1}{4}\)  & \(\frac{1}{4}\)  \\
    \end{tabular}
    % \caption{Таблица вероятности}
    \label{tab:1}
\end{table}
\fbox{Условная вероятность: \(p(A \mid B) = \frac{p(AB)}{p(B)}\) }
\\Для независимых событий:
\[
    p(A \mid  B) = p(A)
\]
\end{document}