\documentclass[12pt]{article}
\usepackage{caption}
\usepackage{subcaption}
\usepackage{amsmath}
\usepackage{changepage}
\usepackage{graphicx}
\usepackage{float}
\usepackage[english,russian]{babel}
\usepackage{amsmath, amsfonts, amssymb, amsthm, mathtools}
\usepackage{xcolor}
\usepackage{array}
\usepackage{hyperref}
\usepackage[top = 1.5cm, left = 1.5 cm, right = 1.5 cm, bottom = 3 cm]{geometry}

\title{Лабораторная работа 1.1.1}
\author{Никита Лучинкин, Б02-304}
\date{\today}

\begin{document}
\begin{titlepage}
	\begin{center}
		{\large МОСКОВСКИЙ ФИЗИКО-ТЕХНИЧЕСКИЙ ИНСТИТУТ (НАЦИОНАЛЬНЫЙ ИССЛЕДОВАТЕЛЬСКИЙ УНИВЕРСИТЕТ)}
	\end{center}
	\begin{center}
		{\large Физтех-школа физики и исследований им. Ландау}
	\end{center}
	
	
	\vspace{3cm}
	{\huge
		\begin{center}
			\textbf{Определение удельного сопротивления нихромовой проволоки с анализом систематических и случайных погрешностей}
		\end{center}
	}
	\vspace{2cm}
	\begin{flushright}
		{\LARGE Автор:\\ Лучинкин Никита Александрович \\
			\vspace{0.2cm}
			Б02-304}
	\end{flushright}
	\vspace{7 cm}
	\begin{center}
		Долгопрудный 2023
	\end{center}
\end{titlepage}

\begin{table}[H]
    \begin{minipage}{0.15\textwidth}
        \textcolor{white}{a}
    \end{minipage}
    \small
    \begin{minipage}{0.7\textwidth}
    Измерено удельное сопротивление нихромовой проволоки двумя свособами. Для измерения сопротивления рассмотрены два метода: анализ вольт-амперной характеристики и измерение напрямую при помощи мостовой схемы.\\
    В результате двумя способами было получено значение удельного сопротивления нихрома, приведено сравнение этих двух способов и сделаны выводы касательно их применимости, а также проведен анализ погрешностей полученных величин. результатов.
    \end{minipage}

    
\end{table}


\section{Введение} %Ё!!!Ё!Ё!ёЁ
Чтобы вычислять удельное сопротивления материалов, важно научиться с хорошей точностью измерять сопротивление образцов и их геометрические размеры. Для измерения сопротивления существует множество методов, однако двумя из самых простых являются измерение прибором напрямую и анализ вольт-амперной характеристики (ВАХ) образца.
\\ 
\\
Для разных технологических задач существуют различные требования к точности, а также различные ограничения. В то время как измерения напрямую дают результат на основе одного измерения с фиксированной точностью, анализ ВАХ требует множество измерений, от количества которых зависит точность получаемого значения.  \\
Основной задачей данной работы является выяснить, в каких случаях выгодно применять один или другой способ измерения сопротивления, реализовав оба на практике, сравнив их особенности и проанализировав погрешности.


\section{Методика}
Для определения удельного сопротивления $\rho$ тонкой длинной проволоки может быть использована общеизестная формула: $\rho = \dfrac{R\cdot S}{l}$, где $R$ — сопротивление проволоки, $S$ - площадь ее поперечного сечения, а $l$ - ее длина. Таким образом для определения $\rho$ необходимо узнать сопротивление проволоки и ее геометрические размеры. 

Для проведения прямых измерений сопротивления был использован прибор мост Уитстона Р4833,  объединяющий в себе мостовую схему и магазин сопротивлений. Он позволяет на основе одного измерения с точностью до пяти значащих цифр определить величину сопротивления образца.\\
Для измерения ВАХ рассмотрены две простейшие электрические схемы с вольтметром и амперметром (характеристики приборов указаны в \hyperref[sec: characteristics]{приложении}), и выбрана одна из этих схем из соображений уменьшения систематической ошибки. Для анализа полученной зависимости применён метод наименьших квадратов.\\
\\
Длина образцов проволоки составляет десятки сантиметров, поэтому для ее измерения выбрана линейка. Диаметр проволоки, с помощью которого рассчитывается площадь ее сечения, имеет характерные размеры меньше миллиметра, поэтому для его измерения выбраны более точные приборы - штангенциркуль и микрометр, и из них определен более подхоящий под нашу задачу.\\
\\
Все измерения проводились для трех образцов проволоки разной длины, после чего результаты усреднены для увеличения точности.
\section{Результаты измерений и их обсуждение}
\subsection{Измерение сопротивления}
\subsubsection{Мост Уитстона}\\
\text{С помощью мостовой схемы получены значения:}\\
Результаты измерений сопротивления для различных участков проволоки длиной $l1 = 50$см, $l2 = 30$см, $l3 = 20$см с помощью моста Уитстона имеют следующий вид:
\begin{align}

R(l_1) &= (5.3164 \pm 0.0001)\,\Omega, \, \varepsilon_R_1 = 0.002\% 

R(l_2) &= (3.2227\pm 0.0001)\,\Omega, \, \varepsilon_R_2 = 0.003\%, 

R(l_3) &= (2.1204 \pm 0.0001)\,\Omega, \, \varepsilon_R_3 = 0.004\%.
\end{align}

Здесь $\varepsilon$ - величины относительных погрешностей

\subsubsection{Измерения и анализ ВАХ в простейшей электрической цепи}
При измерени сопротивления с использованием вольтметра и амперметра в простейшей электрической цепи есть два варианта подключения амперметра. В одной из цепей амперметр подключен параллельно вольтметру, а в другой - последовательно ему (обе схемы и их сравнение см. в \hyperref[sec: comparison]{приложении}).\\
В результате оценки систематической ошибки для обоих схем, получено, что при выборе одной из них ошибка на несколько порядков меньше, чем при выборе другой. Поэтому измерения проводились, используя только  схему с меньшей систематической погрешностью. \\
\\
Таблица с измерениями напряжения и тока длля вольт-ампперной характеристики представлена в \hyperref[sec: measurements]{приложении}.

Анализ измерений с помощью метода наименьших квадратов для модели линейной зависимости $U(I)$ с доверительной вероятностью 95\% после округления дает следующие результаты:\\


\begin{figure}[H] %График
    \centering
    \includegraphics[width = 0.7\textwidth]{Resistance graph.png}
    \centering
    \caption*{Вольт-амперные характеристики образцов (кресты погрешностей в масштабе графиков не видны).}
\end{figure}


%\begin{align*}
%\centering % аналитические зависимости на графиках
% U_1 &= (5.32 \pm 0.03)\Omega\cdot I_1 + (0 \pm 3)   \text{мВ - для образца 50 см}\\
% U_2 &= (3.25 \pm 0.04)\Omega\cdot I_2 + (1 \pm 4)   \text{мВ - для образца 30 см}\\
% U_3 &= (2.12 \pm 0.01)\Omega\cdot I_3 + (1 \pm 1)   \text{мВ - для образца 20 см}\\  
%\end{align*}

На графике ВАХ образцов проволоки представлены прямыми, с хорошей точностью проходящими через начало координат, угловой коэффициент которых является величиной проводимости указанных образцов. Характер зависимости показывает, что металлические проводники на примере нихромовой проволоки действительно описываются моделью омического сопротивления, согласно которой напряжение на элементе всегда пропорционально силе протекающего через него тока.\\
Через угловые коэффициенты, рассчитанные с помощью метода наименьших квадратов, может быть определено сопротивления образцов:

\begin{align*}
\centering % посчитанные сопротивления
 R(l_1) &= (5.32 \pm 0.03)\Omega, \,\varepsilon_R_1 = 0.6\%\\
 R(l_2) &= (3.25 \pm 0.04)\Omega, \,\varepsilon_R_2 = 1.2\%\\
 R(l_3) &= (2.12 \pm 0.01)\Omega, \,\varepsilon_R_3 = 0.5\% 
\end{align*}

При этом большее сопротивление соответствует образцу большей длины в соответствии с используемой формулой $R = \dfrac{\rho\cdot l}{S}$, согласно которой сопротивление прямо пропорционально длине образца.\\
\\
Сравнив случайную погрешность измеренных величин сопротивления со систематической погрешностью, вносимой неидеальностью выбранной для измерений электрической схемы, можно прийти к выводу, что последней можно пренебречь (см. \hyperref[sec: negligence]{вложение}).\\
\\
Для нашей выборки, в сравнении с измерением мостовой схемой, точность получилась заметно меньше. Для того, чтобы точность данного метода сравнилась или превысила точность измерений мостовой схемой, необходима значительно большая выборка. Это реализуемо, однако в рамках лабораторной работы требуемого для этого временного ресурса  доступно не было. \\Также важно отметить, что полученные значения, за исключением точности, в хорошей степени соответствуют результатам измерения мостом Уитстона. Это указывает на действенность обоих методов измерения сопротивления.


\subsection{Измерение геометрических размеров}
\subsubsection{Измерение длины}
В лабораторной работе использованы три участка образца проволоки, измерения длины которых с учетом приборной погрешности линейки представлены ниже: \\
$l_1 = (50\pm 0.1)\text{ см}$, $l_2 = (30\pm 0.1)\text{ см}$, $l_3 = (20\pm 0.1)\text{ см}$ \\

\subsubsection{Измерение диаметра}
В \hyperref[sec: length measurements]{таблицах}приведены результаты многократных измерений диаметра проволоки, проведенных на разных участках образца, с использованием микрометра и штангенциркуля.\\

После сравнения измерений микрометром и штангенциркулем и учета погрешностей (см в \hyperref[sec: diameter measurements]{приложении}), получено следующее значение диаметра:
$d = (0.363 \pm 0.006)$ мм

Далее, несложно вычислить площадь сечения проволоки: $S = \dfrac{\pi d^2}{4} \pm 2\varepsilon_d S  = (0.103 \pm 0.004)$ мм$^2$


\subsection{Расчет удельного сопротивления}
Как было упомянуто, удельное сопротивление рассчитывается по общеизвестной формуле\\ $\rho = \dfrac{R\cdot S}{l}$.
Из полученных данных удельное сопротивление определено двумя способами (используя покцазания мостовой схемы или результат анализа вольт-амперных характеристик). В обоих случаях используется усреднение значений для трех образцов проволоки разных длин для увеличения точности результата.
\subsubsection*{Расчет со значениями сопротивления, полученными на мостовой схеме:}
Используя показания измерительного моста, с учетом погрешностей (рассчеты для этого и следующего пунктов в \hyperref[sec: resistivity1]{приложении}) получены значения:
\begin{align*}
    %\rho(l_1) &= (1.085 \pm 0.043)\Omega\cdot\text{м}\\
    %\rho(l_2) &= (1.096 \pm 0.0467)\Omega\cdot\text{м}\\
    %\rho(l_3) &= (1.081 \pm 0.0503)\Omega\cdot\text{м}\\
    \centering
    \rho(l_1) &= (1.09 \pm 0.05)\Omega\dfrac{\text{мм}^2}{\text{м}}\\
    \rho(l_2) &= (1.09 \pm 0.05)\Omega\dfrac{\text{мм}^2}{\text{м}}\\
    \rho(l_3) &= (1.08 \pm 0.05)\Omega\dfrac{\text{мм}^2}{\text{м}}\\
\end{align*}
Эти значения при усреднении дают величину $\rho &= (1.09 \pm 0.05)\Omega\dfrac{\text{мм}^2}{\text{м}}, \, \varepsilon_\rho = 4.6\%\\$


\subsubsection*{Рассчет со значениями сопротивления, полученными из анализа ВАХ:}

\begin{align*}
    %\rho(l_1) &= (1.085 \pm 0.0443)\Omega\cdot\text{м}\\
    %\rho(l_2) &= (1.105 \pm 0.0490)\Omega\cdot\text{м}\\
    %\rho(l_3) &= (1.081 \pm 0.0503)\Omega\cdot\text{м}\\
    \centering
    \rho(l_1) &= (1.09 \pm 0.05)\Omega\dfrac{\text{мм}^2}{\text{м}}\\
    \rho(l_2) &= (1.11 \pm 0.05)\Omega\dfrac{\text{мм}^2}{\text{м}}\\
    \rho(l_3) &= (1.08 \pm 0.05)\Omega\dfrac{\text{мм}^2}{\text{м}}\\
\end{align*}
Эти значения при усреднении дают величину $\rho &= (1.09 \pm 0.05)\Omega\dfrac{\text{мм}^2}{\text{м}},\, \varepsilon_\rho = 4.6\%$\\ Значение получилось таким же, как и в предыдущих рассчетах. Отсюда следует, что для решения нашей задачи точность измерений мостом Уитстона оказалась избыточной, поскольку основным источником погрешности результата оказались неточности в измерениях геометрических размеров.\\
\\
Сравнение со справочным значением удельного сопротивления нихрома проблематично, поскольку неизвестна точная марка нихрома, из которой изготовлены образцы проволоки. Поэтому сравнить результат можно только с диапозоном удельных сопротивлений нихрома, приведенном в справочнике: \,$\rho_\text{справ} = [1.05; 1.4] \,\Omega\dfrac{\text{мм}^2}{\text{м}}$. Наш результат попадает в этот диапазон, однако для более объективной оценки его точности необходима информация о марке нихрома, из которого изготовлена проволока.

\section{Выводы}

Измерение с помощью моста Уитстона - быстрый и действенный метод измерения омического сопротивления, который дает результат на основе одного измерения с хорошей точностью. Однако эта точность фиксирована, и в некоторых случаях она может быть недостаточной.\\
Анализ ВАХ, с другой стороны, - довольно трудоемкий процесс, отдельные измерения которого значительно уступают по точности предыдущему методу. Преимущества данного метода включают возможность минимизировать погрешность за счет большой выборки. Использование анализа ВАХ для измерения сопротивления актуально в случае, если требуется очень высокая точность, и при этом имеется возможность провести значительное количество измерений для набора достаточной выборки. Анализ ВАХ также может пригодиться в том случае, когда неизвестно, является ли характер сопротивления омическим.\\
%Измерения проводились для трех образцов проволоки различной длины для увеличения точности результата.

\section{Использованная литература}
\begin{thebibliography}
\href{https://dpva.ru/Guide/GuidePhysics/ElectricityAndMagnethism/ElectricalResistanceAndConductivity/ElectricResistancePerVolumeMetalls1/}{Онлайн-справочник удельных сопротивлений}\\
\href{https://npm.mipt.ru/books/lab-intro/main.pdf}{Теория ошибок в эксперименте, метод наименьших квадратов}

\section*{Приложения}
\subsection*{Измерения}
\textbf{Измерения диаметров:}
\begin{table}[H]
\label{sec: length measurements}
    \begin{minipage}[c]{0.55\textwidth}
    \centering
    \small
    \begin{tabular}{|c|c|c|c|c|c|c|c|c|}
        \hline
        $N_\text{измерения}$ & 1 & 2 & 3 & 4 & 5 & 6 & 7 & 8 \\ \hline
        $d$, мм & 0,36 & 0,36 & 0,36 & 0,37 & 0,36 & 0,37 & 0,36 & 0,36\\ \hline
    \end{tabular}
    \captionof{table}{Измерения микрометром}
    \end{minipage}
    \begin{minipage}[c]{0.5\textwidth}
    \centering
    \small
    \begin{tabular}{|c|c|c|c|c|c|c|c|c|}
        \hline
        $N_\text{измерения}$ & 1 & 2 & 3 & 4 & 5 \\ \hline
        $d$, мм & 0,4 & 0,4 & 0,4 & 0,4 & 0,4 \\ \hline
        
    \end{tabular}
    \captionof{table}{Измерения штангенциркулем}
    \end{minipage}
\end{table} % Измерения диаметра

\textbf{Измерения напряжений и сил тока для ВАХ трех образцов:}
\subsection*{Измерения напряжения и тока}
\label{sec: measurements}
\begin{table}[H]
    \begin{minipage}[c]{\textwidth}
    \centering
    \footnotesize
    \begin{tabular}{|m{0.51 cm}|c|c|c||c||c|c|c||c||c|c|c|}
        \hline
$n_\text{изм}$ & $U_V$, mV & $I_A$, mA & $\frac{U_V}{I_A},\,\Omega$ & & $U_V$, mV & $I_A$, mA & $\frac{U_V}{I_A},\,\Omega$ & & $U_V$, mV & $I_A$, mA & $\frac{U_V}{I_A},\,\Omega$\\ \hline
1      & 256   & 48.3  & 1.3251 & &196   & 60.1  & 0.81531 && 128   & 60.5  & 0.52893 \\
2      & 320   & 60.2  & 1.3289 & &228   & 70.4    & 0.81429 && 148   & 69.8  & 0.53009 \\
3      & 372   & 70.6   & 1.3286 & &260   & 80.7   & 0.81250 && 172   & 80.4  & 0.53483 \\
4      & 424   & 79.5  & 1.3333 & &296   & 90.9  & 0.81408 && 192   & 90.1  & 0.53274 \\
5      & 484   & 90.0  & 1.3430 & &324   & 100.2   & 0.81000 && 216   & 101.0   & 0.53465 \\
6      & 528   & 99.8  & 1.3226 & &356   & 110.8   & 0.80909 && 236   & 111.4   & 0.53153 \\
7      & 584   & 110.5 & 1.3213 & &388   & 120.6   & 0.80833 && 256   & 120.5 & 0.53112 \\
8      & 612   & 115.2   & 1.3304 & &460   & 140.0   & 0.82143 && 276   & 130.1 & 0.53036 \\
9      & 548   & 103.4   & 1.3301 & &412   & 128.1   & 0.80469 && 260   & 122.5 & 0.53061 \\
10     & 484   & 90.5  & 1.3370 & &360   & 113.8   & 0.79646 && 244   & 115.2   & 0.53043 \\
11     & 424   & 80.7    & 1.3250 & &340   & 105.9   & 0.80952 && 224   & 105.2 & 0.53232 \\
12     & 372   & 70.3  & 1.3229 & &308   & 95.2  & 0.80882 & &204   & 95.3  & 0.53515 \\
13     & 320   & 60.1  & 1.3311 & &276   & 85.2  & 0.80986 && 180   & 84.3  & 0.53381 \\
14     & 256   & 48.0  & 1.3251 & &240   & 75.0  & 0.80000 && 160   & 75.2  & 0.53191 \\\hline
        
    \end{tabular}
    \captionof{table}{Измерения для образцов длиной 50 см, 30 см и 20 см}
    \end{minipage}
    \vspace{0.5cm}
\end{table} % Измерения U I
Замечание к таблице измерений: абсолютная погрешность измерения напряжения и тока составляют 2 мВ и 0.1 мА, согласно \hyperref[sec: characteristics]{характеристике приборов}. Поэтому, относительная погрешность всех измерений составляет меньше половины процента. В масштабах графика она не видна, и для наших рассчетов ее можно считать несущественной. В нашей модели существенной является погрешность углового коэффициента в методе наименьших квадратов, который не учитывает погрешности индивидуальных точек (в данном случае такая модель применима, как раз вследствие несущественности этих погрешностей).
\subsection*{Хараетристики измерительных приборов}
\label{sec: characteristics}
\begin{table}[H]
\centering
\caption{Устройство вольтметра}
\begin{tabular}{|c|c|}
\hline
Система                              & Магнитоэлектрическая \\\hline
Класс точности                       & 0,5                   \\\hline
Шкала                                & линейная, 150 делений \\\hline
Предел измерений                     & 600 мВ                \\\hline
Цена деления                         & 4 мВ                  \\\hline
Чувствительность                     & 250 дел/В             \\\hline
Внутреннее сопротивление             & $R_v$ = 5кОм           \\\hline
Погрешность при считывании со шкалы  & 2 мВ                  \\\hline
Макс. Погрешность по классу точности & 3 мВ                 \\\hline
\end{tabular}
\end{table}
\begin{table}[H]
\caption{Устройство амперметра}
\centering
\begin{tabular}{|c|c|}
\hline
Система                              & Цифровая\\\hline
Разрядность дисплея                  & 5 единиц\\\hline
Предел измерении                     &  Автоматически определяется прибором\\\hline
Абсолютная погрешность               &  Единица младшего разряда\\\hline
Внутреннее сопротивление             &  $1.2$ $\Omega$\\\hline
\end{tabular}    
\end{table}

\subsection*{Сравнение систематических ошибок для измерительных схем}
\label{sec: comparison}
\begin{figure}[H]% Изображения схем
\centering
    \begin{subfigure}{0.4\textwidth}
    
    \includegraphics[width = \textwidth]{scheme-1.png}
    \large
    
    \caption*{Схема 1 - последовательное подключение}
    \label{fig: Схема 1}
    \end{subfigure}
    \begin{subfigure}{0.37\textwidth}
    \centering
    \includegraphics[width = \textwidth]{scheme-2.png}
    \large
    \caption*{Схема 2 - параллельное подключение}
    \label{fig: Схема 2}
    \end{subfigure}
\end{figure}

Приборы в обеих схемах неидеальны и за счет своих внутренних сопротивлений вносят в результаты измерений систематическую погрешность. 
\textbf{Для первой схемы}: обозначим силу тока, протекающего через образец проволоки сопротивлением $R_0$ как $I_0$, а отношение покцазаний вольтметра и амперметра как $R^*$
$$ I_A = I_0 + I_V $$
$$ I_V \cdot R_V = I_0 \cdot R_0 $$
$$ R^* = \dfrac{U_V}{I_A} = \dfrac{I_0\cdot R_0}{I_0 \dfrac{R_v + R_0}{R_V}}$$
Или, пользуясь тем, что3 $R_0 << R_V$:
$$ R^* = R_0\left(1 - \dfrac{R_0}{R_V} \right)$$
$$ R_0 = R^*\left(1 + \dfrac{R^*}{R_V}\right)$$
Систематическая погрешность метода измерений по такой схеме будет составлять: $$\Delta R_1 = \dfrac{R^*^2}{R_V}$$

\textbf{Для второй схемы}: 
$$ R^* = \dfrac{U_V}{I_A} = R_0 + R_A$$
$$ \Delta R_2 = -R_A$$
Сопротивление амперметра $R_A = 1.2\, \Omega$ по порядку схоже с сопротивлением образца проволоки, а $R_V = 5$ к$\Omega$, значит первая схема вносит отклонение $\Delta R$ на три порядка меньше, чем вторая схема. Поэтому целесообразнее для измерений использовать цепь, соответствующую первой схеме. \\
\label{sec: negligence}
Сравним систематическую ошибку измерения для выбранной схемы $\Delta R_1 = \dfrac{R^*^2}{R_V}$ с ошибкой определения углового коэффициента в методе наименьших квадратов.\\
$\Delta R_1 \approx \dfrac{(5\Omega)^2}{5\text{ к}\Omega} = 5 \text{ м}\Omega$\\
$\Delta \left(\dfrac{U_V}{I_A}\right) \approx 20 \text{ м}\Omega$ \\
Сдвиг, в несколько раз меньший погрешности, не превнесет большей точности, поэтому им можно пренебречь.

\subsection*{Рассчеты погрешностей}
\subsubsection*{Погрешность измерений диаметра проволоки}

\label{sec: diameter measurements}
Измерения штангенциркулем обладают значительной приоборной погрешностью ($d = (0.4 \pm  0.05) \text{мм}$), из-за которой случайная погрешность у измерений отсутствует (все результаты измерений одинаковы). Ввиду этого не имеет смысла проводить большое количество измерений и рассматривать широкую выборку. \\

Для измерений микрометром, получено, что $d = 0.36  \text{ мм}$, при этом приборная погрешность составляет $\Delta d_\text{пр} = 5$ мкм.\\
Дисперсия значений диаметра для выборки из 8 измерений вычисляется по формуле \\$\sigma_d = \sqrt{\dfrac{\sum_{i = 1}^{n}  (x_i - \langle x \rangle)^2}{n(n-1)}} =$  1.7 мкм.
Полная абсолютная погрешность, как корень из суммы квадратов статистической и приборной, равна $\Delta d = 5.3$ мкм $\approx 6$ мкм, а относительная погрешность составляет $\varepsilon_d = 1.7 \% $\\
Так как измерения микрометром дают более точный результат, именно они будут использованы для всех рассчетов.

\subsubsection*{Погрешность удельного сопротивления}
\label{sec: resistivity1}
Погрешность удельного сопротивления для отдельного образца рассчитывается по формуле: 
\begin{align*}

    \Delta \rho &= \sqrt{ \left(\dfrac{\partial \rho}{\partial R}\Delta R\right)^2 + \left(\dfrac{\partial \rho}{\partial S}\Delta S\right)^2 + \left(\dfrac{\partial \rho}{\partial \l}\Delta l\right)^2}\\\\
    
    \Delta \rho &= \dfrac{RS}{l}\sqrt{ \left(\dfrac{\Delta R}{R}\right)^2 + \left(\dfrac{\Delta S}{S}\right)^2 + \left(\dfrac{\Delta l}{l}\right)^2}

    
\end{align*}

\end{document}