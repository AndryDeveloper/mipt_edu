\documentclass[12pt]{article}
\usepackage[T2A]{fontenc}
\usepackage[utf8]{inputenc}
\usepackage{multirow}
\usepackage{caption}
\usepackage{subcaption}
\usepackage{amsmath}
\usepackage{amssymb}
\usepackage{changepage}
\usepackage{graphicx}
\usepackage{float}
\usepackage[english,russian]{babel}
\usepackage{amsmath, amsfonts, amssymb, amsthm, mathtools}
\usepackage{xcolor}
\usepackage{array}
\usepackage{hyperref}
\usepackage{physics}
\usepackage[top = 1.5cm, left = 1.5 cm, right = 1.5 cm, bottom = 3 cm]{geometry}
\usepackage{import}
\usepackage{xifthen}
\usepackage{pdfpages}
\usepackage{transparent}

\newcommand{\incfig}[1]{
    \import{./figures/}{#1.pdf_tex}
}

\title{Дифференциальные уравнения 2.}
\author{Шахматов Андрей, Б02-304}
\date{\today}

\begin{document}
\maketitle
\tableofcontents

\section{Линейные уравнения с постоянными коэффициентами}
\subsection*{8.5}
\[
    y^{\prime\prime} + 5y^{\prime} + 6y = 0
\]
Характеристическое уравнение имеет вид: 
\[
    \lambda^2 + 5\lambda + 6 = 0 \implies 
    \begin{dcases}
        \lambda = -2 \\
        \lambda = -3
    \end{dcases}
\]
Тогда общее решение имеет вид 
\[
    y = C_1 e^{- 2x} + C_2 e^{-3x}
\]

\subsection*{8.8}
\[
    y^{\prime\prime} - 2y^{\prime} + 10y = 0
\]
Характеристическое уравнение имеет вид: 
\[
    \lambda^2 - 2\lambda + 10 = 0 \implies 
    \begin{dcases}
        \lambda = 1 + 3i \\
        \lambda = 1 - 3i
    \end{dcases}
\]
Общее решение имеет вид: 
\[
    y = C_1 e^{\lambda x} \cos 3x + C_2 e^{\lambda x} \sin 3x
\]
\subsection*{8.25}
\[
    y^{\prime\prime\prime\prime} - 5 y^{\prime\prime\prime} + 7 y^{\prime\prime} - 3y^{\prime} = 0
\]
Сделаем замену $z = y^{\prime}$, тогда 
\[
    z^{\prime\prime\prime} - 5 z^{\prime\prime} + 7z^{\prime} - 3z = 0
\] 
Характеристическое уравнение имеет вид: 
\[
    \lambda^3 - 5\lambda^2 + 7z - 3 = 0 \implies 
    \begin{dcases}
        \lambda = 3 \\
        \lambda = 1 - \text{кратный корень} 
    \end{dcases}
\]
Решением будет являться
\[
    z = C_1 e^{3x} + C_2 x e^x + C_3 e^x
\]
Проинтегрируем это уравнение и получим вырадение для $y$:
\[
    y = C_1^{\prime} e^{3x} + C_2^{\prime} x e^x + C_3^{\prime} e^x + C_4
\] 
\subsection*{8.36}
\[
    y^{\prime\prime\prime\prime} + 3y^{\prime\prime} + 2y = 0
\]
Характеристическое уравнение имеет вид: 
\[
    \lambda^4 + 3\lambda^2 + 2 = 0 \implies 
    \begin{dcases}
        \lambda^2 = -2 \\
        \lambda^2 = -1
    \end{dcases} \implies 
    \begin{dcases}
        \lambda = -\sqrt{2} i \\
        \lambda = \sqrt{2} i \\
        \lambda = -i \\
        \lambda = i
    \end{dcases}
\]
Решение уравнения будет иметь вид: 
\[
    y = C_1 \cos x\sqrt{2}  + C_2 \sin x\sqrt{2}  + C_3 \cos x + C_4 \sin x
\]
\subsection*{8.47}
\[
    y^{\prime\prime} + 4y = 4xe^{-2x} - \sin 2x
\]
Сначала найдём общее решение 
\[
    y^{\prime\prime} + 4y = 0
\]
С учётом решений характеристических уравнений $\lambda^2 = -4$ общее решение имеет вид 
\[
    y = C_1 \sin 2x + C_2 \cos 2x
\]
(((((Доделать)))))

\subsection*{8.69}
\[
    y^{\prime\prime} + 2y^{\prime} + y = xe^{-x}
\]
Характеристическое уравнение имеет вид: 
\[
    \lambda^2 + 2\lambda + 1 = 0 \implies \lambda = -1 - \text{кратный корень} 
\]
Общее решение имеет вид: 
\[
    y = (C_1 + x C_2) e^{-x}
\]
Методом вариации постоянной 
\[
    \begin{dcases}
        C_1^{\prime} e^{-x} + C_2^{\prime} (1 + x) e^{-x} = 0 \implies C_1^{\prime} + (1 + x) C_2^{\prime} = 0  \\
        -C_1^{\prime} e^{-x} - C_2^{\prime} x e^{-x} = x e^{-x} \implies -C_1^{\prime} - C_2^{\prime} x = x
    \end{dcases}
\]
Тогда 
\[
    \begin{dcases}
        C_2^{\prime} = x \\
        C_1^{\prime} = -x - x^2
    \end{dcases} \implies 
    \begin{dcases}
        C_2 = \frac{x^2}{2} + A \\
        C_1 = -\frac{x^2}{2} - \frac{x^3}{3} + B
    \end{dcases}
\]
И финальное решение имеет вид
\[
    y = \left( \frac{x^2}{2} + A \right) x e^{-x} + \left( -\frac{x^2}{2} - \frac{x^3}{3} + B \right) e^{-x}  
\]
((((Не понятно чёт не сошлось))))

\subsection*{8.159}
\[
    y^{\prime\prime} + y = -\frac{\cos^2 x}{\sin^2 x}
\]
Сделаем замену $z = - y + 1$, тогда 
\[
    -z^{\prime\prime} - z + 1 = -\frac{\cos^2 x}{\sin^2 x} \implies z^{\prime\prime} + z = 1 + \frac{\cos^2 x}{\sin^2 x} = \frac{1}{\sin^2 x}
\] 
После замены $q \leftrightarrow x - \frac{\pi}{2}$ имеем уравнение 
\[
    z^{\prime\prime} + z = \frac{1}{\cos^2 q}
\] 
Такой пример был разобран перед параграфом, а значит решение: 
\[
    - y + 1 = A \sin x + B \cos x + \frac{1}{2} \ln \left\vert \frac{\cos x + 1}{\cos x -1} \right\vert \cos x - 1 
\]
\subsection*{8.195}
\[
    x^2 y^{\prime\prime} - 3xy^{\prime} + 4y = 4x^3
\]
Подставим $x = e^t, y^{\prime} = e^{-t} y^{\prime}_t, y^{\prime\prime} = e^{-2t} (y^{\prime\prime}_t - y^{\prime}_t)$. 
Тогда 
\[
    y^{\prime\prime} - 4y^{\prime} + 4y = 4e^{3t}
\] 
Общее решение имеет вид 
\[
    y = (A + Bt) e^{2t}
\]
Частное решение ищем в виде $y = a e^{3t}$: 
\[
    9a e^{3t} - 12 a e^{3t} + 4a e^{3t} = 4e^{3t} \implies 
    a = 4   
\]
Тогда решение: 
\[
    y = (A + Bt) e^{2t} + 4 e^{3t} = \left( A + \frac{B}{\ln x} \right) x^2 + 4x^3  
\]
\subsection*{8.196}
\[
    x^2 y^{\prime\prime} + xy^{\prime} + y = 10x^2
\]
Подставим $x = e^t, y^{\prime} = e^{-t} y^{\prime}_t, y^{\prime\prime} = e^{-2t} (y^{\prime\prime}_t - y^{\prime}_t)$. 
\[
    y^{\prime\prime} + y = 10 e^{2t} 
\]
Общее решение иммет вид: 
\[
    y = A \cos x + B \sin x
\]
В качестве частного решение подойдёт $y = 2 e^{2t}$
Тогда решение: 
\[
    y = A \cos \ln x + B \sin \ln x + 2 x
\] 
\subsection*{T.1}
\[
    y^{\prime\prime} + ay = \sin x
\]
Отдельно рассмотрим случай $a = 1$: 
\[
    y^{\prime\prime} + y = \sin x
\]
В таком случае решением будет являтся 
\[
    y = A\sin x + B\cos x - \frac{1}{2} x \sin x
\] 
Такие решения всегда являются неограниченными и апериодическими. 
В другом случае частным решением является $y = \frac{1}{a - 1} \sin x$, 
которое является ограниченным и периодическим.
А таком случае рассмотрим $a < 0$, тогда общее решение будет: 
\[
    y = A e^{-\sqrt{a} x} + B e^{\sqrt{a} x} + \frac{1}{a - 1} \sin x
\]    
В таком случае существует ограниченное решение при $B = 0$, и имеет единственное периодическое решение при $A = B = 0$. 
При $a = 0$ решение будет иметь вид: 
\[
    y = A + Bt + \frac{1}{a - 1} \sin x
\]
Что также являются ограниченным и периодическим при $B = 0, A \in \mathbb{R}$. 
При $a > 0$ решение имеет вид: 
\[
    y = A \sin \sqrt{a} x + B \cos \sqrt{a} x + \frac{1}{a - 1} \sin x 
\]  
Такое решение всегда является ограниченным и периодическим. 
Подведём итоги, существует хотя бы одно ограниченное решение при
\[
    a \neq 1
\] 
Тогда как существует единственное периодическое решение при 
\[
    a \in (-\infty, 0)
\]

\subsection*{614}
Подойдёт уравнение: 
\[
    y^{\prime\prime} - 4y^{\prime}  + 5y = 0
\]
Проверим:
\[
    \begin{dcases}
        y = e^{2x}\cos x \\
        y^{\prime} = 2e^{2x}\cos x - e^{2x} \sin x \\
        y^{\prime\prime} = 3e^{2x}\cos x - 4e^{2x} \sin x  
    \end{dcases}
\]
Легко убедится, что при подстановке в уравнение выполняется торждество.

\subsection*{617}
Тогда общее решение уравнения должно иметь вид: 
\[
    y = A x e^x + B e^x + C e^{-x} 
\]
т.е характеристическое уравнение иммет вид: 
\[
    0 = (x - 1)^2 (x + 1) = x^3 - x^2 - x + 1
\]
Тогда исходное дифференциальные уравнение можно записать как 
\[
    y^{\prime\prime\prime} - y^{\prime\prime} - y^{\prime} + y = 0
\]

\subsection*{618}
Аналогично предыдущей задаче общее решение имеет вид: 
\[
    y = A + Bx + C \sin x
\]
С характеристическим уравнением 
\[
    0 = x^2 (x - i) (x + i) = x^2(x^2 + 1) = x^4 + x^2
\]
В таком случае дифференциальное уравнение имеет вид: 
\[
    y^{\prime\prime\prime\prime} + y^{\prime\prime} = 0
\]

\section{Элементы вариационного исчисления}
\subsection*{19.14}
\[
    J(y) = \int_{1}^{2} \left[ x(y^{\prime})^2 + \frac{y^2}{x} + 4y \right] dx, \, y(1) = 0, \, y(2) = 2\ln 2 
\]
Уравнение Лагранжа: 
\[
    \frac{2y}{x} + 4 = \frac{d}{dx} \left( 2x y^{\prime} \right) = 2y^{\prime} + 2x y^{\prime\prime} \implies x^2 y^{\prime\prime} + xy^{\prime} - y = 2x
\]
Подставим $x = e^t, y^{\prime} = e^{-t} y^{\prime}_t, y^{\prime\prime} = e^{-2t} (y^{\prime\prime}_t - y^{\prime}_t)$. 
\[
    y^{\prime\prime} - y = 2e^{t}
\]
Его общим решением является: 
\[
    y = A e^t + B e^{-t}
\]
Тогда как частное решение ищем в виде $y = at e^t$:
\[
    y^{\prime} = ae^t + at e^t 
\] 
\[
    y^{\prime\prime} = 2ae^t + at e^t
\]
Подставляя в исходное уравнение: 
\[
    2ae^t + at e^t - ate^t = 2ae^t = 2 e^t \implies a = 1
\]
Тогда уравнение имеет решение: 
\[
    y = A e^t + B e^{-t} + t e^t
\]
Переходя к перемнной $x$: 
\[
    y = Ax + \frac{B}{x} + x \ln x 
\] 
Подставляя граничные условия найдём экстремальную функцию: 
\[
    y = x \ln x
\]
Исследуем на максимум-минимум: 
\[
    J(y + h) - J(y) = 
    \int_{1}^{2} x (y^{\prime} + h^{\prime} )^2 + \frac{(y + h)^2}{x} + 4(y + h) - x(y^{\prime})^2 - \frac{y^2}{x} - 4y dx
\]
Первые степени по $h$ сокращаются в силу уравнения Лагранжа: 
\[
    \Delta J(y, h) = \int_{1}^{2} x (h^{\prime})^2 + \frac{h^2}{x} dx \geq 0
\] 
Значит существует минимум на $y = x \ln x$. 
\subsection*{19.39}
\[
    J(y) = \int_{-2}^{-1} [x^3 (y^{\prime})^2 + 3xy^2]dx, \, y(-2) = \frac{15}{2}, \, y(-1) = 0
\]
Уравнение Лагранжа: 
\[
    6xy = \frac{d}{dx} (2x^3 y^{\prime}) = 6 x^2 y^{\prime} + 2x^3 y^{\prime\prime} \implies 
    x^3 y^{\prime\prime} + 3 x^2 y^{\prime} - 3xy = 0 \implies 
    x^2 y^{\prime\prime} + 3x y^{\prime} - 3y = 0
\]
Подставим $x = e^t, y^{\prime} = e^{-t} y^{\prime}_t, y^{\prime\prime} = e^{-2t} (y^{\prime\prime}_t - y^{\prime}_t)$. 
\[
    y^{\prime\prime} + 2y^{\prime} - 3y = 0
\]
Решениями хар. уравнения являются $\lambda = 1, \, \lambda = -3$: 
\[
    y = A e^t + B e^{-3t} \implies 
    y = A x + \frac{B}{x^3}
\] 
Подставляя граничные условия находим: 
\[
    y = \frac{1}{x^3} - x
\]
Проверим на максимум и минимум, сразу сократим члены первого порядка по $h$: 
\[
    \Delta J(x, h) = 
    \int_{-2}^{-1} x^3 (h^{\prime})^2 + 3x h^2 dx = 
    \int_{-2}^{-1} x \left( x^2 (h^{\prime})^2 + 3 h^2 \right) dx \leq 0, \, \text{так как $x < 0$}  
\]
Существует максимум на решении $y = \frac{1}{x^3} - x$. 
\subsection*{19.58}
\[
    J(y) = \int_{1}^{2} \left[ x^2 (y^{\prime})^2 + y y^{\prime} + 12xy \right] dx, \, y(1) = 1, \, y(2) = 5
\]
Уравнение Лагранжа: 
\[
    12x + y^{\prime} = \frac{d}{dt} \left( y + 2 x^2 y^{\prime} \right) = 
    y^{\prime} + 2x^2 y^{\prime\prime} + 4 x y^{\prime} \implies 
    x y^{\prime\prime} + 2 y^{\prime} - 6 = 0 
\]
Введя замену $z = y^{\prime}$ решаем уравнение: 
\[
    xz^{\prime} + 2z - 6 = 0
\] 
Решением является: 
\[
    z = \frac{A}{x^2} + 3
\]
Тогда относительно $y$: 
\[
    y = \frac{A}{x} + B + 3x
\]
Подставляя граничные условия получим: 
\[
    y = 3x - \frac{2}{x}
\]
Проверим на минимум-максимум:
\[
    \Delta J(y, h) = \int_{1}^{2} x^2(h^{\prime})^2 + h h^{\prime} dx = 
    \int_{1}^{2} \left( xh^{\prime} + \frac{h}{2x} \right)^2 - \frac{h^2}{4x^2} dx 
\]  
(((((А как дальше не ясно)))))
\subsection*{19.102}
\[
    J(y) = \int_{0}^{\pi} \left[ (y^{\prime})^2 - \frac{16}{9} y^2 + 2y \sin x \right] dx, \, y(0) = 0, \, y(\pi) = \frac{-\sqrt{3} }{2} 
\]
Уравнение Лагранжа: 
\[
    2\sin x - \frac{32}{9} y = 2y^{\prime\prime} \implies 
    y^{\prime\prime} + \frac{16}{9} y = \sin x
\]
Общее решение имеет вид: 
\[
    y = A \sin \frac{4}{3} x + B \cos \frac{4}{3} x
\]
Частное решение найдём в виде $a \sin x$: 
\[
    -a \sin x + \frac{16}{9} a \sin x = \sin x \implies 
    \frac{7}{9} a = 1 \implies a = \frac{9}{7}
\]
Всё решение выглядит как 
\[
    y = A \sin \frac{4x}{3} + B \cos \frac{4x}{3} + \frac{9}{7} \sin x
\]
Подставим начальные условия: 
\[
    y = \sin \frac{4x}{3}
\]
Такс: 
\[
    \Delta J(y, h) = 
    \int_{0}^{\pi} \left[ (h^{\prime})^2 - \frac{16}{9} h^2 \right] dx
\]
Заменим $x \leftrightarrow \frac{x}{\pi} \implies h^{\prime} \leftrightarrow \frac{h^{\prime}}{\pi}$: 
\[
    \Delta J(y, h) = \frac{1}{\pi} \int_{0}^{1} \left[ (h^{\prime})^2 - \frac{16 \pi^2}{9} h^2  \right]  dx
\] 
Так как $\frac{4\pi}{3} > \pi$, то согласно упражнению из параграфа к задаче нет ни минимума ни максимума. 

\subsection*{20.5}
\[
    J(y) = \int_{1}^{2} [x^2 (y^{\prime})^2 + 12y^2] dx, \, y(1) = 97
\]
Уравнение Лагранжа: 
\[
    24y = \frac{d}{dx} (2x^2 y^{\prime}) = 4x y^{\prime} + 2x^2 y^{\prime\prime} \implies 
    x^2 y^{\prime\prime} + 2xy^{\prime} - 12 y = 0
\]
Подставим $x = e^t, y^{\prime} = e^{-t} y^{\prime}_t, y^{\prime\prime} = e^{-2t} (y^{\prime\prime}_t - y^{\prime}_t)$. 
\[
    y^{\prime\prime} + y^{\prime} - 12y = 0
\]
Решением явялется: 
\[
    y = A e^{3t} + B e^{-4t} = Ax^3 + \frac{B}{x^4}
\]
Из первого граничного условия находим $A + B = 97$, второе граничное условие: 
\[
    0 = 2 \cdot 2^2 y^{\prime}(2) \implies y^{\prime}(2) = 0 \implies 
    3 A \cdot 4 = \frac{4B}{32} \implies B = 96 A
\]
Тогда решением является: 
\[
    y = x^3 + \frac{96}{x^4}
\]
Проверим на мин-макс: 
\[
    \Delta J(y, h) = \int_{1}^2 \left[ x^2 (h^{\prime})^2 + 12h^2 \right] \geq 0 - \text{минимум} 
\]

\subsection*{5.9}
\[
    J(y) = \int_{1}^{3} [8y y^{\prime} \ln x - x(y^{\prime} )^2 + 6xy^{\prime}]dx, \, y(3) = 15
\]
Уравнение Лагранжа: 
\[
    8y^{\prime} \ln x = \frac{d}{dx} (8y\ln x - 2xy^{\prime} + 6x) = 
    8y^{\prime} \ln x + \frac{8y}{x} - 2y^{\prime} - 2xy^{\prime\prime} + 6
\]
Сокращаем слагаемые и получаем: 
\[
    xy^{\prime\prime} + y^{\prime} - \frac{4y}{x} = 3 \implies 
    x^2 y^{\prime\prime} + xy^{\prime} - 4y = 3x  
\]
Подставим $x = e^t, y^{\prime} = e^{-t} y^{\prime}_t, y^{\prime\prime} = e^{-2t} (y^{\prime\prime}_t - y^{\prime}_t)$.
\[
    y^{\prime\prime} - 4y = 3e^t
\]
Решением является: 
\[
    y = Ae^{2t} + Be^{-2t} - e^t = A x^2 + \frac{B}{x^2} - x
\]
Условием на конец является: 
\[
    - 2 y^{\prime}(1) + 6 = 0 \implies y^{\prime} = 3 \implies 
    2A - 2B - 1 = 3 \implies A + B = 2
\]
Из условия на другой конец находим $A = 2, B = 0$:
\[
    y = 2x^2 - x
\] 
Тогда 
\[
    \Delta J(y, h) = \int_{1}^{3} [8 h h^{\prime} \ln x - x (h^{\prime})^2] dx < 
    \int_{1}^{3} [8 (x - 1) h h^{\prime} - x (h^{\prime})^2] dx = 
    \int_{1}^{3} [x (8h h^{\prime} - (h^{\prime})^2) - 8h h^{\prime}]dx < 0
\]
(((((А я не знаю что делать....)))))
\subsection*{20.14}
\[
    J(y) = \int_{0}^{1} [(y^{\prime})^2 + y^2 - 4e^x y - 8x e^x y^{\prime}] dx 
\]
Уравнение Лагранжа: 
\[
    2y - 4e^x = 2y^{\prime\prime} - 8 e^x - 8x e^x \implies 
    y^{\prime\prime} - y = 2e^x + 4x e^x
\]
В задаче оба конца свободные: 
\[
    \begin{dcases}
        y^{\prime}(0) = 0 \\
        y^{\prime}(1) = 4e
    \end{dcases}
\] 
Общим решением является: 
\[
    y = A \cosh x + B \sinh x + x^2 e^x
\]
Производная: 
\[
    y^{\prime} = A \sinh x + B \cosh x + 2x e^x + x^2 e^x \implies B = 0, \, A = \frac{e}{\sinh e}
\]

(((((Ну чёт как-то не пошло)))))

\end{document}