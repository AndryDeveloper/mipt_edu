\documentclass[12pt]{article}
\usepackage[T2A]{fontenc}
\usepackage[utf8]{inputenc}
\usepackage{multirow}
\usepackage{caption}
\usepackage{subcaption}
\usepackage{amsmath}
\usepackage{amssymb}
\usepackage{changepage}
\usepackage{graphicx}
\usepackage{float}
\usepackage[english,russian]{babel}
\usepackage{amsmath, amsfonts, amssymb, amsthm, mathtools}
\usepackage{xcolor}
\usepackage{array}
\usepackage{hyperref}
\usepackage{physics}
\usepackage[top = 1.5cm, left = 1.5 cm, right = 1.5 cm, bottom = 3 cm]{geometry}
\usepackage{import}
\usepackage{xifthen}
\usepackage{pdfpages}
\usepackage{transparent}

\newcommand{\incfig}[1]{
    \import{./figures/}{#1.pdf_tex}
}

\title{Работа над ошибками.}
\author{Шахматов Андрей, Б02-304}
\date{\today}

\begin{document}
\maketitle
% \tableofcontents

\section*{16.8(6)}
\[
    I = \int_{0}^{1} \frac{\sqrt{x(1-x)} }{(1 + x)^3} \,\mathrm{d}x 
\]
Сделаем замену $t = \frac{1 - x}{1 + x}, \, \mathrm{d}t = \frac{-2\mathrm{d} x}{(1 + x)^2}$: 
\[
    I = \frac{1}{2} \int_{0}^{1} \frac{1 + t}{2} \cdot \sqrt{\frac{1 - t}{1 + t} \cdot \frac{2t}{1 + t}} \,\mathrm{d}t = 
    \frac{\sqrt{2}}{4} \int_{0}^{1} t^{\frac{1}{2}} (1 - t)^{\frac{1}{2}} \,\mathrm{d}t = 
    \frac{\sqrt{2}}{4} B\left( \frac{3}{2}, \frac{3}{2} \right) = \frac{\pi \sqrt{2} }{32} 
\] 

\section*{Т3}
\[
    \begin{dcases}
        x = \sinh \theta \cos \varphi \\
        y = \sinh \theta \sin \varphi \\
        z = \cosh \theta
    \end{dcases}
\]
Тогда 
\[
    \begin{dcases}
        dx = \cosh \theta \cos \varphi d \theta - \sinh \theta \sin \varphi d \varphi \\
        dy = \cosh \theta \sin \varphi d \theta + \sinh \theta \cos \varphi d \varphi \\
        dz = \sinh \theta d \theta  
    \end{dcases}
\]

Тогда прямой образ формы $\omega = x dy \wedge dz + y dz \wedge dx + zdx \wedge dy$:
\begin{align*}
    \phi^{\ast}\omega = \sinh \theta \cos \varphi \cdot (\cosh \theta \sin \varphi d \theta + \sinh \theta \cos \varphi d \varphi) \wedge (\sinh \theta d \theta ) + \\
    \sinh \theta \sin \varphi \cdot (\sinh \theta d \theta) \wedge (\cosh \theta \cos \varphi d \theta - \sinh \theta \sin \varphi d \varphi ) + \\
    \cosh \theta \cdot (\cosh \theta \cos \varphi d \theta - \sinh \theta \sin \varphi d \varphi) \wedge (\cosh \theta \sin \varphi d \theta + \sinh \theta \cos \varphi d \varphi ) = \\ 
    \sinh^3 \theta \cos^2 \varphi d\varphi \wedge d \theta - \sinh^3 \theta \sin^2 \varphi d \theta \wedge d \varphi + \\
    \cosh^2 \theta \sinh \theta \cos^2 \varphi d \theta \wedge d \varphi - \cosh^2 \theta \sinh \theta \sin^2 \varphi d \varphi \wedge d \theta = \\
    \sinh^3 \theta d \varphi \wedge d \theta - \cosh^2 \theta \sinh \theta d \varphi \wedge d \theta = \\
    -\sinh \theta d\varphi \wedge d \theta 
\end{align*}
\section*{4.52(5)}
\[
    2(x + y) z_{yy} + z_x = 0
\]
\[
    u = x, \, v = \sqrt{x + y} 
\]
Найдём матрицу Якоби: 
\[
    \begin{pmatrix}
        1 & 0 \\
        \frac{1}{2v} & \frac{1}{2v}
    \end{pmatrix}
\]

Тогда 
\[
    z_x = z_u u_x + z_v v_x = 
    z_u + \frac{z_{v}}{2v}
\] 
\[
    z_y = z_u u_y + z_v v_y = 
    \frac{z_v}{2v}   
\] 
Для второй производной: 
\[
    z_{yy} = z_{yu} u_y + z_{yv} v_y = 
    \frac{ \frac{z_{vv}}{2v} - \frac{z_v}{2v^2}}{2v} = 
    \frac{z_{vv}}{4v^2} - \frac{z_v}{4v^3}
\]
Тогда перепишем исходное уравнение:
\[
    2v^{2} \left( \frac{z_{vv}}{4v^2} - \frac{z_v}{4v^3} \right) + z_u + \frac{z_{v}}{2v} = 
    \frac{z_{vv}}{2} + z_u = 0
\]
\section*{3.92(2)}
Выразим $w_v$ в исходных координатах: 
\[
    w_v = (xy)_v - z_v 
\]

Матрица перехода 
\[
    \begin{pmatrix}
        -1 & z & y \\
        z & -1 & x \\
        y & x & -1
    \end{pmatrix}
\]
Обратная к ней: 
\[
    C
    \begin{pmatrix}
        1 - x^2 & xy + z & xz + y \\
        xy + z & 1 - y^2 & x + yz \\
        xz + y & x + yz & 1 - z^2
    \end{pmatrix}
\]
Тогда: 
\[
    z_v = z_x x_v + z_y y_v = C (z_x (xy + z) + z_y (1 - y^2)) 
\]
\[
    (xy)_v = x y_v + y x_v = C(x(1 - y^2) + y(xy + z)) = 
    C(x + yz)
\]
Тогда 
\[
    w_v = C(x + yz) - C(z_x (xy + z) + z_y (1 - y^2)) = 
    C(x + yz - z_x (xy + z) + z_y (1 - y^2)) = 0
\]
Тогда интегрируя $w_v = 0$ получим: 
\[
    w = f(u)
\] 
Или в исходных координатах 
\[
    xy - z = f(yz - x)
\]
*Где $f$ - произвольная функция. 

\end{document}