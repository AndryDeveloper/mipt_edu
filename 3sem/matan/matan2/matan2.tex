\documentclass[12pt]{article}
\usepackage[T2A]{fontenc}
\usepackage[utf8]{inputenc}
\usepackage{multirow}
\usepackage{caption}
\usepackage{subcaption}
\usepackage{amsmath}
\usepackage{amssymb}
\usepackage{changepage}
\usepackage{graphicx}
\usepackage{float}
\usepackage[english,russian]{babel}
\usepackage{amsmath, amsfonts, amssymb, amsthm, mathtools}
\usepackage{xcolor}
\usepackage{array}
\usepackage{hyperref}
\usepackage{physics}
\usepackage[top = 1.5cm, left = 1.5 cm, right = 1.5 cm, bottom = 3 cm]{geometry}
\usepackage{import}
\usepackage{xifthen}
\usepackage{pdfpages}
\usepackage{transparent}

\newcommand{\incfig}[1]{
    \import{./figures/}{#1.pdf_tex}
}

\title{Матан вторая домашка.}
\author{Шахматов Андрей, Б02-304}
\date{\today}

\begin{document}
\maketitle
\tableofcontents

\section{T1}
Множество $X$ задаётся следующим неравенством: 
\[
    (x^2 + y^2 + z^2)^2 \leq az(x^2 + y^2)
\]
Тогда 
\[
    \mu X = \int_{X} 1 \,dx dy dz
\]
Введём сферическую замену координат, его якобиан $\vert J \vert = r^2 \sin \theta$, и неравенство преобразуется как
\[
    r^4 \leq az r^2 \sin^2 \theta \implies r^2 \leq \sin^2 \theta 
\]
Тогда объём равен: 
\[
    \mu X = \int_{r^2 \leq \sin^2 \theta} r^2 \sin \theta \,dr d \theta d \phi = 2\pi \int_{0}^{\pi} \sin \theta d \theta \int_{0}^{\vert \sin \theta \vert} r^2 dr = 
    \frac{2\pi}{3} \int_0^{\pi} \sin^4 \theta d \theta = \frac{2\pi}{3} \frac{3\pi}{8} = \frac{\pi^2}{4}
\]
\section{T2}
б)
\[
    \int_{\vert \frac{y}{b} \vert \leq \frac{x}{a}} e^{-\frac{x^2 + y^2}{2c^2}} dx dy
\]
Потом сделаю((()))


в)
\[
    \int_{a^2 \leq x^2 + y^2 + z^2 \leq b^2} x^2 + y^2 - z^2 dx dy dz
\]
Перейдём в сферическую систему координат: 
\[
    \int_{a^2 \leq r^2 \leq b^2} r^2(\sin^2 \theta - \cos^2 \theta) r^2 \sin \theta dr d \theta  d \phi = 
    2\pi \int_a^b r^4 dr \int_0^{\pi} (\sin^3 \theta - \sin \theta \cos^2 \theta ) d \theta = 
    2\pi \frac{2}{3} \left( \frac{b^5 - a^5}{5} \right)
\]
Тогда ответ: 
\[
    I = \frac{4\pi}{15} (b^5 - a^5)
\]

д)
\[
    \int_{x^2 + y^2 \leq az \leq b^2} z^2 dx dy dz
\] 

е) 
\[
    \int_{\frac{x^2}{a^2} + \frac{y^2}{b^2} + \frac{z^2}{c^2} \leq 1} x^2 + y^2 + z^2 dx dy dz = 
    \frac{1}{abc} \int_{x^2 + y^2 + z^2 \leq 1} (ax)^2 + (by)^2 + (cz)^2 dx dy dz
\]
Перейдём к сферическим координатам: 
\[
    I = \frac{1}{abc} \int_{r^2 \leq 1} r^2 dr \int_{0}^{2\pi} \int_{0}^{\pi} (a\sin \theta \cos \phi)^2 + (b\sin \theta \sin \phi)^2 + (c\cos \theta)^2 d \theta d \phi
\]
\[
    I = \frac{2}{3abc} \left( a^2\frac{\pi^2}{2} + b^2\frac{\pi^2}{2} + c^2\frac{\pi^2}{2} \right) = 
    \frac{\pi^2 (a^2 + b^2 + c^2)}{3abc}
\]

\section{T3}
б)

\section{T4}
\[
    \int_{\mathbb{R}^2} e^{-a(x^2 + y^2)} \sin (x^2 + y^2) dx dy
\]
Интеграл существует при любых значениях параметра a т.к. он мажорируется сходящимся. Введём полряные координаты. 
\[
    I = 2 \pi \int_{0}^{+\infty} e^{-ar^2} \sin (r^2) r dr = \pi \int_{0}^{+\infty} e^{-at} \sin t dt = \frac{\pi}{a^2 + 1}
\]



\end{document}