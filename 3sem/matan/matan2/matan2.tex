\documentclass[12pt]{article}
\usepackage[T2A]{fontenc}
\usepackage[utf8]{inputenc}
\usepackage{multirow}
\usepackage{caption}
\usepackage{subcaption}
\usepackage{amsmath}
\usepackage{amssymb}
\usepackage{changepage}
\usepackage{graphicx}
\usepackage{float}
\usepackage[english,russian]{babel}
\usepackage{amsmath, amsfonts, amssymb, amsthm, mathtools}
\usepackage{xcolor}
\usepackage{array}
\usepackage{hyperref}
\usepackage{physics}
\usepackage[top = 1.5cm, left = 1.5 cm, right = 1.5 cm, bottom = 3 cm]{geometry}
\usepackage{import}
\usepackage{xifthen}
\usepackage{pdfpages}
\usepackage{transparent}

\newcommand{\incfig}[1]{
    \import{./figures/}{#1.pdf_tex}
}

\title{Матан вторая домашка.}
\author{Шахматов Андрей, Б02-304}
\date{\today}

\begin{document}
\maketitle
\tableofcontents

\section{T1}
Множество $X$ задаётся следующим неравенством: 
\[
    (x^2 + y^2 + z^2)^2 \leq az(x^2 + y^2)
\]
Тогда 
\[
    \mu X = \int_{X} 1 \,dx dy dz
\]
Введём сферическую замену координат, его якобиан $\vert J \vert = r^2 \sin \theta$, и неравенство преобразуется как
\[
    r^4 \leq az r^2 \sin^2 \theta \implies r^2 \leq \sin^2 \theta 
\]
Тогда объём равен: 
\[
    \mu X = \int_{r^2 \leq \sin^2 \theta} r^2 \sin \theta \,dr d \theta d \phi = 2\pi \int_{0}^{\pi} \sin \theta d \theta \int_{0}^{\vert \sin \theta \vert} r^2 dr = 
    \frac{2\pi}{3} \int_0^{\pi} \sin^4 \theta d \theta = \frac{2\pi}{3} \frac{3\pi}{8} = \frac{\pi^2}{4}
\]
\section{T2}
б)
\[
    \int_{\vert \frac{y}{b} \vert \leq \frac{x}{a}} e^{-\frac{x^2 + y^2}{2c^2}} dx dy
\]
Домножим существующие координаты на $x = ax, y = by$
\[
    ab \int_{-x \leq y \leq x} \exp {-\frac{1}{2}\left( \frac{ax}{c} \right)^2 - \frac{1}{2} \left( \frac{by}{c} \right)^2} dx dy = 
    ab \int_0^{\infty} dx \exp {-\frac{1}{2} \left( \frac{ax}{c} \right)^2 } \int_{-x}^{x} dy \exp {-\frac{1}{2} \left( \frac{by}{c} \right)^2 }
\] 
((((потом сделаю))))

в)
\[
    \int_{a^2 \leq x^2 + y^2 + z^2 \leq b^2} x^2 + y^2 - z^2 dx dy dz
\]
Перейдём в сферическую систему координат: 
\[
    \int_{a^2 \leq r^2 \leq b^2} r^2(\sin^2 \theta - \cos^2 \theta) r^2 \sin \theta dr d \theta  d \phi = 
    2\pi \int_a^b r^4 dr \int_0^{\pi} (\sin^3 \theta - \sin \theta \cos^2 \theta ) d \theta = 
    2\pi \frac{2}{3} \left( \frac{b^5 - a^5}{5} \right)
\]
Тогда ответ: 
\[
    I = \frac{4\pi}{15} (b^5 - a^5)
\]

д)
\[
    I = \int_{x^2 + y^2 \leq az \leq b^2} z^2 dx dy dz
\] 
Перейдём в циллиндрическую систему координат: 
\[
    I = \int_{r^2 \leq az \leq b^2} z^2 r dr d \phi dz = 
    2\pi \int_0^{b^2} z^2 dz \int_0^{\sqrt{az}} r dr = 
    a\pi \int_0^{b^2} z^3 dz = \frac{\pi}{4} ab^8
\]

е) 
\[
    \int_{\frac{x^2}{a^2} + \frac{y^2}{b^2} + \frac{z^2}{c^2} \leq 1} x^2 + y^2 + z^2 dx dy dz = 
    \frac{1}{abc} \int_{x^2 + y^2 + z^2 \leq 1} (ax)^2 + (by)^2 + (cz)^2 dx dy dz
\]
Перейдём к сферическим координатам: 
\[
    I = \frac{1}{abc} \int_{r^2 \leq 1} r^2 dr \int_{0}^{2\pi} \int_{0}^{\pi} (a\sin \theta \cos \phi)^2 + (b\sin \theta \sin \phi)^2 + (c\cos \theta)^2 d \theta d \phi
\]
\[
    I = \frac{2}{3abc} \left( a^2\frac{\pi^2}{2} + b^2\frac{\pi^2}{2} + c^2\frac{\pi^2}{2} \right) = 
    \frac{\pi^2 (a^2 + b^2 + c^2)}{3abc}
\]

\section{T3}
б)
Плотность тела примем равной $1$.
Введём циллиндрическую систему координат, $x = ar\cos \phi$, $y = ar \sin \phi$, $h = z c$. 
\[
    M = a^2c \int_{r \leq h \leq 1} r dr d\phi dh = 
    2\pi a^2c \int_0^1 dh \int_0^h dr = \pi a^2c 
\]
Тогда так как тело является телом вращения, то $x_0 = y_0 = 0$. Найдём $z_0$
\[
    z_0 = 2\pi a^2c \int_{r \leq h \leq 1} ch r dr d\phi dh = 
    2\pi a^2c^2 \int_0^1 h dh \cdot h = a^2 c^2 \frac{2\pi}{3} = \frac{2M c}{3}
\]     

\section{T4}
\[
    \int_{\mathbb{R}^2} e^{-a(x^2 + y^2)} \sin (x^2 + y^2) dx dy
\]
Интеграл существует при любых значениях параметра a т.к. он мажорируется сходящимся. Введём полряные координаты. 
\[
    I = 2 \pi \int_{0}^{+\infty} e^{-ar^2} \sin (r^2) r dr = \pi \int_{0}^{+\infty} e^{-at} \sin t dt = \frac{\pi}{a^2 + 1}
\]
Последний интеграл мы находили в прошлом году беря его два раза по частям.
\section{8.201}
\[
    f(x, y) = 
    \begin{dcases}
        \frac{x^2 - y^2}{(x^2 + y^2)^2}, \, x \geq 1, y \geq 1 \\
        0, \text{иначе}.
    \end{dcases}
\]
Введём сферическую замену координат, рассмотрим пока что область $x\geq 1, y\geq 1$ 
\[
    f(x, y) = \frac{1}{r^2} \sin^2 \theta \cos 2\phi,
\]
Интеграл по ограниченной области $A$ соответственно равен
\[
    F = \int_A \frac{1}{r^2} \sin^2 \theta \cos 2\phi r^2 \sin \theta  \, dr d \theta d \phi = 
    \int_A \sin^3 \theta \cos 2 \phi \, dr d \theta d \phi 
\]
Тогда в дальнейшем при $r \geq \sqrt{2} $  в интеграле возникнет $\int_{0}^{2\pi} \cos 2 \phi d \phi = 0$. 
Тогда интеграл при $r \geq \sqrt{2}$ равен 0, а при $r \leq \sqrt{2}$ очевидно сходится так как 
ограничен на множестве конечной меры.    

\section{Т6}
в) 
Я не уверен, но стереографической проекцией из точки шара $(0, 0, -1)$ получим, что полусфера 
диффеоморфна диску на плоскости (а для него мы доказывали в предыдущем пункте). Замена координат следующая: 
\[
    \begin{dcases}
        t_A^1 = \frac{x}{1 + z} \\
        t_A^2 = \frac{y}{1 + z}
    \end{dcases}
\]

\section{Т7}
\[
    Q(x, y) = Ax^2 + 2Bxy + Cy^2 + Dx + Ey + F
\]
Не знаю((()))

\section{Т8}
Введём 2 карты, соответствующие двум стереографическим проекциям сферы на две плоскости. 
$\phi_A: V_A \to U_A$, $V_A = \mathbb{R}^2$, $U_A = \mathbb{S}^2 \setminus (0, 0, 1)$   
\[
    \begin{dcases}
        t_A^1 = \frac{x}{1 - z} \\
        t_A^2 = \frac{y}{1 - z}
    \end{dcases}
\]
Обратное имеет вид 
\[
    \begin{dcases}
        x = \frac{2t_A^1}{1 + (t_A^1)^2 + (t_A^2)^2} \\ 
        y = \frac{2t_A^2}{1 + (t_A^1)^2 + (t_A^2)^2} \\ 
        z = \frac{(t_A^1)^2 + (t_A^2)^2 - 1}{1 + (t_A^1)^2 + (t_A^2)^2}
    \end{dcases}
\]
Найдём ранг карты: 
\[
    \begin{pmatrix}
        \frac{1}{1 - z} & 0 & \frac{x}{(1-z)^2} \\ 
        0 & \frac{1}{1 - z} & \frac{y}{(1-z)^2} \\ 
    \end{pmatrix}
\]
На $V_A$ ранг карты очевидно равен двум.
Аналогично строится вторая карта $\phi_B: V_B \to U_B$, разве, что $1 - z \leftrightarrow 1 + z$. То есть: 
\[
    \begin{dcases}
        t_B^1 = \frac{x}{1 + z} \\
        t_B^2 = \frac{y}{1 + z}
    \end{dcases}
\]
Все остальные формальные утверждения делаются аналогично, из чего следует, что сфера 
является вложенным многообразием. 
Посмотрим на гладкость функции свзяки: 
\[
    \begin{dcases}
        t_A^1 = \frac{1}{t_B^1} \\
        t_A^2 = \frac{1}{t_B^2}
    \end{dcases}
\]
Такие функции очевидно бесконечно гладкие, а значит такой набор карт будет являтся 
атласом для сферы ранга 2.


\end{document}