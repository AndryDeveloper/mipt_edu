\documentclass[12pt]{article}
\usepackage[T2A]{fontenc}
\usepackage[utf8]{inputenc}
\usepackage{multirow}
\usepackage{caption}
\usepackage{subcaption}
\usepackage{amsmath}
\usepackage{amssymb}
\usepackage{changepage}
\usepackage{graphicx}
\usepackage{float}
\usepackage[english,russian]{babel}
\usepackage{amsmath, amsfonts, amssymb, amsthm, mathtools}
\usepackage{xcolor}
\usepackage{array}
\usepackage{hyperref}
\usepackage{physics}
\usepackage[top = 1.5cm, left = 1.5 cm, right = 1.5 cm, bottom = 3 cm]{geometry}
\usepackage{import}
\usepackage{xifthen}
\usepackage{pdfpages}
\usepackage{transparent}

\newcommand{\incfig}[1]{
    \import{./figures/}{#1.pdf_tex}
}

\title{Матан вторая домашка.}
\author{Шахматов Андрей, Б02-304}
\date{\today}

\begin{document}
\maketitle
\tableofcontents

\section{T1}
Множество $X$ задаётся следующим неравенством: 
\[
    (x^2 + y^2 + z^2)^2 \leq az(x^2 + y^2)
\]
Тогда 
\[
    \mu X = \int_{X} 1 \,dx dy dz
\]
Введём сферическую замену координат, его якобиан $\vert J \vert = r^2 \sin \theta$, и неравенство преобразуется как
\[
    r^4 \leq az r^2 \sin^2 \theta \implies r^2 \leq \sin^2 \theta 
\]
Тогда объём равен: 
\[
    \mu X = \int_{r^2 \leq \sin^2 \theta} r^2 \sin \theta \,dr d \theta d \phi = 2\pi \int_{0}^{\pi} \sin \theta d \theta \int_{0}^{\vert \sin \theta \vert} r^2 dr = 
    \frac{2\pi}{3} \int_0^{\pi} \sin^4 \theta d \theta = \frac{2\pi}{3} \frac{3\pi}{8} = \frac{\pi^2}{4}
\]
\section{T2}
б)
\[
    \int_{\vert \frac{y}{b} \vert \leq \frac{x}{a}} e^{-\frac{x^2 + y^2}{2c^2}} dx dy
\]
Домножим существующие координаты на $x = ax, y = by$
\[
    ab \int_{-x \leq y \leq x} \exp {-\frac{1}{2}\left( \frac{ax}{c} \right)^2 - \frac{1}{2} \left( \frac{by}{c} \right)^2} dx dy = 
    ab \int_0^{\infty} dx \exp {-\frac{1}{2} \left( \frac{ax}{c} \right)^2 } \int_{-x}^{x} dy \exp {-\frac{1}{2} \left( \frac{by}{c} \right)^2 }
\] 
((((потом сделаю))))

в)
\[
    \int_{a^2 \leq x^2 + y^2 + z^2 \leq b^2} x^2 + y^2 - z^2 dx dy dz
\]
Перейдём в сферическую систему координат: 
\[
    \int_{a^2 \leq r^2 \leq b^2} r^2(\sin^2 \theta - \cos^2 \theta) r^2 \sin \theta dr d \theta  d \phi = 
    2\pi \int_a^b r^4 dr \int_0^{\pi} (\sin^3 \theta - \sin \theta \cos^2 \theta ) d \theta = 
    2\pi \frac{2}{3} \left( \frac{b^5 - a^5}{5} \right)
\]
Тогда ответ: 
\[
    I = \frac{4\pi}{15} (b^5 - a^5)
\]

д)
\[
    I = \int_{x^2 + y^2 \leq az \leq b^2} z^2 dx dy dz
\] 
Перейдём в циллиндрическую систему координат: 
\[
    I = \int_{r^2 \leq az \leq b^2} z^2 r dr d \phi dz = 
    2\pi \int_0^{b^2} z^2 dz \int_0^{\sqrt{az}} r dr = 
    a\pi \int_0^{b^2} z^3 dz = \frac{\pi}{4} ab^8
\]

е) 
\[
    \int_{\frac{x^2}{a^2} + \frac{y^2}{b^2} + \frac{z^2}{c^2} \leq 1} x^2 + y^2 + z^2 dx dy dz = 
    \frac{1}{abc} \int_{x^2 + y^2 + z^2 \leq 1} (ax)^2 + (by)^2 + (cz)^2 dx dy dz
\]
Перейдём к сферическим координатам: 
\[
    I = \frac{1}{abc} \int_{r^2 \leq 1} r^2 dr \int_{0}^{2\pi} \int_{0}^{\pi} (a\sin \theta \cos \phi)^2 + (b\sin \theta \sin \phi)^2 + (c\cos \theta)^2 d \theta d \phi
\]
\[
    I = \frac{2}{3abc} \left( a^2\frac{\pi^2}{2} + b^2\frac{\pi^2}{2} + c^2\frac{\pi^2}{2} \right) = 
    \frac{\pi^2 (a^2 + b^2 + c^2)}{3abc}
\]

\section{T3}
б)
Плотность тела примем равной $1$.
Введём циллиндрическую систему координат, $x = ar\cos \phi$, $y = ar \sin \phi$, $h = z c$. 
\[
    M = a^2c \int_{r \leq h \leq 1} r dr d\phi dh = 
    2\pi a^2c \int_0^1 dh \int_0^h dr = \pi a^2c 
\]
Тогда так как тело является телом вращения, то $x_0 = y_0 = 0$. Найдём $z_0$
\[
    z_0 = 2\pi a^2c \int_{r \leq h \leq 1} ch r dr d\phi dh = 
    2\pi a^2c^2 \int_0^1 h dh \cdot h = a^2 c^2 \frac{2\pi}{3} = \frac{2M c}{3}
\]     

\section{T4}
\[
    \int_{\mathbb{R}^2} e^{-a(x^2 + y^2)} \sin (x^2 + y^2) dx dy
\]
Интеграл существует при любых значениях параметра a т.к. он мажорируется сходящимся. Введём полряные координаты. 
\[
    I = 2 \pi \int_{0}^{+\infty} e^{-ar^2} \sin (r^2) r dr = \pi \int_{0}^{+\infty} e^{-at} \sin t dt = \frac{\pi}{a^2 + 1}
\]
Последний интеграл мы находили в прошлом году беря его два раза по частям.
\section{8.201}
\[
    f(x, y) = 
    \begin{dcases}
        \frac{x^2 - y^2}{(x^2 + y^2)^2}, \, x \geq 1, y \geq 1 \\
        0, \text{иначе}.
    \end{dcases}
\]
Введём сферическую замену координат, рассмотрим пока что область $x\geq 1, y\geq 1$ 
\[
    f(x, y) = \frac{1}{r^2} \sin^2 \theta \cos 2\phi,
\]
Интеграл по ограниченной области $A$ соответственно равен
\[
    F = \int_A \frac{1}{r^2} \sin^2 \theta \cos 2\phi r^2 \sin \theta  \, dr d \theta d \phi = 
    \int_A \sin^3 \theta \cos 2 \phi \, dr d \theta d \phi 
\]
Тогда в дальнейшем при $r \geq \sqrt{2} $  в интеграле возникнет $\int_{0}^{2\pi} \cos 2 \phi d \phi = 0$. 
Тогда интеграл при $r \geq \sqrt{2}$ равен 0, а при $r \leq \sqrt{2}$ очевидно сходится так как 
ограничен на множестве конечной меры.    

\section{Т6}
в) 
Я не уверен, но стереографической проекцией из точки шара $(0, 0, -1)$ получим, что полусфера 
диффеоморфна диску на плоскости (а для него мы доказывали в предыдущем пункте). Замена координат следующая: 
\[
    \begin{dcases}
        t_A^1 = \frac{x}{1 + z} \\
        t_A^2 = \frac{y}{1 + z}
    \end{dcases}
\]

\section{Т7}
\[
    Q(x, y) = Ax^2 + 2Bxy + Cy^2 + Dx + Ey + F
\]
Судя по всему единственным случаем, когда нули не являются многообразием это случай пересекающихся прямых, 
так как выбросив точку пересечения получим 4 компоненты связности, чего невозможно получить на прямой выбросом одной точки.

\section{Т8}
Введём 2 карты, соответствующие двум стереографическим проекциям сферы на две плоскости. 
$\phi_A: V_A \to U_A$, $V_A = \mathbb{R}^2$, $U_A = \mathbb{S}^2 \setminus (0, 0, 1)$   
\[
    \begin{dcases}
        t_A^1 = \frac{x}{1 - z} \\
        t_A^2 = \frac{y}{1 - z}
    \end{dcases}
\]
Обратное имеет вид 
\[
    \begin{dcases}
        x = \frac{2t_A^1}{1 + (t_A^1)^2 + (t_A^2)^2} \\ 
        y = \frac{2t_A^2}{1 + (t_A^1)^2 + (t_A^2)^2} \\ 
        z = \frac{(t_A^1)^2 + (t_A^2)^2 - 1}{1 + (t_A^1)^2 + (t_A^2)^2}
    \end{dcases}
\]
Найдём ранг карты: 
\[
    \begin{pmatrix}
        \frac{1}{1 - z} & 0 & \frac{x}{(1-z)^2} \\ 
        0 & \frac{1}{1 - z} & \frac{y}{(1-z)^2} \\ 
    \end{pmatrix}
\]
На $V_A$ ранг карты очевидно равен двум.
Аналогично строится вторая карта $\phi_B: V_B \to U_B$, разве, что $1 - z \leftrightarrow 1 + z$. То есть: 
\[
    \begin{dcases}
        t_B^1 = \frac{x}{1 + z} \\
        t_B^2 = \frac{y}{1 + z}
    \end{dcases}
\]
Все остальные формальные утверждения делаются аналогично, из чего следует, что сфера 
является вложенным многообразием. 
Посмотрим на гладкость функции свзяки: 
\[
    \begin{dcases}
        t_A^1 = \frac{1}{t_B^1} \\
        t_A^2 = \frac{1}{t_B^2}
    \end{dcases}
\]
Такие функции очевидно бесконечно гладкие, а значит такой набор карт будет являтся 
атласом для сферы ранга 2.

\section{Т11}
б) 
$\gamma: \frac{x^2}{a^2} + \frac{y^2}{b^2} = 1$. 
\[
    \int_{\gamma} (2xy - y) dx + x^2 dy = \int_{\frac{x^2}{a^2} + \frac{y^2}{b^2} = 1} \left( 2x - (2x - 1) \right) dx \wedge dy = 
    \int_{\frac{x^2}{a^2} + \frac{y^2}{b^2} = 1} dx \wedge dy = \mu \left\{ \frac{x^2}{a^2} + \frac{y^2}{b^2} \right\} = \pi a b
\]


\section{Т12}
a) 
По формуле Остроградского-Гаусса
\[
    \iint_S x^2 y^2 z dy \wedge dz = \int_{x^2 + y^2 + z^2 = a^2, z \geq 0} 2xzy^2 dx \wedge dy \wedge dz
\]
В сферических координатах
\[
    2 \int_{-a \leq r \leq a, z \geq 0} r^6 \cos \theta \sin^4 \theta \cos \varphi \sin^2 \varphi dr d \theta d \varphi = 
    4 \frac{a^7}{7} \int_{0}^{\frac{\pi}{2}} \sin^4 \theta \cos \theta dr d \theta \int_{0}^{2\pi} \sin^2 \varphi \cos \varphi d \varphi = 0
\]
((((Странный ответ))))\\
в)
Перейдём к интегралу по циллиндру по формуле Остроградского 
\[
    \int_{x^2 + y^2 = a^2, 0 \leq z \leq b, x \geq 0, y \geq 0} (x + y + z) dx \wedge dy \wedge dz = 
    2 \int_{x^2 + y^2 = a^2, 0 \leq z \leq b, x \geq 0, y \geq 0} x dx dy dz + \int_{x^2 + y^2 = a^2, 0 \leq z \leq b, x \geq 0, y \geq 0} z dx dy dz
\]
В цилиндрических координатах первый интеграл равен
\[
    \int_{0 < r \leq a, 0 < z < b, x \geq 0, y \geq 0} r^2 \cos \theta dr d \theta dz = 
    b \frac{a^3}{3}
\]
Второй интеграл равен $I_2 = \frac{b^2}{2} \cdot \frac{\pi a^2}{4}$ 
Тогда ответ: 
\[
    I = \frac{a^3 b}{3} + \pi \frac{a^2 b^2}{8}
\]

% г)
% По формуле Остроградского 
% \[
%     \int_{x^2 + y^2 = z^2, 0 \leq z \leq a} (x + y + z) dx \wedge dy \wedge dz = 
% \]
% Перейдём в циллиндрические координаты: 
% \[
%     \int_{r < z < a} r^2 (\sin \theta + \cos \theta) + zr dr d \theta dz = 
%     2\pi \int_{0}^{a} z dz \int_{0}^{z} r dr = \frac{\pi a^3}{3}
% \]


д)
\[
    \int_{x^2 + y^2 + z^2 = a^2, x,y,z \geq 0} (x + y + z) dx \wedge dy \wedge dz = 
    3 \int_{x^2 + y^2 + z^2 = a^2, x,y,z \geq 0} x dx dy dz = 3I
\]
Перейдём в сферические координаты 
\[
    I = \int_{a}^{0} r^3 dr \int_{0}^{\frac{\pi}{2}} \sin \theta d \theta \int_{0}^{\frac{\pi}{2}} d \varphi = 
    \frac{a^4}{4} \cdot 1 \cdot \frac{\pi}{2} = \frac{\pi a^4}{8}
\]
И исходный интеграл равен 
\[
    3I = \frac{3 \pi a^4}{8}
\] 
(((Тут ещё вычесть интегралы через боковые поверхности)))
\section{Т18}
Докажем лемму о том, что у замкнутой гладкой кривой существует точка, в которой касательная к кривой делит 
пространство на две части, одна из которых полностью содержит кривую. 
Так как кривая замкнута, то она ограничена, тогда рассмотрим расстояние от точки $(0, 0)$ до кривой, по теореме 1 
семестра максимальное и минимальное расстояние достигается. Тогда расстморим окружность $B((0, 0), \rho_{\max})$. 
Такая окружность касается кривой и кривая полностью лежит внутри окружности, в таком случае вся кривая лежит в 
одном полупространстве от касательной к окружности в этой точке. 

Далее перейдём к доказательству того, что вектор скорости делает только один поворот на замкнутой гладкой кривой без 
пересечений. Рассмотрим кривую $\gamma(t)$, где $t \in [0, 1]$ --- натуральный параметр. Тогда введём вектор 
\[
    e(t, s) = \frac{\gamma(s) - \gamma(t)}{\vert \gamma(s) - \gamma(t) \vert}
\]   
Рассмотрим треугольник $D = \left\{ (t, s) \mid 0 \leq t \leq s \leq 1 \right\} $. Так как кривая не имеет 
самопересечений, то на внутренности $D$ вектор $e$ всегда определён. Доопределим $e$ на диагонали по непрерывности: 
\[
    e(t, s) = \frac{\gamma(s) - \gamma(t)}{s - t} : \left\vert \frac{\gamma(s) - \gamma(t)}{ s - t } \right\vert \to \frac{\dot{\gamma(t)}}{\vert \dot{\gamma(t)} \vert } = \dot{\gamma(t)}
\]    
Аналогично показывается, что при $s \to 1, t \to 0, \, e \to -\gamma(0)$. 
Мы определили непрерывную функцию на треугольнике $D$, тогда так как $\vert e \vert = 1$, можем представить 
$e = (\cos \phi(t, s), \sin \phi(t, s))$. Вращение вектора скорости есть число оборотов $e$ вдоль диагонали $D$:  
\[
    \alpha = \phi(1, 1) - \phi(0, 0) = (\phi(1, 1) - \phi(0, 1)) + (\phi(0, 1) - \phi(0, 0))
\]
Согласно лемме найдётся точка, что кривая лежит по одну сторону от касательной, если определить эту точку за $t = 0$, 
тогда $\phi(1, 1) - \phi(0, 1) = \pm \pi$ и аналогично $\phi(0, 1) - \phi(0, 0) = \pm \pi$ тогда 
\[
    \alpha = \pm 2\pi 
\]
Что и требовалось доказать.

\section{Т.20}
Пусть $d \omega_1 = 0$ и $\omega_2 = d \omega$, 
Тогда 
\[
    d( (-1)^{\text{deg}} \omega_1 \wedge \omega) = 
    (-1)^{\text{deg}} d \omega_1 \wedge \omega + \omega_1 \wedge d \omega = 0 + \omega_1 \wedge \omega_2 = \omega_1 \wedge \omega_2
\]
Всё, форма точная.

\section{11.68}
Рассмотрим кривую заданную пересечением уравнений 
\[
    \begin{dcases}
        2x^2 - y^2 + z^2 = a^2 \\ 
        x + y = 0
    \end{dcases} \implies 
    \begin{dcases}
        x^2 + x^2 - y^2 + z^2 = a^2 \\ 
        x + y = 0
    \end{dcases} \implies 
    \begin{dcases}
        x^2 + z^2 = a^2 \\ 
        x + y = 0
    \end{dcases}
\]
Тогда введём параметризацию: 
\[
    \begin{dcases}
        x = a \cos \varphi \\
        z = a \sin \varphi \\
        y = - a \cos \varphi
    \end{dcases}
\]
где $\varphi \in [0, 2\pi]$. У такой параметризации другая ориентация, поэтому 
при переходе к ней поставим перед инетгралом знак минус.
\[
    \int_{L} z^3 dx + x^3 dy + y^3 dz = 
    -a^4 \int_{0}^{2\pi} (-\sin^4 \varphi + \cos^3 \varphi \sin \varphi - \cos^4 \varphi) d \varphi = 
    -a^4 \left( - \frac{3\pi}{4} - \frac{3\pi}{4}  \right) = 
    \frac{3\pi a^4}{2}
\]

\section{11.67}
\[
    \int_{L} (y^2 + z^2)dx + (z^2 + x^2)dy + (x^2 + y^2) dz = 
    2 \int_{S} (y - z) dy \wedge dz + z dz \wedge dx + x dx \wedge dy
\]
\[
    \begin{dcases}
        x^2 + y^2 + z^2 = 2ax \\
        x^2 + y^2 = 2bx
    \end{dcases} \implies 
    \begin{dcases}
        2x(b - a) = z^2 \\
        (x - b)^2 + y^2 = b^2
    \end{dcases}
\]
Возьмём карту $\varphi(t, p) = (\frac{p^2}{2(b - a)}, t, p)$, 
Из чего следует: $dx = \frac{p}{(b - a)} dp$.
Интеграл: 
\[
    2 \int_{S} (y - z) dy \wedge dz + z dz \wedge dx + x dx \wedge dy = 
    2 \int_{\left( \frac{p^2}{2(b - a)} - b \right)^2 + t^2 = b^2 } (t - p) dt \wedge dp + 0 + \frac{p^2}{2(b - a)} \cdot \frac{p}{(b - a)} dp \wedge dt
\]
Что равно: 
\[
    2 \int_{\left( \frac{p^2}{2(b - a)} - b \right)^2 + t^2 = b^2 } \left( t - p - \frac{p^3}{2(b - a)^2} \right)  dt dp
\]
((((НУ дальше в полярне и что-то получится...))))

\section{Т.24}
а) 
\[
    (\grad r)_\alpha = r^{\prime}(r) \frac{x_\alpha}{r} = \frac{x_\alpha}{r} \implies 
    \grad r = e_r
\]

б) 
\[
    (\grad \vert a, r \vert^2)_\sigma = 
    \partial_\sigma (\varepsilon_{\alpha \beta \gamma} a_\beta r_\gamma \varepsilon_{\alpha \mu \nu} a_\mu r_\nu) = 
    2\varepsilon_{\alpha \beta \gamma} a_\beta \delta_{\sigma \gamma} \varepsilon_{\alpha \mu \nu} a_\mu r_\nu = 
    2 \varepsilon_{\alpha \beta \sigma} \varepsilon_{\alpha \mu \nu} a_\beta a_\mu r_\nu = 
\]
Что равно 
\[
    2(\delta_{\beta \mu} \delta_{\sigma \nu} - \delta_{\beta \nu} \delta_{\sigma \mu}) a_\beta a_\mu r_\nu = 
    2 r_\sigma a^2 - 2a_\sigma (r, a)
\]
И тогда в векторной форме: 
\[
    \grad \vert a, r \vert^2 = 2 [r (a, a) - a (r, a)]
\]

\section{T.25}
б) 
\[
    \div [a, r] = (r, rot \, a) - (a, rot \, r) = 
    0 - 0 = 0
\]

\section{T.26}
б)
\[
    rot [a, r] = a \div r - (a \nabla) r = 
    3a - (a \nabla) r
\]

\end{document}