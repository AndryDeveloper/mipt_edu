\documentclass[12pt]{article}
\usepackage[T2A]{fontenc}
\usepackage[utf8]{inputenc}
\usepackage{multirow}
\usepackage{caption}
\usepackage{subcaption}
\usepackage{amsmath}
\usepackage{amssymb}
\usepackage{changepage}
\usepackage{graphicx}
\usepackage{float}
\usepackage[english,russian]{babel}
\usepackage{amsmath, amsfonts, amssymb, amsthm, mathtools}
\usepackage{xcolor}
\usepackage{array}
\usepackage{hyperref}
\usepackage{physics}
\usepackage[top = 1.5cm, left = 1.5 cm, right = 1.5 cm, bottom = 3 cm]{geometry}
\usepackage{import}
\usepackage{xifthen}
\usepackage{pdfpages}
\usepackage{transparent}

\newcommand{\incfig}[1]{
    \import{./figures/}{#1.pdf_tex}
}

\title{Теорвер 7.}
\author{Шахматов Андрей, Б02-304}
\date{\today}

\begin{document}
\maketitle
\tableofcontents

\section{Т3}
Начнём с сходимости по распределению:
\[
    F_{\xi_n} = 
    \begin{dcases}
        0, t < 0 \\
        (1 - p_n), 0 \leq t < 1 \\
        1, t \geq 1
    \end{dcases}
\]
Тогда в случае $1 - p_n \to 1$ $F_{\xi_n} \to F_{\xi = 0}$, из чего следует необходимым и достаточным условием 
слабой сходиомсти является $p_n \to 0$. Тогда так как сходимость является сходиомстью к константе, то такие же 
условия накладываются и на сходимость по вероятности. 
Исследуем сходиомсть в $L_2$: 
\[
    E \xi_n = p_n \to 0 \leftrightarrow p_n \to 0. 
\]    
Остаётся сходимость почти наверное. Я не знаю, но думаю, что тут нужно будет по определению показывать только.

\section{Т4}
\[
    \forall m \in \mathbb{Z} \, P(\xi_n = m) = 
    F_{\xi_n}\left( m + \frac{1}{2} \right) - F_{\xi_n}\left( m - \frac{1}{2} \right)
\]
Что стремится к
\[
    F_{\xi}\left( m + \frac{1}{2} \right) - F_{\xi}\left( m - \frac{1}{2} \right) = 
    F_\xi(m) - F_\xi(m-1) = P(\xi = m) \implies P(\xi_n = m) \to P(\xi = m).
\]
Аналогично повторяя в обратную сторону получаем доказательство в другую сторону.

\section{Т7}
а) \, Исследуем на слабую сходимость к $\xi \equiv 1$. 
Распределение $F_{\max \xi_1, \xi_2, \dots \xi_n} = \prod_{k=1}^{n} F_k$. 
Каждое из этих распределений равно $F_k = t$ на $(0, 1)$, тогда их произведение: 
\[
    \prod_{k=1}^n F_k = \prod_{k=1}^n t = t^n \to 
    \begin{dcases}
        0, t < 1 \\
        1, t \geq 1
    \end{dcases}
\]  
Функция распределения $\xi = 1$ имеет такой же вид, потому по теореме Александрова имеет место слабая 
сходимость. \\
б) \, Согласно задаче T2 слабая сходимость эквивалента сходмости по вероятности в случае сходимости к константе, 
поэтому сходимость по вероятности также имеет место. \\
в) \, Так как для последовательности выполнено, что $\eta_{n + 1} \geq \eta_{n}$, то верно: 
\[
    \left\{ \omega \mid \vert \eta_{n+1} - 1 \vert > \varepsilon \right\} \subset \left\{ \omega \mid \vert \eta_{n} - 1 \vert > \varepsilon \right\} \implies 
    \bigcup_{n \geq N}^{\infty} \left\{ \omega \mid \vert \eta_n - 1 \vert > \varepsilon \right\} = \left\{ \omega \mid \vert \eta_n - 1 \vert > \varepsilon \right\}. 
\] 
Тогда критерий сводимости почти наверное превращается в определение сходиомсти по мере, которую мы уже доказали.
г) \, Исследуем на сходимость в среднем, как известно: 
\[
    F_{\max \xi_1, \xi_2, \dots \xi_n} = t^n \implies \rho_{\max \xi_1, \xi_2, \dots \xi_n} = 
    n t^{n-1} I(0, 1)
\] 
Тогда достаточно доказать, что $E \max (\xi_1, \xi_2, \dots \xi_n) \to 1$: 
\[
    E \max (\xi_1, \xi_2, \dots \xi_n) = \int_{0}^{1} n t^{n-1} \,\mathrm{d}t = 1
\]


\end{document}